\section{Funktionentheorie}

Die \emph{Funktionentheorie} befasst sich mit komplexwertigen Funktionen
komplexer Variablen,
\begin{align*}
f: \C\to\C,\; z\mapsto f(z).
\end{align*}
Hier kommt es zu überraschenden Ergebnissen, denn man kann
$\C$ zwar mit $\R^2$ identifizieren und die mehrdimensionale Analysis für
Abbildungen,
\begin{align*}
\tilde{f}: \R^2\to\R^2,\; (x,y)\mapsto \tilde{f}(x,y)
\end{align*}
anwenden, da $\C$ jedoch ein Körper ist, haben wir viel mehr
Struktur zur Verfügung. So existiert für komplexe Funktionen ein
neuer Differenzierbarkeitsbegriff, die \emph{komplexe Differenzierbarkeit}.
Funktionen, die komplex differenzierbar sind, haben zahlreiche 
angenehmene Eigenschaften. Beispielsweise ist eine Funktion, die einmal komplex
differenzierbar ist, auch unendlich oft komplex differenzierbar und sogar
analytisch, also in eine Potenzreihe entwickelbar,
\begin{align*}
f(z) = \sum\limits_{n=0}^\infty a_n (z-z_0)^n.
\end{align*}
Dies ist im Allgemeinen nicht einmal für reelle $C^\infty$ Funktionen der Fall.

\subsection{Grundlangen}

\begin{defn}
\label{defn:2.1}
Die \emph{komplexen Zahlen} bestehen aus der Menge
\begin{align*}
\C = \R\times\R = \setdef{(a_1,a_2)}{a_1,a_2\in\R},
\end{align*}
mit den Verknüpfungen
\begin{align*}
&(a_1,a_2)+(b_1,b_2) = (a_1+b_1,a_2+b_2),\\
&(a_1,a_2)\cdot(b_1,b_2) = (a_1b_1-a_2b_2,a_1b_2+a_2b_1).\fishhere
\end{align*}
\end{defn}
\begin{bem}
\label{bem:2.2}
\begin{enumerate}
  \item $(\C,+,\cdot)$ ist ein Körper. Im Gegensatz zu $\R$ lässt sich dieser
  jedoch nicht mehr anordnen, weshalb wir nicht mehr auf Eigenschaften wie
  Monotonie oder Trichotomie zurückgreifen können.
  \item $\ph: \R\to\C,\;x\mapsto (x,0)$ ist ein injektiver
  Körperhomomorphismus, d.h. es gilt insbesondere
  \begin{align*}
  &\ph(a+b) = \ph(a)+\ph(b),\\
  &\ph(a\cdot b) = \ph(a)\cdot\ph(b).
  \end{align*}
  $\R$ kann also mit der Menge $\setdef{(a,0)\in\C}{a\in\R}$ identifiziert
  werden. In Zukunft schreiben wir $(a,0) := a$.
  \item Setze $i:=(0,1)$ dann folgt
  $i^2 = (-1,0) = -1$. Damit kann man $(a_1,a_2) =
  (a_1,0)+i(a_2,0)$ als $a_1+ia_2$ schreiben.\\
  Der \emph{Realteil} einer komplexen Zahl ist definiert als $\Re (a_1,a_2) :=
  a_1$. Analog dazu ist der \emph{Imaginärteil} $\Im(a_1,a_2) := a_2$
  definiert. Beide sind reell.
  \item \emph{Gauß'sche Zahlenebene}
  \begin{center}
\psset{unit=1cm}
\begin{pspicture}(-1,-1)(4.5,2.5)
 \psaxes[labels=none,ticks=none]{->}%
 (0,0)(-0.5,-0.5)(4,2)[\color{gdarkgray}$\Re$,-90][\color{gdarkgray}$\Im$,0]
 
 \psxTick(1){\color{gdarkgray}1}
 \psyTick(1){\color{gdarkgray}1}
 \psxTick(3.5){\color{gdarkgray}x}
 \psyTick(1.5){\color{gdarkgray}y}
 
 \psline[linestyle=dotted](3.5,0)(3.5,1.5)
 \psline[linestyle=dotted](0,1.5)(3.5,1.5)
 \psline[linecolor=darkblue,arrows=-*](0,0)(3.5,1.5)
 \psarc[arrows=->,linestyle=dashed](0,0){1.1}{0}{21.6}
 
 \rput(1.3,0.2){\color{gdarkgray}$\ph$}
 \rput[b]{29.7}(1.8,0.9){\color{gdarkgray}$\abs{z}$}
 
\end{pspicture}
\end{center}
\hfill\maphere
\end{enumerate}
\end{bem}

$\C$ lässt sich zwar nicht anordnen, dennoch können wir einen Abstand
definieren, sogar mehr, einen Absolutbetrag.

\begin{defn}
\label{defn:2.3}
\begin{enumerate}
  \item Die Abbildung $\overline{a_1+ia_2} = a_1-ia_2$ heißt \emph{komplexe
  Konjugation}.
  \item $\abs{z} = \sqrt{a_1^2+a_2^2} = \sqrt{z\overline{z}}$ heißt der
  \emph{Betrag} von $z=a_1+ia_2$.
  \item Für $z\neq 0$, lässt sich ein \emph{Argument} definieren als $\arg(z) =
  \ph$. Dabei ist $\ph$ eindeutig
  durch die Vereinbarung
  \begin{align*}
  -\pi \le \ph < \pi,\quad \cos \ph = \frac{a_1}{\abs{z}},\quad \sin \ph =
  \frac{a_2}{\abs{z}}.
  \end{align*}
  \item Die Darstellung $z=r(\cos \ph + i\sin\ph)$ mit $r=\abs{z}, \ph=\arg(z)$
  heißt \emph{Polardarstellung} von $z$.\fishhere
\end{enumerate}
\end{defn}

Der Absolutbetrag von $z$ ist eine Norm und induziert dadurch ebenfalls eine
Metrik, wie der folgende Satz zeigt.

\begin{prop}
\label{prop:2.4}
Für $z,z_1,z_2\in\C$ gilt
\begin{enumerate}
  \item $\abs{z} \ge 0$ und $\abs{z} = 0 \Leftrightarrow z = 0$.
  \item $\abs{z_1}\abs{z_2} = \abs{z_1z_2}$.
  \item $\abs{z_1+z_2} \le  \abs{z_1} + \abs{z_2}$.\fishhere
\end{enumerate}
\end{prop}
\begin{proof}
Folgt durch direktes Rechnen.\qedhere
\end{proof}

Somit ist $(\C,+,\cdot,\abs{\cdot})$ ein \emph{bewerteter Körper}.

\begin{cor}
\label{prop:2.5}
$d: \C\times\C \to \R,\quad (z_1,z_2) \mapsto \abs{z_1-z_2}$ ist eine Metrik auf
$\C$.\fishhere
\end{cor}

Damit steht uns auf $\C$ ein Konvergenz- und ein Stetigkeitsbegriff zur
Verfügung, die wir nun genauer untersuchen wollen.

\begin{prop}
\label{prop:2.6}
Seien $f,g : D\subseteq \C\to\C$ stetig in $z_0\in\C$, dann sind $f+g$ und
$f\cdot g$ stetig in $z_0$, und falls $g(z_0)\neq 0$ ist auch $f/g$ stetig
in $z_0$.\fishhere
\end{prop}
\begin{proof}
Analog zum Beweis in $\R$.\qedhere
\end{proof}

Wie wir im letzten Semester gesehen haben, sind die Funktionen $e^x,\;\sin x$
und $\cos x$ reell analytisch, ihre Taylorreihen konvergieren und stimmen
mit den Funktionen überein. Mithilfe dieser Reihen, kann man nun $e^x,\;\sin x$
und $\cos x$ auch auf $\C$ fortsetzen.

\begin{defn}
\label{defn:2.7}
Die Exponentialfunktion, Sinus und Cosinus sind auf $\C$ definiert als
  \begin{align*}
  &e^z := \sum\limits_{n=0}^\infty \frac{z^n}{n!}, &&z\in\C,\\
  &\sin(z) := \sum\limits_{n=0}^\infty (-1)^n\frac{z^{2n+1}}{(2n+1)!}, && z\in
  \C,\\
  &\cos(z) := \sum\limits_{n=0}^\infty (-1)^n\frac{z^{2n}}{(2n)!}, && z\in
  \C.
  \end{align*}
\end{defn}

Ob diese Reihen für irgendein $z\in\C\setminus\R$ konvergieren, ist dadurch noch
nicht klar. Die Konvergenzkriterien für Potenzreihen geben aber Aufschluss.

\begin{cor}
\label{prop:2.8}
\begin{enumerate}
\item 
$e^z$ konvergiert auf ganz $\C$ und ist dort ebenfalls $\C$ stetig.
\item
$\sin z$ und $\cos z$ konvergieren auf ganz $\C$ und sind dort ebenfalls 
stetig.
\item Es gilt die Identität,
$e^ze^w = e^{z+w}$.\fishhere
\end{enumerate}
\end{cor}

\begin{proof}
\begin{enumerate}
  \item Setzt man $e^z = \sum\limits_{n=0}^\infty a_n z^n$, so ergibt sich
\begin{align*}
R:=
\lim\limits_{n\to\infty}\frac{\abs{a_n}}{\abs{a_{n+1}}}
= \lim\limits_{n\to\infty}\abs{\frac{(n+1)!}{n!}}
=\lim\limits_{n\to\infty}\abs{n+1} = \infty.
\end{align*}
Nach Satz \ref{prop:1.27} ist damit $z\mapsto e^z$ stetig auf $\C$.
\item Man kann die Potenzreihe von $\sin z$ schreiben als
\begin{align*}
\sin z = z\sum\limits_{n=0}^\infty \frac{(-1)^n}{(2n+1)!}(z^2)^n
\end{align*}
und dann die Folge betrachten
\begin{align*}
R:=\lim\limits_{n\to\infty} \abs{(-1)\frac{(2n+3)!}{(2n+1)!}}
= \lim\limits_{n\to\infty} (2n+3)(2n+2) = \infty.
\end{align*}
Also ist die Reihe konvergent für alle $z^2$ und damit auch für alle $z\in\C$.
\item  Mit Hilfe des \emph{Cauchyprodukts} folgt,
\begin{align*}
e^ze^w &= \left(\sum\limits_{n=0}^\infty \frac{z^n}{n!} \right)
\left(\sum\limits_{k=0}^\infty \frac{w^k}{k!} \right)
\overset{\text{CP}}{=} \sum\limits_{m=0}^\infty\sum\limits_{l=0}^m
\frac{z^l}{l!}\frac{w^{m-l}}{(m-l)!}
\\ &= \sum\limits_{m=0}^\infty \frac{1}{m!} \sum\limits_{l=0}^m
\frac{z^l}{l!}\frac{w^{m-l}}{(m-l)!}m! = \sum\limits_{m=0}^\infty \frac{1}{m!}
\underbrace{\sum\limits_{l=0}^m {m \choose l} z^lw^{m-l}}_{\text{Binomischer
Lehrsatz}}\\ &= \sum\limits_{m=0}^\infty \frac{1}{m!}(z+w)^m = e^{z+w}.\qedhere
\end{align*}
\end{enumerate}
\end{proof}

\begin{bem}
\label{bem:2.9}
Die Taylorreihen von $\sin z,\;\cos z$ und $e^z$ stimmen für $z\in\R$ mit den
hier betrachteten Funktionen überein.\maphere
\end{bem}

\begin{cor}
\label{prop:2.10}
\begin{enumerate}
\item
$e^{iz} = \cos z + i\sin z$.
\item
Für die Polardarstellung von $z\in\C$ mit $z = r(\cos\ph+i\sin\ph) =
re^{i\ph}$, gilt
\begin{align*}
&z^n = r^ne^{in\ph},\\
&z_1z_2 = r_1r_2e^{i(\ph_1+\ph_2)},\\
&\frac{z_1}{z_2} = \frac{r_1}{r_2}e^{i(\ph_1-\ph_2)}.\fishhere
\end{align*}
Hier ist $\ph_1\pm\ph_2$ so zu verstehen, dass das Ergebnis wieder aus
$[-\pi,\pi)$ ist.\fishhere
\end{enumerate}
\end{cor}

\begin{proof}
\begin{enumerate}
\item
Da $e^z$ auf $\C$ absolut konvergent ist, erlaubt der Umordnungssatz die
Aufspaltung in Summanden für gerade und ungerade Indizes,
\begin{align*}
e^{iz} &= \sum\limits_{n=0}^\infty \frac{i^nz^n}{n!} = 
\sum\limits_{k=0,n=2k}^\infty i^{2k}\frac{z^{2k}}{(2k)!}
+\sum\limits_{k=0,n=2k+1}^\infty i^{2k+1}\frac{z^{2k+1}}{(2k+1)!}\\
&=\sum\limits_{k=0}^\infty (-1)^k \frac{z^{2k}}{(2k)!} +
i\sum\limits_{k=0}^\infty (-1)^k \frac{z^{2k+1}}{(2k+1)!} = \cos z + i\sin z.
\end{align*}
\item
Folgt durch direktes Rechnen.\qedhere  
\end{enumerate}
\end{proof}

\begin{defn}
\label{defn:2.11}
\begin{enumerate}
  \item Seien $z_1,z_2\in\C$, $\gamma\in C^k([a,b] \to \C)$ mit $\gamma(a) =
  z_1$, $\gamma(b) = z_2$, dann heißt $\gamma$ eine \emph{$C^k$-Kurve von $z_1$
  nach $z_2$}.
  \item Sei $O\subseteq \C$ offen, $f: O\to\C$ stetig, $\gamma\in
  C^1([a,b]\to O)$, dann heißt
  \begin{align*}
  \int\limits_{\gamma} f(z)\dz = \int\limits_a^b
  f\circ\gamma(t)\ \dot{\gamma}(t)\dt,
  \end{align*}
\emph{Integral über $f$ längs $\gamma$}.\fishhere
\end{enumerate}
\end{defn}

\begin{bem}[Bemerkungen]
\label{bem:2.12}
\begin{enumerate}[label=\arabic{*}.)]
  \item Die rechte Seite kann als Summe reeller Integrale verstanden werden
  \begin{align*}
  \int\limits_a^b \Re \left(f\circ\gamma(t)\ \dot{\gamma}(t) \right)\dt + i 
  \int\limits_a^b \Im \left(f\circ\gamma(t)\ \dot{\gamma}(t) \right)\dt. 
  \end{align*}
  \item Ist $\tilde{\gamma}$ eine andere Kurve mit folgenden Eigenschaften
  \begin{enumerate}
    \item $\tilde{\gamma}(s) = \gamma\circ\ph(s)$, für $\tilde{a}\le s\le
    \tilde{b}$,
    \item $\ph\in C^1([\tilde{a},\tilde{b}]\to[a,b])$ bijektiv,
    \item $\ph(\tilde{a}) = a,\;\ph(\tilde{b}) = b$,
    \end{enumerate}
      dann folgt aus der Substitutionsregel für relle
    Integrale mit $t=\ph(s)$
    \begin{align*}
    \int\limits_a^b f\circ\gamma(t)\ \dot{\gamma}(t)\dt
    &= \int\limits_{\ph^{-1}(a) = \tilde{a}}^{\ph^{-1}(b) = \tilde{b}}
    f\circ\gamma\circ\ph(s)\ \dot{\gamma}\circ\ph(s)\ \dot{\ph}(s)\ds\\
    &= \int\limits_{\tilde{a}}^{\tilde{b}}
    f\circ\tilde{\gamma}(s)\ \dot{\tilde{\gamma}}(s)\ds.
    \end{align*}
  \item Ist $-\gamma: [a,b]\to \C,\;t\mapsto \gamma(a+b-t)$, so gilt
  \begin{align*}
  \int\limits_{-\gamma} f(z)\dz = -\int\limits_a^b
  f\circ\gamma(a+b-t)\ \dot{\gamma}(a+b-t)\dt
  \end{align*}
Setze $s=a+b-t \Rightarrow \ds = -\dt$, dann folgt
\begin{align*}
\ldots = \int\limits_b^a f\circ\gamma(s)\dot{\gamma}(s)\ds = -\int\limits_a^b
f\circ\gamma(s)\dot{\gamma}(s)\ds = -\int\limits_{\gamma}f(z)\dz.
\end{align*}
  \item Jede Kurve kann so parametrisiert werden, dass $a=0,\;b=1$.
  \item Ist $\gamma$ stückweise $C^1$-Kurve und eine Teilung von $[a,b]$ gegeben
  mit $a = t_0 < t_1 < \ldots < t_k = b$, dann gilt
  \begin{align*}
  &\gamma_j := \gamma\big|_{[t_{j-1},t_j]} \in C^1([t_{j-1},t_j]\to\C),\\
  &\int\limits_{\gamma} f(z)\dz := \sum\limits_{j=1}^n \int\limits_{\gamma_j}
  f(z)\dz.\maphere
  \end{align*}
\end{enumerate}
\end{bem}

\begin{bemn}
Bei Funktionen von $\R$ nach $\C$ kann man die Differentiation des Real- und
des Imaginärteils getrennt betrachten.
\begin{align*}
\dot{\gamma}(t) = \left(\Re \gamma(t) \right)' + i\left(\Im \gamma(t)
\right)'.\maphere
\end{align*}
\end{bemn}

\begin{bemn}[Vereinbarung]
Von nun an sei $O$ stets eine offene Teilmenge der komplexen Zahlen.
\begin{align*}
f:\C\opento \C,
\end{align*}
bezeichnet, dass $f$ von einer offenen Menge der komplexen
Zahlen in die komplexen Zahlen abbildet.
\end{bemn}

\begin{bsp}
\label{prop:2.13}
Sei $\gamma:[a,b]\to O, f\in C(O\to\C)$, so dass
\begin{align*}
\int\limits_\gamma f(z)\dz = \int\limits_a^b f\circ \gamma(t)\
\dot{\gamma}(t)\dt.
\end{align*}
\begin{enumerate}
  \item $\gamma(t) = (1+i)t$ für $0\le t\le 2$, und $f(z) = z^3$.
  \begin{align*}
  \int\limits_\gamma z^3\dz &= \int\limits_0^2 ((1+i)t)^3(1+i) \dt
  = (1+i)^4\int\limits_0^2 t^3 \dt \\
  &= \frac{(1+i)^4}{4}t^4\big|_0^2 = (1+i)^44.
  \end{align*}
  \item $\gamma(t) = e^{it}$ für $0\le t\le 2\pi$, und $f(z) = z^{-1}$ für 
  $z\in\C\setminus\{ 0\}$.
  \begin{align*}
  \int\limits_\gamma z^{-1}\dz = \int\limits_0^{2\pi}e^{-it}ie^{it} \dt =
  2\pi i.\bsphere
  \end{align*}
\end{enumerate}
\end{bsp}

\begin{bemn}
Die Abschätzung
\begin{align*}
\abs{\int\limits_\gamma f(z) \dz} \le \int\limits_\gamma \abs{f(z)}\dz,
\end{align*}
ist im Allgemeinen falsch, sie gilt ja nicht einmal im reellen Fall, wenn
$\gamma = \tilde{\gamma}\circ\ph$, mit $\tilde{\gamma}: [a,b]\to \R^n$ und
$\ph:[0,1]\to[a,b]$ streng monoton fallend.\maphere
\end{bemn}

\begin{prop}
\label{prop:2.14}
Sei $\gamma\in C^1([a,b]\to O)$, $f\in C(O\to C)$, dann gilt
\begin{align*}
\abs{\int\limits_\gamma f(z) \dz} \le \int\limits_a^b \abs{f\circ
\gamma(t)}\abs{\dot{\gamma}(t)}\dt \le \max\limits_{z\in\im\gamma}
\abs{f(z)} \int\limits_a^b \abs{\dot{\gamma}(t)}\dt.\fishhere
\end{align*}
\end{prop}
\begin{proof}
Für reelle Zahlen gilt $\abs{r} = (\sign r)\cdot r$, wir multiplizieren die
Zahl mit ihrem Vorzeichen. Für komplexe Zahlen können wir nicht nur
zwischen $\pm$ unterscheiden, sondern zwischen jeder Richtung in der komplexe
Zahlenebene. Den Betrag einer komplexen Zahl erhalten wir, indem wir die Zahl
auf die reelle Achse drehen, ohne ihren Abstand zur $0$ zu verändern:
\begin{align*}
\abs{\int\limits_\gamma f(z) \dz} = e^{i\ph}\int\limits_\gamma f(z) \dz,
\end{align*}
wobei wir $\ph = - \arg \int\limits_\gamma f(z) \dz$ setzen.

Mit der Monotonie des reellen Integrals und dem Mittelwertsatz erhalten wir
\begin{align*}
\abs{\int\limits_\gamma f(z) \dz} &= e^{i\ph}\int\limits_a^b f\circ\gamma(t)
\dot{\gamma}(t) \dt \\ &= \int\limits_a^b \underbrace{\Re\left(e^{i\ph}
f\circ\gamma(t)
\dot{\gamma}(t) \right)}_{\le \abs{f\circ\gamma(t)\dot{\gamma}(t)}}
 \dt + i\int\limits_a^b \underbrace{\Im  \left(e^{i\ph}
f\circ\gamma(t) \dot{\gamma}(t) \right)}_{=0} \dt
\\ &\le \int\limits_a^b \abs{f\circ\gamma(t)}\abs{\dot{\gamma}(t)}
\le \max\limits_{z\in\gamma([a,b])} \abs{f(z)}  \int\limits_a^b
\abs{\dot{\gamma}(t)} \dt.\qedhere
\end{align*}
\end{proof}

\begin{defn}
\label{defn:2.15}
Sei $\gamma\in C^1([a,b]\to O)$, dann ist die Länge von $\gamma$ definiert als
\begin{align*}
L(\gamma) := \int\limits_a^b \abs{\dot{\gamma}(t)}\dt.\fishhere
\end{align*}
\end{defn}

Damit kann man Satz \ref{prop:2.14} auch so formulieren
\begin{propn}
Sei $\gamma\in C^1([a,b]\to O)$, $f\in C(O\to C)$, dann gilt
\begin{align*}
\abs{\int\limits_\gamma f(z) \dz} \le \max\limits_{z\in \gamma([a,b])}
\abs{f(z)} L(\gamma).\fishhere
\end{align*}
\end{propn}

\begin{bsp}
\label{bsp:2.16}
Sei $\gamma(t) = e^{it}$ mit $0\le t\le 2\pi$, dann gilt
\begin{align*}
L(\gamma) = \int\limits_0^{2\pi} \abs{\dot{\gamma(t)}}\dt =
\int\limits_0^{2\pi} \abs{ie^{it}}\dt
= \int\limits_0^{2\pi} \sqrt{\cos^2 t+\sin^2 t}\dt =  2\pi.
\end{align*}
\begin{enumerate}
  \item 
\begin{align*}
\abs{\int\limits_\gamma \frac{1}{z}\dz} \le
\max\limits_{\abs{z}=1}\;\abs{\frac{1}{z}} L(\gamma) = 2\pi.
\end{align*}
\item
\begin{align*}
\abs{\int\limits_\gamma z\dz} \le \max\limits_{\abs{z}=1}\abs{z}\;
L(\gamma) = 2\pi.
\end{align*}
Wir wissen aber bereits, dass $\int\limits_\gamma z\dz = 0$. Die Abschätzung
ist also oft sehr ungenau.\bsphere
\end{enumerate}
\end{bsp}

In der Funktionentheorie ist es oft geschickt, die Funktionen zu verschieben, zu
dehnen oder zu stauchen. Dazu benötigen wir den Begriff der \emph{Homotopie}.

\begin{defn}
\label{defn:2.17}
Seien $\gamma_1,\gamma_2\in C^k([0,1]\to O)$ ($k\ge0$). Dann heißen
$\gamma_1,\gamma_2$ \emph{$C^k$-homotop in $O$}, falls es eine Abbildung
$\phi\in C^k([0,1]\times[0,1]\to O)$ gibt mit $\phi(\cdot,0) = \gamma_1$ und
$\phi(\cdot,1) = \gamma_2$, die einer der folgenden Eigenschaften genügt:
\begin{enumerate}
  \item $\phi(0,s) = \phi(0,0),\; \phi(1,s) = \phi(1,0)$ für $s\in[0,1]$,\\ d.h.
  die ``Zwischenkurven'' $t\mapsto \phi(t,s)$ für festes $s$ haben alle den
  selben Anfangs und Endpunkt $\phi(0,s) = \gamma_1(0) = \gamma_2(0),
  \; \phi(1,s) = \gamma_1(1) = \gamma_2(1)$.
  \item $\phi(0,s) = \phi(1,s)$ für alle $s\in[0,1]$, d.h. alle Kurven sind geschlossen.
\end{enumerate}
Homotop sein ist eine Äquivalenzrelation und wir schreiben
$\gamma_1\sim\gamma_2$. $\phi$ heißt \emph{Homotopie} zwischen $\gamma_1$ und
$\gamma_2$. Ein geschlossener Weg $\gamma$ heißt \emph{$C^k$-nullhomotop}, falls
$\gamma$ $C^k$-homotop zu einem konstanten Weg ist.\fishhere
\end{defn}

\begin{bsp}
\begin{enumerate}[label=(\alph{*})]
  \item\label{bsp:2.18:1} Sind $\gamma_1,\gamma_2\in C([0,1]\to\C)$,
  mit $\gamma_1(0) = \gamma_2(0)$ und $\gamma_1(1) = \gamma_2(1)$, dann ist $\gamma_1\sim\gamma_2$:
  \begin{align*}
  \phi(t,s) &:= (1-s)\gamma_1(t) + s\gamma_2(t),\\
  \phi(0,s) &= (1-s)\gamma_1(0) + s\gamma_2(0) = \gamma_1(0) = \phi(0,0),\\
  \phi(1,s) &= (1-s)\gamma_1(1) + s\gamma_2(1) = \gamma_2(1) = \phi(1,0).
  \end{align*}
  \begin{center}
\psset{unit=0.5cm}
\begin{pspicture}(-1.5,-1.5)(7.5,5.5)

 %\psline[linecolor=framecolor](-4.5,-4.5)(-4.5,4.5)(4.5,4.5)(4.5,-4.5)(-4.5,-4.5)
 
 \psaxes[labels=none,ticks=none,linecolor=gdarkgray,tickcolor=gdarkgray]{->}%
 (0,0)(-1,-1)(7,5)
 
	
 \psbezier[linecolor=gdarkgray,linestyle=dashed]%
	(1,1)(2,1.8)%
	(4,0)(5,3)
	
	
 \psbezier[linecolor=gdarkgray,linestyle=dashed]%
	(1,1)(2,2)%
	(4,1)(5,3)
	
	
 \psbezier[linecolor=gdarkgray,linestyle=dashed]%
	(1,1)(1.5,1.5)%
	(4,3)(5,3)
	
 \psbezier[linecolor=gdarkgray,linestyle=dashed]%
	(1,1)(1.2,2)%
	(3.8,4)(5,3)
	
	
 \psbezier[linecolor=darkblue]%
	(1,1)(2,1.5)%
	(4,-0.5)(5,3)
	
 \psbezier[linecolor=darkblue,arrows=*-*]%
	(1,1)(1,3)%
	(3.5,5)(5,3)
	
	\rput(2,0.5){\color{gdarkgray}$\gamma_1$}
	\rput[b](4,4){\color{gdarkgray}$\gamma_2$}

\end{pspicture}
\end{center}
  \item Ist $\gamma_1$ geschlossen und befinden sich im Einschluss von
  $\gamma_1$ keine ``Löcher'', so ist $\gamma_1$ nullhomotop. Setze
  $\gamma_2(t) := z_0$ für $z_0\in\C$ beliebig aber fest und wähle $\phi$ wie in \ref{bsp:2.18:1}, dann gilt
  \begin{align*}
  &\phi\in C^1, \phi(\cdot,0) = \gamma_1, \phi(\cdot,1) = \gamma_2,\\
    &\phi(0,s) = (1-s)\gamma_1(0) + s\gamma_2(0) =
    (1-s)\gamma_1(1) + s z_0 = \phi(1,s).
  \end{align*}
  \begin{center}
\psset{unit=0.5cm}
\begin{pspicture}(-1.5,-1.5)(7.5,5.5)

 %\psline[linecolor=framecolor](-4.5,-4.5)(-4.5,4.5)(4.5,4.5)(4.5,-4.5)(-4.5,-4.5)
 
 \psaxes[labels=none,ticks=none,linecolor=gdarkgray,tickcolor=gdarkgray]{->}%
 (0,0)(-1,-1)(7,5)
 
 
 \pscircle[linecolor=gdarkgray,linestyle=dashed](3.5,1.5){0.7}
 
 \pscircle[linecolor=gdarkgray,linestyle=dashed](4.2,1.5){0.35}
 
 \pscircle[linecolor=gdarkgray,linestyle=dashed](4.6,1.5){0.17}
 
 \pscircle[linecolor=darkblue](2.3,1.5){1.3}
 
 \psdot[linecolor=darkblue](4.77,1.5)
 
 \rput(1,2.8){\color{gdarkgray}$\gamma_1$}
 
 \rput(5.4,1.5){\color{gdarkgray}$\gamma_2$}
\end{pspicture}
\end{center}
\item $\gamma_1\sim\gamma_3,\gamma_1\nsim\gamma_2,\gamma_2\nsim\gamma_3$.
\begin{center}
\psset{unit=0.5cm}
\begin{pspicture}(-3.5,-3.5)(3.5,3.5)

 %\psline[linecolor=framecolor](-4.5,-4.5)(-4.5,4.5)(4.5,4.5)(4.5,-4.5)(-4.5,-4.5)
 
 \psaxes[labels=none,ticks=none,linecolor=gdarkgray,tickcolor=gdarkgray]{->}%
 (0,0)(-3,-3)(3,3)
 
 \pscircle(0,0){0.2}
 
%  
%  \psbezier[linecolor=gdarkgray]%
% 	(-2,-2)(1,-2)%
% 	(1,2)(2,2)
% 	
 \psbezier[linecolor=gdarkgray]%
	(-2,-2)(2,-2)%
	(2,1)(2,2)
	
 \psbezier[linecolor=gdarkgray]%
	(-2,-2)(-2,0.5)%
	(2,0.5)(2,2)
% 	
  \psbezier[linecolor=gdarkgray]%
 	(-2,-2)(-3,0.5)%
 	(0.5,1.5)(2,2)
 	
 	\rput(1,0.3){\color{gdarkgray}$\gamma_1$}
 	
 	\rput(1.8,-1.8){\color{gdarkgray}$\gamma_2$}
 	
 	\rput(1.2,2.2){\color{gdarkgray}$\gamma_3$}
\end{pspicture}
\end{center}
\item $\gamma$ umläuft $0$ einmal bzw. zweimal.
\begin{center}
\psset{unit=0.6cm}
\begin{pspicture}(0,-3.046538)(6.0,3.066538)
\rput(3.0,-0.046538047)%
{\psaxes[linecolor=gdarkgray,linewidth=0.028222222,labels=none,ticks=none,ticksize=0.10583334cm]{->}(0,0)(-3,-3)(3,3)}
\pscircle[linewidth=0.028222222,dimen=outer](3.0,-0.046538047){0.2}
\psbezier[linecolor=gdarkgray,linewidth=0.04](4.365889,1.613462)(4.18918,0.62919885)(1.4043703,2.7922022)(0.9258889,1.9134619)(0.44740748,1.0347217)(0.6543075,1.2952845)(1.1458889,0.45346195)(1.6374702,-0.3883606)(1.1058145,-1.0067931)(2.065889,-1.286538)(3.0259633,-1.5662829)(3.7794652,-0.77259237)(3.945889,0.21346195)(4.1123123,1.1995163)(2.0115757,1.5003859)(2.6458888,2.273462)(3.2802022,3.046538)(4.542598,2.5977252)(4.365889,1.613462)
\rput(4.3872952,2.603462){\color{gdarkgray}$\gamma$}
\end{pspicture}
\begin{pspicture}(0,-3.0)(6.0,3.0)
\rput(3.0,0.0){\psaxes[linecolor=gdarkgray,linewidth=0.028222222,labels=none,ticks=none,ticksize=0.10583334cm]{->}(0,0)(-3,-3)(3,3)}
\pscircle[linewidth=0.028222222,dimen=outer](3.0,0.0){0.2}
\rput(4.027295,1.89){\color{gdarkgray}$\gamma$}
\psbezier[linecolor=gdarkgray,linewidth=0.04](3.945889,1.42)(3.4442725,2.2850902)(2.2621074,2.2178762)(1.5458889,1.58)(0.82967037,0.9421239)(0.81418884,-0.82049483)(1.5658889,-1.48)(2.317589,-2.1395051)(3.5857859,-1.9124936)(4.025889,-1.48)(4.465992,-1.0475063)(4.2858887,-0.56)(3.945889,-0.02)(3.6058888,0.52)(3.045889,0.98)(2.5858889,0.48)(2.1258888,-0.02)(2.465889,-0.68)(3.185889,-0.68)(3.9058888,-0.68)(4.4475055,0.5549099)(3.945889,1.42)
\end{pspicture}
\end{center}
\hfill\bsphere
\end{enumerate}
\end{bsp}

\begin{prop}[Definition und Satz]
\label{prop:2.19}
Sei $f: \C\opento \C,\; z_0\in \C$, dann sind folgende Aussagen äquivalent.
\begin{enumerate}
  \item $f$ ist \emph{komplex differenzierbar in $z_0$}.
  \item $f'(z) := \lim\limits_{z\to z_0} \frac{f(z) - f(z_0)}{z-z_0}$ existiert.
  \item Es existiert eine Zahl $f'(z_0)\in\C$, so dass
  \begin{align*}
  f(z) = f(z_0) + f'(z_0)(z-z_0) + o(z-z_0).
  \end{align*}
\end{enumerate}
$f'(z_0)$ heißt \emph{Ableitung} von $f$ in $z_0$. Ist $f$ für jedes $z\in O$
differenzierbar, so heißt $f$ differenzierbar auf $O$.\fishhere
\end{prop}

\begin{prop}
\label{prop:2.20}
Seien $f,g : O \to \C$ in $z_0$ differenzierbar, dann gilt in $z_0$.
\begin{enumerate}
  \item $f,g$ sind stetig,
  \item $(f+g)' = f' + g'$,
  \item $(f\cdot g)' = f'\cdot g + f\cdot g'$,
  \item falls $g(z_0)\neq 0$ ist $\left(\frac{f}{g}\right)' =
  \frac{f'g-fg'}{g^2}$,
  \item $\left(f\circ g \right)' = (f'\circ g)\ g'$.\fishhere
\end{enumerate}
\end{prop}

\begin{proof}
Die Beweise funktionieren analog zum reellen Fall.\qedhere
\end{proof}

\begin{cor}
\label{prop:2.21}
Sei $f : \C\opento \C$  in $z\in\C$ differenzierbar.
\begin{enumerate}
  \item Ist $f(z)$ konstant, folgt $f'(z) = 0$.
  \item Sei $f(z) = z^k,\; k\in\N$, dann ist $f'(z) = kz^{k-1}$.
  \item Polynome und gebrochenrationale Funktionen sind
  differenzierbar.\fishhere
\end{enumerate}
\end{cor}

\begin{prop}
\label{prop:2.22}
Sei $F : O \to \C$ differenzierbar in $O$, $\gamma\in C^1([a,b]\to O)$. Ist
$F' = f$, dann gilt
\begin{align*}
\int\limits_\gamma f(z) \dz = F\circ \gamma(b) - F\circ\gamma(a).\fishhere
\end{align*}
\end{prop}
\begin{proof}
Dies kann auf den Hauptsatz für reelle Funktionen zurückgeführt werden
\begin{align*}
\int \limits_\gamma f(z) \dz &= \int\limits_a^b
f\circ\gamma(t)\ \dot{\gamma}(t)\dt = \int\limits_a^b (F\circ \gamma)'(t) \dt
\\
&= \int\limits_a^b \left[ (\Re F\circ \gamma)'(t) + i(\Im F\circ
\gamma)'(t)\right] \dt  \\ &= \int\limits_a^b (\Re F\circ \gamma)'(t) \dt + i
\int\limits_a^b (\Im F\circ \gamma)'(t) \dt.
\end{align*}
Wendet man nun den Hauptsatz an, ergibt sich
\begin{align*}
\ldots &= \Re F\circ \gamma(b) - \Re F\circ \gamma(a) + i  \Im F\circ \gamma(b)
- i\Im F\circ \gamma(a) \\
&= F\circ\gamma(b) - F\circ\gamma(a).\qedhere
\end{align*}
\end{proof}

\begin{cor}
\label{prop:2.23}
\begin{enumerate}
  \item Besitzt $f$ eine Stammfunktion und ist $\gamma$ geschlossen, dann gilt
  $\int\limits_\gamma f(z)\dz = 0$.
  \item Die Funktion $f:\C\setminus\{0\}\to\C,\; z\mapsto z^{-1}$ besitzt keine
  Stammfunktion.\fishhere
\end{enumerate}
\end{cor}
\begin{proof}
\begin{enumerate}
  \item $\gamma$ ist geschlossen, also ist $\gamma(a) = \gamma(b)$ und damit
  gilt,
  \begin{align*}
  \int\limits_\gamma f(z)\dz = F\circ\gamma(b) - F\circ\gamma(a) = 0.
  \end{align*}
  \item Die Kurve $\gamma: t\mapsto e^{it}$ für $t\in[0,2\pi]$ ist geschlossen
  aber wir haben bereits gesehen, dass
  \begin{align*}
  \int\limits_\gamma z^{-1}\dz = 2\pi i\neq 0.
  \end{align*}
  Die Funktion $\ln z$ mit $(\ln z)' = z^{-1}$ ist lediglich eine lokale
  Stammfunktion. Sie ist auf $\C$ nicht mehr eindeutig sondern besitzt
  verschiedene Zweige, die $2\pi i$ auseinanderliegen.\qedhere
\end{enumerate}
\end{proof}

\subsection{Holomorphie und Analytizität}

\begin{bem}[Vereinbarung]
\label{bem:2.24}
Zu $f: O\to\C$ setzen wir
\begin{align*}
\tilde{O} &:= \setdef{(x,y)\in\R^2}{x+iy \in O},\\
(u,v) &: \tilde{O}\to\R^2,\;(x,y)\mapsto (u(x,y),v(x,y)) \\ &:= (\Re f(x+iy),\Im
f(x+iy)),
\end{align*}
wir interpretieren $f$ also als Abbildung von $\R^2$ nach $\R^2$.\maphere 
\end{bem}

\begin{prop}
\label{prop:2.25}
Sei $f:O\to\C$, dann ist äquivalent:
\begin{enumerate}[label=(\roman{*})]
  \item\label{prop:2.25.1} $f$ ist differenzierbar in $O$.
  \item\label{prop:2.25.2} $f$ ist stetig in $O$ und für je zwei $C^1$ Kurven
  in $O$ gilt
  \begin{align*}
  \gamma_1\sim\gamma_2 \Rightarrow \int\limits_{\gamma_1} f(z)\dz =
  \int\limits_{\gamma_2} f(z)\dz.
  \end{align*}
  \item\label{prop:2.25.3} $f$ ist stetig in $O$ und für jede
  $C^1$-nullhomotope Kurve $\gamma$ in $O$ gilt,
  \begin{align*}
  \int\limits_\gamma f(z) \dz = 0.
  \end{align*}
  \item\label{prop:2.25.4} Für jedes $z_0\in O$ existiert ein $R>0$ und eine
  Potenzreihe, so dass gilt
  \begin{align*}
  f(z) = \sum\limits_{n=0}^\infty a_n (z-z_0)^n,
  \end{align*}
  für $\abs{z-z_0} < R$.
  \item\label{prop:2.25.5} $f$ ist beliebig oft differenzierbar in $O$.
  \item\label{prop:2.25.6} Die Abbildung $(u,v)$ ist $C^1(\tilde{O}\to\R^2)$
  und erfüllt die
  \emph{Cauchy-Riemannschen Differenzialgleichungen}
  \begin{align*}
  u_x = v_y, \quad u_y = - v_x\quad \text{ in }\tilde{O}.\fishhere
  \end{align*}
\end{enumerate}
\end{prop}
Der Beweis dieses Satzes wird sich über das ganze Kapitel erstrecken.

\begin{defn}
\label{defn:2.26}
Erfüllt $f: O\to\C$ eine, und damit alle Bedingungen, aus Satz \ref{prop:2.25},
so heißt $f$ \emph{holomorph} oder \emph{analytisch} in $O$.
\end{defn}

\begin{prop}[Cauchyscher Integralsatz für Bilder von Rechtecken]
\label{prop:2.27}
Sei $f: O\to\C$ differenzierbar, $R$ eine achsenparallele, abgeschlossene
Rechteckfläche in $\R^2$,\\ $\ph\in C^1(R\to O)$ und $\gamma$ eine geschlossene
stückweise $C^1$ Randkurve von $R$, dann gilt
\begin{align*}
\int\limits_{\ph\circ\gamma} f(z)\dz = 0.\fishhere
\end{align*}
\end{prop}

\begin{center}
\psset{unit=1cm}
\begin{pspicture}(-5,0)(6,5)

%\psgrid

 \psline[fillstyle=solid,%
 fillcolor=glightgray,%
 linestyle=none]%
 (-1,1)(-4,1)(-4,3)(-1,3)(-1,1)
 
 \psline[linecolor=gdarkgray,arrows=<-](-1,2)(-1,1)(-2.5,1)
 \psline[linecolor=gdarkgray,arrows=<-](-2.5,1)(-4,1)(-4,2)
 \psline[linecolor=gdarkgray,arrows=<-](-4,2)(-4,3)(-2.5,3)
 \psline[linecolor=gdarkgray,arrows=<-](-2.5,3)(-1,3)(-1,2)

 % Kartoffel
 \psccurve[fillstyle=solid,%
 fillcolor=glightgray,%
 linestyle=dotted,%
 linecolor=gdarkgray]%
 (1,0.5)(1,3)(3,2.5)(4.5,3)(4.5,1)(3,1)
 
 \psbezier[linecolor=darkblue,arrows=->]%
	(-1.4,3.2)(-0.5,3.5)%
	(0.5,3.5)(1.2,2.1)

\pscustom[linecolor=gdarkgray]{%
 \psbezier[liftpen=2]%
	(1,1)(1,1.5)(2,1.5)(2,2)
 \psbezier[liftpen=2]%
	(2,2)(2.25,2.3)(2.75,1.8)(3,1.9)
 \psbezier[liftpen=2]%
	(3,1.9)(3,1.4)(2,1.4)(2,0.9)
 \psbezier[liftpen=2]%
	(2,0.9)(1.75,0.8)(1.25,1.3)(1,1)
}

 \rput(0,3.8){\color{gdarkgray}$\ph$}
 \rput(1.2,1.8){\color{gdarkgray}$\ph\circ\gamma$}
 \rput(-2.5,2){\color{gdarkgray}$R$}
 \rput(5.2,3){\color{gdarkgray}$O$}
 \rput[l](-0.8,1){\color{gdarkgray}$\gamma$}
\end{pspicture}
\end{center}


\begin{proof}
\begin{enumerate}[label=\arabic{*}.)]
  \item $\ph\circ\gamma$ ist stückweise $C^1$ in $O$, also ist das das Integral
  $\int\limits_{\ph\circ\gamma} f(z)\dz$ definiert.
  \item $R$ ist kompakt und $\nabla\ph=(\partial_1 \ph, \partial_2 \ph)^t$
  stetig auf $R$. Daher nimmt $\abs{\nabla \ph}$ dort sein Maximum an, es
  existiert also ein $C\in\R$, so dass
  \begin{align*}
  \abs{\nabla\ph(x,y)}\le \norm{\nabla\ph}_\infty =: C,\quad\forall (x,y)\in R.
  \end{align*}
  \item\label{proof:2.26.3} Definiere eine Folge $(R_n)$ von Rechtecken mit
  Randkurve $\gamma_n$. Dabei sei $R_0:=R$ und $\gamma_0:=\gamma$.
  
  Teile $R_n$ durch Seitenhalbierung in 4 Rechtecke, wobei $R_{n+1}:=$ dasjenige
  der 4, für das $\abs{\int_{\ph\circ\gamma_{n+1}} f(z) \dz}$ am größten ist.

  Offensichtlich liegt $R_{n+1}$ in $R_n$ und es gilt
  \begin{align*}
  \int\limits_{\ph\circ\gamma_n} f(z)\dz = \sum\limits_{j=1}^4
  \int\limits_{\ph\circ\alpha_j} f(z)\dz,
  \end{align*}
  da sich jeweils die überlagernden Kurvenstücke von $\alpha_j$ wegheben:
  \begin{center}
\psset{unit=1cm}
\begin{pspicture}(0,0)(5,4)

%\psgrid

 \psline[linecolor=gdarkgray]%
 (1,1)(4,1)(4,3)(1,3)(1,1)
 
 \psline[linecolor=gdarkgray]%
 (2.5,1)(2.5,3)
 
 \psline[linecolor=gdarkgray]%
 (1,2)(4,2)
 
 \psline[linecolor=darkblue,arrows=->]%
 (2.5,0.9)(4.1,0.9)(4.1,3.1)(0.9,3.1)(0.9,0.9)(2.5,0.9)
 
 \psline[linecolor=yellow,arrows=->]%
 (2.6,2.4)(2.6,2.1)(2.6,2.1)(3.9,2.1)(3.9,2.9)(2.6,2.9)(2.6,2.4)
 
 \psline[linecolor=yellow,arrows=->]%
 (2.6,1.4)(2.6,1.1)(2.6,1.1)(3.9,1.1)(3.9,1.9)(2.6,1.9)(2.6,1.4)
 
 \psline[linecolor=yellow,arrows=->]%
 (2.4,2.6)(2.4,2.9)(1.1,2.9)(1.1,2.1)(2.4,2.1)(2.4,2.6)
 
 \psline[linecolor=yellow,arrows=->]%
 (2.4,1.6)(2.4,1.9)(1.1,1.9)(1.1,1.1)(2.4,1.1)(2.4,1.6)
 
 \pscircle(2.5,1.5){0.25}
 
 \rput[lb](4.2,3.2){\color{gdarkgray}$R_n$}
 \rput[t](0.8,0.8){\color{darkblue}$\gamma_n$}
 \rput[lb](1.2,1.2){\color{yellow}$\alpha_1$}
 \rput[rb](3.8,1.2){\color{yellow}$\alpha_2$}
 \rput[rb](3.8,2.2){\color{yellow}$\alpha_3$}
 \rput[lb](1.2,2.2){\color{yellow}$\alpha_4$}
 
\end{pspicture}
\end{center}
 Es gilt damit
  \begin{align*}
  \abs{\int\limits_{\ph\circ\gamma_{n}} f(z) \dz} \le 4
  \abs{\int\limits_{\ph\circ\gamma_{n+1}} f(z) \dz}.
  \end{align*}
  Per Induktion erhalten wir
  \begin{align*}
  \abs{\int\limits_{\ph\circ\gamma_0} f(z) \dz} \le 4^n
  \abs{\int\limits_{\ph\circ\gamma_{n}} f(z) \dz}.
  \end{align*}
  \item
  Definiere eine Folge $(x_n)$ mit $x_n:=$ Mittelpunkt von $R_n$, dann folgt
  \begin{align*}
  \abs{x-x_n} \le L(\gamma_n) = \frac{1}{2^n}L(\gamma_0),
  \end{align*}
für $x\in R_n$. Für $m\ge n$ und $x_m\in R_m$ gilt dann
\begin{align*}
\abs{x_m-x_n} < \frac{1}{2^n} L(\gamma_0),
\end{align*}
also ist $(x_n)$ eine Cauchyfolge und damit konvergent gegen ein $y\in R_0$.
Dabei ist insbesondere $\abs{y-x_n} \le \frac{1}{2^n} L(\gamma_0)$.
  \item Da $\ph$ stetig ist, gilt für $x_n\to y$ auch $\ph(x_n)\to \ph(y) :=
  z_0\in O$.\\
  Sei $z\in \im \ph\circ\gamma_n,\;z = \ph(x),\;x\in R_n$, dann gilt nach dem
  Mittelwertsatz für Funktionen $f: \R^n\to\R$,
  \begin{align*}
  \abs{\Re (z-z_0)} &= \abs{\Re(\ph(x)) - \Re(\ph(y))} \overset{\text{MWS}}{=}
  \norm{\nabla \Re \ph(x)}\abs{x-y} \\
  &\le C\cdot\abs{x-y} \le CL(\gamma_n) \le \frac{C}{2^n}
  L(\gamma_0).
  \end{align*}
  Analog folgt für den Imaginärteil $\abs{\Im (z-z_0)} \le \frac{C}{2^n}
  L(\gamma_0)$, und damit gilt
  \begin{align*}
  \abs{z-z_0} \le \sqrt{2} \frac{C}{2^n} L(\gamma_0).
  \end{align*}
  \item
  Sei $\ep >0$ vorgegeben, $f$ ist differenzierbar, also gilt,
  \begin{align*}
  f(z) = f(z_0) + (z-z_0)f'(z_0) + r(z,z_0),
  \end{align*}
mit $\abs{\frac{r(z,z_0)}{z-z_0}} < \ep$ für $0 < \abs{z-z_0} < \delta$.

Für $n$ hinreichend groß gilt
\begin{align*}
&\abs{\ph(x_n) - z_0} < \delta,\\
&\abs{r(\ph(x_n),z_0)} < \ep\abs{\ph(x_n)-z_0} \le \ep\sqrt{2} \frac{C}{2^n}
L(\gamma_0).
\end{align*}
Daraus folgt
\begin{align*}
\abs{\int\limits_{\ph\circ\gamma_n} f(z) \dz} &\le
\underbrace{\abs{\int\limits_{\ph\circ\gamma_n} f(z_0) + (z-z_0)f'(z_0) \dz}}_{=
0, \text{ da Polynom}} +
\abs{\int\limits_{\ph\circ\gamma_n} r(z,z_0) \dz} \\ &\le \max \abs{r(z,z_0)}
L(\ph\circ \gamma_n)\\
&\le \ep\sqrt{2}\; \frac{C}{2^n} L(\gamma_0) \int\limits_a^b \abs{\nabla
\ph\circ\gamma_n(t)}\abs{\dot{\gamma}_n(t)}\dt \\
&\le \ep\sqrt{2}\; \frac{C}{2^n} L(\gamma_0)\; C \frac{L(\gamma_0)}{2^n}
= \ep \sqrt{2} \frac{C^2}{4^n} L^2(\gamma_0) = \frac{\ep D}{4^n}.
\end{align*}
\item Zusammen mit \ref{proof:2.26.3} gilt somit für jedes $\ep > 0$,
\begin{align*}
\abs{\int\limits_{\ph\circ\gamma} f(z)\dz} \le 4^n \frac{\ep}{4^n}D = \ep
D.\qedhere
\end{align*}
\end{enumerate}
\end{proof}

\begin{cor}
\label{prop:2.28}
Sei $f: O\to\C$ differenzierbar, $\gamma_1,\gamma_2$ $C^1$-Kurven in $O$ und
$\gamma_1\sim\gamma_2$, dann gilt
\begin{align*}
\int_{\gamma_1} f(z)\dz = \int_{\gamma_2} f(z)\dz.\fishhere
\end{align*}
\end{cor}
\begin{proof}
Sei $\phi$ eine Homotopie zwishen $\gamma_1$ und $\gamma_2$. Setze $\ph :=
\phi$ und $R=[0,1]\times[0,1]$.
Wir müssen zwischen den zwei Definitionen einer Homotopie unterscheiden
\begin{enumerate}[label=Fall \arabic{*})]
  \item $\phi(0,s) = \phi(0,0),\;\phi(1,s) = \phi(1,0)$. Sei $\gamma$ die
  Aneinanderknüpfung von $\alpha_1,\alpha_2, \alpha_3$ und $\alpha_4$ (Skizze), 
  setze,
  \begin{align*}
  &\ph\circ\alpha_1(t) := \ph(t,0) = \gamma_1(t),\\
  &\ph\circ\alpha_2(t) := \ph(1,0) ,\\
  &\ph\circ\alpha_3(t) := \ph(1-t,1) = -\gamma_2(t),\\
  &\ph\circ\alpha_4(t) := \ph(0,1) = \ph(0,0),
  \end{align*}
  dann folgt mit \ref{prop:2.27},
  \begin{align*}
  0 &= \int_{\ph\circ\gamma} f(z)\dz \\ &= \int_{\gamma_1} f(z)\dz +
  \underbrace{\int_{\ph\circ\alpha_2} f(z)\dz}_{=0} + \int_{-\gamma_2} f(z)\dz +
  \underbrace{\int_{\ph\circ\alpha_4} f(z)\dz}_{=0}
  \\ &= \int_{\gamma_1} f(z)\dz - \int_{\gamma_2} f(z)\dz.
  \end{align*}
\begin{center}
\psset{unit=1cm}
\begin{pspicture}(-1,-1)(7,4)

%\psgrid

 \psaxes[labels=none,ticks=none,linecolor=gdarkgray,tickcolor=gdarkgray]{->}%
 (0,0)(-0.5,-0.5)(2,2)[\color{gdarkgray}$t$,-90][\color{gdarkgray}$s$,0]
 
 \psline[linecolor=gdarkgray,%
 fillstyle=solid,%
 fillcolor=glightgray]
 (0,0)(1,0)(1,1)(0,1)(0,0)
 
 \psline[linecolor=yellow,arrows=->]%
 (-0.1,-0.1)(0.5,-0.1)
 \psline[linecolor=yellow]%
 (0.5,-0.1)(1.1,-0.1)
 
  \psline[linecolor=yellow,arrows=->]%
 (1.1,1.1)(0.5,1.1)
 \psline[linecolor=yellow]%
 (0.5,1.1)(-0.1,1.1)
 
  \psline[linecolor=yellow,arrows=->]%
 (-0.1,1.1)(-0.1,0.5)
 \psline[linecolor=yellow]%
(-0.1,0.5)(-0.1,-0.1)

  \psline[linecolor=yellow,arrows=->]%
 (1.1,-0.1)(1.1,0.5)
 \psline[linecolor=yellow]%
(1.1,0.5)(1.1,1.1)

\psyTick[linecolor=gdarkgray](1){\color{gdarkgray}$1$}
\psxTick[linecolor=gdarkgray](1){\color{gdarkgray}$1$}

 % Kartoffel
 \psccurve[fillstyle=solid,%
 fillcolor=glightgray,%
 linestyle=dotted,%
 linecolor=gdarkgray]%
 (3,0)(3,2.5)(5,2)(6.5,2.5)(6.5,0.5)(5,0.5)
 
 \psbezier[linecolor=gdarkgray](2.8,1.4)(3.2,2)(4.2,1.4)(4.8,1.8)
 \psbezier[linecolor=gdarkgray](2.8,1.4)(3.2,0.4)(4.6,1)(4.8,1.8)
 \psdots[dotstyle=*,linecolor=darkblue](2.8,1.4)(4.8,1.8)
 
 \psbezier[arrows=->,linecolor=darkblue](0.6,1.6)(1,3)(2,3)(2.8,2.6)
 
 \rput[r](2.4,1.4){\color{darkblue}$\ph\circ\alpha_4$}
 \rput[t](3.7,0.8){\color{gdarkgray}$\ph\circ\alpha_3:=-\gamma_2$}
 \rput[b](3.7,1.8){\color{gdarkgray}$\ph\circ\alpha_1:=\gamma_1$}
 \rput[l](5,1.8){\color{darkblue}$\ph\circ\alpha_2$}
 \rput[t](6.4,3){\color{gdarkgray}$O$}
 \rput(1.4,3){\color{gdarkgray}$\ph=\phi$}

  \rput[t](0.5,-0.2){\color{yellow}$\alpha_1$}
  \rput[b](0.5,1.2){\color{yellow}$\alpha_3$}
  \rput[l](1.2,0.5){\color{yellow}$\alpha_2$}
  \rput[r](-0.2,0.5){\color{yellow}$\alpha_4$}

\end{pspicture}
\end{center}

  \item $\phi(0,s) = \phi(1,s)$, ebenso mit \ref{prop:2.27} folgt
  \begin{align*}
  0 &= \int_{\ph\circ\gamma} f(z)\dz \\ &= \int_{\gamma_1} f(z)\dz +
  \int_{\ph\circ\alpha_2} f(z)\dz + \int_{-\gamma_2} f(z)\dz +
  \int_{\ph\circ\alpha_4} f(z)\dz \\
  &= \int_{\gamma_1} f(z)\dz - \int_{\gamma_2} f(z)\dz.
  \end{align*}


\begin{center}
\psset{unit=1cm}
\begin{pspicture}(0,-0.5)(5,3)

%\psgrid

 % Kartoffel
 \psccurve[fillstyle=solid,%
 fillcolor=glightgray,%
 linestyle=dotted,%
 linecolor=gdarkgray]%
 (1,0)(1,2.5)(3,2)(4.5,2.5)(4.5,0.5)(3,0.5)
 
 \rput[linecolor=gdarkgray]{-45}(1.4,1.6){\psellipse(0,0)(0.4,0.6)}
 \rput[linecolor=gdarkgray]{-30}(3,1.2){\psellipse(0,0)(0.4,0.6)}
 
 \psbezier[linecolor=darkblue,arrows=*-*](0.97,1.2)(1.8,1.8)(1.8,0.6)(2.55,1)
 
 \rput[lt](1.2,2.4){\color{gdarkgray}$\gamma_1$}
 \rput[lt](3.6,1.8){\color{gdarkgray}$-\gamma_2$}
 \rput[rt](2.4,0.8){\color{darkblue}\small{$\ph\circ\alpha_4 =
 -\ph\circ\alpha_2$}}
 
 %  
%  \psbezier[linecolor=gdarkgray](2.8,1.4)(3.2,2)(4.2,1.4)(4.8,1.8)
%  \psbezier[linecolor=gdarkgray](2.8,1.4)(3.2,0.4)(4.6,1)(4.8,1.8)
%  \psdots[dotstyle=*,linecolor=darkblue](2.8,1.4)(4.8,1.8)
%  
%  \psbezier[arrows=->,linecolor=darkblue](0.6,1.6)(1,3)(2,3)(2.8,2.6)
%  
%  \rput[r](2.4,1.4){\color{darkblue}$\ph\circ\alpha_4$}
%  \rput[t](3.6,0.8){\color{gdarkgray}$\ph\circ\alpha_3$}
%  \rput[b](3.8,1.8){\color{gdarkgray}$\ph\circ\alpha_1$}
%  \rput[l](5,1.8){\color{darkblue}$\ph\circ\alpha_2$}
%  \rput[t](6.4,3){\color{gdarkgray}$O$}
%  \rput(1.4,3){\color{gdarkgray}$\ph=\phi$}
% 
%   \rput[t](0.5,-0.2){\color{yellow}$\alpha_1$}
%   \rput[b](0.5,1.2){\color{yellow}$\alpha_3$}
%   \rput[l](1.2,0.5){\color{yellow}$\alpha_2$}
%   \rput[r](-0.2,0.5){\color{yellow}$\alpha_4$}

\end{pspicture}
\end{center}
\end{enumerate} 
\hfill\qedhere
\end{proof}

\begin{proof}[Teilbeweis Satz \ref{prop:2.25}]
 \ref{prop:2.25.1} $\Rightarrow$ \ref{prop:2.25.2}: Folgt aus \ref{prop:2.28}
 da $f$ stetig.\\
 \ref{prop:2.25.2} $\Rightarrow$ \ref{prop:2.25.3}: Ist $\gamma$
 $C^1$-nullhomotop, dann ist $\gamma\sim\tilde{\gamma}$ mit einer konstanten
 Kurve $\tilde{\gamma}$. Daraus folgt mit \ref{prop:2.25.2}, dass
 \begin{align*}
 \int\limits_{\gamma} f(z)\dz =\int\limits_{\tilde{\gamma}} f(z)\dz = 0.\qedhere 
 \end{align*}
\end{proof}

\begin{prop}[Cauchyscher Integralsatz für Kreisscheiben]
\label{prop:2.29}
Sei $f: O\to\C$ differenzierbar, sowie $z_0\in O$, $r> 0$ mit
$\overline{K_r(z_0)} := \setdef{z\in\C}{\abs{z-z_0} \le r} \subseteq O$.
Dann gilt
\begin{align*}
f(z) = \frac{1}{2\pi i} \int\limits_{\abs{\zeta-z_0}=r}
\frac{f(\zeta)}{\zeta-z}\dzeta, \quad \text{ für } \abs{z-z_0} <r.
\end{align*}
Ist $f$ auf dem Kreisrand bekannt, so ist es damit im Kreisinneren
eindeutig bestimmt.

Das Integral ist hierbei zu verstehen als Integral längs
\begin{align*}
\gamma(t) = z_0 + r e^{it},\; 0\le t \le 2\pi.\fishhere
\end{align*}
\end{prop}

%\end{center}

\begin{proof}
Sei $\gamma_\ep(t) = z+\ep\ e^{it}$, $0\le t\le 2\pi$, dann ist offensichtlich
$\gamma\sim\gamma_\ep$ in $O\setminus \{z\}$, und $\zeta\mapsto
\dfrac{f(\zeta)}{\zeta-z}$ differenzierbar in $O\setminus \{z\}$.
\begin{center}
\psset{unit=1cm}
\begin{pspicture}(0,-0.5)(5,3)

 % Kartoffel
 \psccurve[fillstyle=solid,%
 fillcolor=glightgray,%
 linestyle=dotted,%
 linecolor=gdarkgray]%
 (1,0)(1,2.5)(3,2)(4.5,2.5)(4.5,0.5)(3,0.5)
 
 \pscircle[linecolor=gdarkgray,linestyle=dashed](1.6,1.2){1}
 
 \pscircle[linecolor=gdarkgray](1.4,1.7){0.4}
 
 \rput[lt](1.65,1.15){\color{gdarkgray}$z_0$}
 \rput[lt](1.45,1.65){\color{gdarkgray}$z$}
 \psdot[linecolor=gdarkgray](1.6,1.2)
 \psdot[linecolor=gdarkgray](1.4,1.7)
\end{pspicture}
\end{center}

Aus \ref{prop:2.25.2} folgt
\begin{align*}
\int\limits_{\gamma} \frac{f(\zeta)}{\zeta-z}\dzeta &= \int\limits_{\gamma_\ep}
\frac{f(\zeta)}{\zeta-z}\dzeta = \int\limits_{\gamma_\ep}
\frac{f(\zeta)-f(z)+f(z)}{\zeta-z}\dzeta \\ &=  \int\limits_{\abs{\zeta-z}=\ep}
\frac{f(\zeta)-f(z)}{\zeta-z} \dzeta + f(z)\int\limits_{\abs{\zeta-z}=\ep}
\frac{1}{\zeta-z}\dzeta.
\end{align*}
Da $f$ in $z$ differenzierbar ist, ist $\abs{\frac{f(\zeta)-f(z)}{\zeta-z}} <
C$ für $\abs{\zeta-z}<\ep$ und damit ist
\begin{align*}
\abs{\int\limits_{\gamma_\ep} \frac{f(\zeta)-f(z)}{\zeta-z}}\dzeta \le
CL(\gamma_\ep) = C2\pi\ep.
\end{align*}
Also gilt für $\ep \to 0$,
\begin{align*}
\int\limits_\gamma\frac{f(\zeta)}{\zeta-z}\dzeta = 0 +
f(z)\int\limits_{\abs{\zeta}=\ep} \frac{1}{\zeta}\dzeta = 2\pi i\; f(z).\qedhere
\end{align*}
\end{proof}

\begin{prop}[Potenzreihenentwicklungssatz]
\label{prop:2.30}
Sei $f:O\to\C$ differenzierbar, $z_0\in O$, $r > 0$ mit
$\overline{K_r(z_0)}\subseteq O$, dann gibt es $a_n\in\C$, so dass
\begin{align*}
f(z) = \sum\limits_{n=0}^\infty a_n(z-z_0)^n, \text{ für } \abs{z-z_0} <
r.\fishhere
\end{align*}
\end{prop}
\begin{proof}
Aus Satz \ref{prop:2.29} folgt: Für $\abs{z-z_0} < r$ gilt
\begin{align*}
f(z) &= \frac{1}{2\pi i}\int\limits_{\abs{\zeta-z_0}=r}
\frac{f(\zeta)}{\zeta-z}\dzeta = \frac{1}{2\pi i}\int\limits_{\abs{\zeta-z_0}=r}
\frac{f(\zeta)}{\zeta-z_0} \frac{1}{1-\frac{z-z_0}{\zeta-z_0}} \dzeta
\\ &= \frac{1}{2\pi i}\int\limits_{\abs{\zeta-z_0}=r} \frac{f(\zeta)}{\zeta-z_0}
\sum\limits_{k=0}^\infty \left(\frac{z-z_0}{\zeta-z_0}\right)^k \dzeta,
\end{align*}
da $\abs{\frac{z-z_0}{\zeta-z_0}}<1$ konvergiert die Reihe gleichmäßig bezüglich $\zeta$
auf dem Kreis.
\begin{align*}
\ldots &= \frac{1}{2\pi i}\sum\limits_{k=0}^\infty \int\limits_{\abs{\zeta-z_0}=r}
\frac{f(\zeta)}{\zeta-z_0}\left(\frac{z-z_0}{\zeta-z_0}\right)^k \dzeta
\\ &= \sum\limits_{k=0}^\infty \frac{1}{2\pi i} \int\limits_{\abs{\zeta-z_0}=r}
\frac{f(\zeta)}{(\zeta-z_0)^{k+1}}\dzeta (z-z_0)^k
= \sum\limits_{k=0}^\infty a_k (z-z_0)^k,
\end{align*}
für $a_k = \frac{1}{2\pi i} \int_{\abs{\zeta-z_0}=r}
\frac{f(\zeta)}{(\zeta-z_0)^{k+1}}\dzeta$.\qedhere
\end{proof}

\begin{cor}
\label{prop:2.31}
Sei $f: O\to \C$ differenzierbar, $z_0\in O$, sowie
\begin{align*}
f(z) = \sum\limits_{n=0}^\infty a_n(z-z_0)^n,
\end{align*} 
die Potenzreihenentwicklung von $f$.
\begin{enumerate}
  \item Für die Glieder der Potenzreihe gilt
  \begin{align*}
  a_n = \frac{1}{2\pi i} \int_{\abs{\zeta-z_0}=r}
\frac{f(\zeta)}{(\zeta-z_0)^{n+1}}\dz = \frac{f^{(n)}(z_0)}{n!}.
  \end{align*}
  \item Für den Konvergenzradius $R$ der Potenzreihe gilt
  \begin{align*}
  R\ge \sup\setdef{r > 0}{\overline{K_r(w_0)}\subseteq O}.\fishhere
  \end{align*}
\end{enumerate}
\begin{proof}
\begin{enumerate}
  \item Da man im Konvergenzradius gliedweise differenzieren kann, folgt
  dies direkt indem man $z = z_0$ setzt.
  \item Sei $z\in\C$ mit $\abs{z-z_0} < \sup\setdef{r >
  0}{\overline{K_r(z_0)}\subseteq O}$, dann existiert ein $r>0$, so dass
  $\overline{K_r(z_0)}\subseteq O$ und damit konvergiert die Potenzreihe im
  Punkt $z$.\qedhere
\end{enumerate}
\end{proof}
\end{cor}

\begin{bsp}
\label{bsp:2.32}
Sei $O=\C\setminus\{i,-i\}$ und
\begin{align*}
f(z) = \frac{1}{z^2+1},\quad z\in O.
\end{align*}
Entwickelt man $f$ um $z_0\in O$ in eine Potenzreihe, folgt aus \ref{prop:2.31},
\begin{align*}
R \ge \min\{\abs{z_0 - i}, \abs{z_0+i}\}.
\end{align*}
Wegen $\abs{f(z)}\to\infty$ für $z\to \pm i$ folgt sogar $R=\min\{\abs{z_0 -
i}, \abs{z_0+i}\}$.
\begin{center}
\psset{unit=1cm}
\begin{pspicture}(-1,-2)(5,3)
 %\psgrid

 \psaxes[labels=none,ticks=none,linecolor=gdarkgray,tickcolor=gdarkgray]{->}%
 (0,0)(-0.5,-1.5)(4.5,2.5)[\color{gdarkgray}$Re$,-90][\color{gdarkgray}$Im$,0]
 
 \pscircle(0,1){0.1}
 \pscircle(0,-1){0.1}

 \psdot[linecolor=gdarkgray](3,2)
 
 \psline[linecolor=gdarkgray](0,1)(3,2)
 \psline[linecolor=gdarkgray](0,-1)(3,2)
 
 \pscircle[linecolor=gdarkgray,linestyle=dotted](3,2){3.16}
 
 \rput[lt](3.05,1.95){\color{gdarkgray}$z_0$}
\end{pspicture}
\end{center}

Setzen wir beispielsweise $z_0=0$, ist die Potenzreihe von $f$ gegeben durch,
\begin{align*}
f(z) = \frac{1}{1-(-z^2)} = \sum\limits_{n=0}^\infty (-z^2)^n, \text{ für }
\abs{z}<R =1,
\end{align*}
was $R=\min\{\abs{-i}, \abs{i}\}$ bestätigt.

Setzen wir nun $z_0=2+i$, erhalten wir
\begin{align*}
R=\min\{\abs{2}, \abs{2+2i}\} = 2,
\end{align*}
was sich durch einfaches Nachrechnen bestätigen lässt.\bsphere
\end{bsp}

\begin{defn}
\label{defn:2.33}
Eine Funktion $f:\C\to\C$, die auf ganz $\C$ differenzierbar ist, heißt
\emph{ganze Funktion}. Ist $f$ ganz, dann gilt für beliebige $z_0\in\C$,
\begin{align*}
f(z) = \sum\limits_{n=0}^\infty a_n(z-z_0)^n, \text{ für alle } z\in\C.\fishhere
\end{align*}
\end{defn}

\begin{proof}[Teilbeweis Satz \ref{prop:2.25}]
\ref{prop:2.25.1} $\Rightarrow$ \ref{prop:2.25.4}: Siehe Satz \ref{prop:2.30}.\\
\ref{prop:2.25.4} $\Rightarrow$ \ref{prop:2.25.5}: Siehe Satz \ref{prop:2.34}.\\
\ref{prop:2.25.5} $\Rightarrow$ \ref{prop:2.25.1}: trivial.\qedhere
\end{proof}

\begin{prop}
\label{prop:2.34} Sei $f(z) = \sum\limits_{n=0}^\infty a_n(z-z_0)^n$ für
$\abs{z-z_0} < R$ mit $R> 0$, dann ist $f$ in $K_R(z_0) :=
\setdef{z\in\C}{\abs{z-z_0}<R}$ beliebig oft differenzierbar und 
\begin{align*}
f^{(k)}(z) = \sum\limits_{n=0}^\infty n(n-1)\ldots(n-k+1) a_n(z-z_0)^{n-k},
\text{ für } \abs{z-z_0}<R.\fishhere
\end{align*}
\end{prop}
\begin{proof}
Für $k=1$ (Rest folgt mit Induktion)

Die gliedweise Ableitung
\begin{align*}
g(z) = \sum\limits_{n=1}^\infty n a_n (z-z_0)^{n-1},
\end{align*}
hat denselben Konvergenzradius wie $f(z)$.

$s_k(z) = \sum\limits_{n=0}^k a_n (z-z_0)^{n}$ ist ein Polynom und damit 
insbesondere differenzierbar, also gilt
\begin{align*}
\int\limits_\gamma s_k'(z)\dz = s_k(z_2) - s_k(z_1), \text{ für $\gamma$ Kurve
in $O$ von $z_1$ nach $z_2$}.
\end{align*}
Daraus folgt $\int\limits_\gamma s_k'(z)\dz$ ist
unabhängig von der stückweisen $C^1$ Kurve $\gamma$. 
Für $z\in K_R(z_0)$ sei $\gamma_{z_0,z}$ beliebige stückweise $C^1$-Kurve von
$z_0$ nach $z$, die ganz in $\overline{K_r(z_0)}$ verläuft für ein $r\in(0,R)$,
damit folgt
\begin{align*}
\underbrace{s_k(z)}_{\to f(z)} = \underbrace{s_k(z_0)}_{\to f(z_0)} +
\int\limits_{\gamma_{z_0,z}} \underbrace{s_k'(\zeta)}_{\unito g(\zeta)\dzeta}
\dzeta \Leftrightarrow\; f(z) = f(z_0) + \int\limits_{\gamma_{z_0,z}} g(\zeta)\dzeta,
\end{align*}
unabhängig von $\gamma$.

Für $z,w\in K_R(z_0)$ wähle $\gamma_{z_0,z}, \gamma_{z_0,w}$ wie in Skizze,
\begin{center}
\psset{unit=1cm}
\begin{pspicture}(-2.2,-2.2)(2.2,2.2)
 %\psgrid

 \pscircle[linecolor=gdarkgray,%
 linestyle=dotted,%
 fillcolor=glightgray,%
 fillstyle=solid]%
 (0,0){2}
 
 \pscircle[linecolor=gdarkgray](0,0){1.5}

 \psdot[linecolor=gdarkgray](0,0)
 \psdot[linecolor=gdarkgray](-0.8,0.8)
 \psdot[linecolor=gdarkgray](0.6,1)
 
 \psline[linecolor=darkblue,arrows=->](0,0)(-0.8,0.8)(0.6,1)
 \psline[linecolor=gdarkgray,arrows=->](0,0)(-0.8,0.8)
 
 \psline[linecolor=gdarkgray,arrows=->](0,0)(-1.41,-1.41)
 \psline[linecolor=gdarkgray,arrows=->](0,0)(1.06,-1.06)
 
 \rput[lb](0.2,0){\color{gdarkgray}$z_0$}
 \rput[rt](-0.5,0.5){\color{gdarkgray}$\gamma_{z_0,z}$}
 \rput[lt](-0.4,1.3){\color{darkblue}$\gamma_{z_0,w}$}
 \rput[rb](-0.8,0.9){\color{gdarkgray}$z$}
 \rput[lt](0.7,0.9){\color{gdarkgray}$w$}
 
 \rput(-1.4,-1){\color{gdarkgray}$R$}
 \rput[rb](1.1,-0.8){\color{gdarkgray}$r$}
\end{pspicture}
\end{center}
dann gilt $f(w) - f(z) = \int\limits_{\gamma_{z_0,w}} g(\zeta)\dzeta -
\int\limits_{\gamma_{z_0,z}} g(\zeta)\dzeta = \int\limits_{\gamma_{z,w}}
g(\zeta)\dzeta$. Damit folgt
\begin{align*}
\abs{\frac{f(w)-f(z)}{w-z}-g(z)} &= 
\abs{\frac{1}{w-z}\int\limits_{\gamma_{z,w}} g(\zeta) - g(z) \dzeta}
\\ & \le \frac{1}{\abs{w-z}}\max_{\zeta\in\gamma_{z,w}} \abs{g(\zeta) - g(z)}
L(\gamma_{z,w}) < \ep,
\end{align*}
für $\abs{w-z} < \delta$. Also ist $f$ differenzierbar und $f'(z) = g(z)$.\qedhere
\end{proof}

\begin{cor}
\label{prop:2.35}
\begin{enumerate}
  \item $a_n = \frac{f^{(n)}(z_0)}{n!}$, insbesondere ist die Potenzreihe
  eindeutig.
  \item \emph{Cauchysche Koeffizientenformel},
  \begin{align*}
  f^{(n)}(z_0) = \frac{n!}{2\pi i}
  \int\limits_{\abs{\zeta-z_0}=r} \frac{f(\zeta)}{(\zeta-z_0)^{n+1}}\dzeta,
  \end{align*}
  für jedes $r\in(0,R)$.\fishhere
\end{enumerate}
\end{cor}

\begin{bsp}
\label{bsp:2.36}
\begin{enumerate}
  \item $(e^z)' = e^z$.
  \item $(\cos z)' = -\sin z$.
  \item $(\sin z)' = \cos z$.
\end{enumerate}
Insbesondere sind $e^z, \sin z, \cos z$ ganz.\bsphere
\end{bsp}

\begin{prop}[Cauchyabschätzung für Taylorkoeffizienten]
\label{prop:2.37}
Sei $f: O\to\C$ differenzierbar, $\overline{K_r(z_0)}\subseteq O$, $\abs{f(z)}
\le M$ für $\abs{z-z_0}=r$ und
\begin{align*}
f(z) = \sum\limits_{n=0}^\infty a_n(z-z_0)^n,
\end{align*}
für $\abs{z-z_0}<r$, dann gilt
\begin{align*}
\abs{\frac{f^{(n)}(z_0)}{n!}} = \abs{a_n} \le \frac{M}{r^n}.\fishhere
\end{align*}
\end{prop}
\begin{proof}
Hier können wir \ref{prop:2.31} verwenden,
\begin{align*}
\abs{a_n} &= \abs{\frac{1}{2\pi i} \int\limits_{\abs{z-z_0}=r}
\frac{f(z)}{(z-z_0)^{n+1}}\dz} \\ &\le \frac{1}{2\pi} \max_{\abs{z-z_0}= r}
\frac{\abs{f(z)}}{\abs{z-z_0}^{n+1}}L(\gamma) \le \frac{M}{r^n}.\qedhere
\end{align*}
\end{proof}

\begin{prop}[Satz von Liouvielle]
\label{prop:2.38}
Jede beschränkte ganze Funktion ist konstant.\fishhere
\end{prop}
\begin{proof}
Sei $\abs{f(z)} \le M$ für $z\in\C$,
\begin{align*}
f(z) = \sum\limits_{n=0}^\infty a_n\;z^n, \text{ für } z\in\C,
\end{align*} 
dann folgt aus \ref{prop:2.37}, dass $\abs{a_n} \le \frac{M}{r^n}$ für alle
$r\ge 0$ mit $\overline{K_r(0)} \subseteq O = \C$. Da $r > 0$ beliebig war
folgt,
\begin{align*}
\abs{a_n} \le \lim\limits_{r\to\infty} \frac{M}{r^n} =
\begin{cases}
M, & \text{für } n=0,\\
0, & \text{für } n\ge1.
\end{cases}
\end{align*}
Also ist $f(z) = a_0$ für $z\in\C$.\qedhere
\end{proof}

\begin{prop}[Riemannscher Hebbarkeitssatz]
\label{prop:2.39}
Sei $z_0\in O$, $f: O\setminus\{z_0\}\to \C$ holomorph.
Ferner
\begin{align*}
\exists M > 0 \exists r > 0 : \abs{f(z)} \le M, \text{ für } 0<\abs{z-z_0}<r.
\end{align*}
Dann ist $f$ in $z_0$ holomorph ergänzbar, d.h. es existiert ein $a\in\C$ so,
dass
\begin{align*}
\tilde{f}(z) = \begin{cases}
               a, & z = z_0,\\
               f(z), & \text{sonst},
               \end{cases}
\end{align*}
holomorph ist auf $O$.\fishhere
\end{prop}
\begin{proof}
Setzen wir
\begin{align*}
g(z) := 
\begin{cases}
0, & \text{für } z = z_0,\\
(z-z_0)^2f(z), &\text{sonst},
\end{cases}
\end{align*}
dann ist $g$ differenzierbar für $z\neq z_0$ und für $z=z_0$. Damit folgt,
\begin{align*}
\frac{g(z)-g(z_0)}{z-z_0} = (z-z_0)f(z) \to 0 \text{ für } z\to z_0.
\end{align*}
Insbesondere ist $g'(z_0) = 0,\; g(z_0) = 0$ und $g$ ist holomorph auf ganz $O$.
Es existiert also ein $r > 0$, so dass die Potenzreihe
\begin{align*}
g(z) = \sum\limits_{n=0}^\infty a_n(z-z_0)^n = \sum\limits_{n=2}^\infty
a_n(z-z_0)^n,
\end{align*}
für $\abs{z-z_0} < r$ konvergiert. $f$ hat somit die Darstellung,
\begin{align*}
f(z) = \frac{g(z)}{(z-z_0)^2} = \sum\limits_{n=2}^\infty a_n(z-z_0)^{n-2},
\end{align*}
für $0 < \abs{z-z_0} <r$. Setzten wir nun,
\begin{align*}
a := \lim\limits_{z\to z_0} \frac{g(z)}{(z-z_0)^2} = a_2,
\end{align*}
können wir $f$ in $z_0$ holomorph fortsetzen und erhalten,
\begin{align*}
\tilde{f}(z) = \sum\limits_{n=2}^\infty a_n(z-z_0)^{n-2}
= \sum\limits_{n=0}^\infty a_{n+2}(z-z_0)^{n}
\end{align*}
für $\abs{z-z_0}<r$.\qedhere
\end{proof}
Der Hebbarkeitssatz ist eine weitere bemerkenswerte Eigenschaft von
holomorphen Funktionen, in $\R$ ist eine solche Aussage nicht möglich.
Die Forderung, dass $z_0\in O$ liegt ist dabei nicht
unerheblich, wie man sich leicht am Beispiel $z\mapsto e^{\frac{1}{z}}$
überlegen kann.

\begin{bsp}
\label{bsp:2.40}
Sei $f$ holomorph in $O$, dann ist auch
\begin{align*}
g(z) :=
\begin{cases}
f'(z_0), & \text{für } z=z_0,\\
\frac{f(z)-f(z_0)}{z-z_0}, & \text{sonst},
\end{cases}
\end{align*}
holomorph in $O$.\bsphere
\end{bsp}

\begin{prop}[Hilfssatz]
\label{prop:2.41}
Sei $f:O\to\C$ wie in \ref{prop:2.25.3} gefordert. Ist $D\subseteq O$
abgeschlossene Dreiecksfläche mit geschlossener stückweiser $C^1$ Randkurve
$\gamma$, so gilt
\begin{align*}
\int\limits_\gamma f(z)\dz = 0.\fishhere
\end{align*}
\end{prop}

\begin{proof}
\begin{enumerate}
  \item Sei $p(t) = 3t^2-2t^3$
  \begin{center}
\psset{unit=1cm}
\begin{pspicture}(-1,-1)(3.5,2.5)
 %\psgrid

 \psaxes[labels=none,ticks=none,linecolor=gdarkgray,tickcolor=gdarkgray]{->}%
 (0,0)(-0.5,-0.5)(3,2)[\color{gdarkgray}$t$,-90][\color{gdarkgray}$y$,0]

\psplot[linewidth=1.2pt,%
	     linecolor=darkblue,%
	     algebraic=true]%
	     {-0.5}{1.6}{3*x^2-2*x^3}
	     
\rput(1.4,1.4){\color{gdarkgray}$y=p(t)$}
\end{pspicture}
\end{center}
\begin{align*}
&p(0) = 0, && p(1) = 1,\\
&p'(t) > 0, && \text{ für 0<t<1},\\
\end{align*}
insbesondere ist $p: [0,1]\to [0,1]$ bijektiv mit $p'(0) = p'(1) = 0$.
\item
Nach Voraussetzung ist $\gamma$ stückweise $C^1$. Wir können also $\gamma$
nach eventueller Umparametrisierung wie folgt durch $C^1$ Kurven
$\alpha_i$ beschreiben
\begin{align*}
\gamma(t) = \begin{cases} \alpha_1(t), & 0 < t \le \frac{1}{3},\\ \alpha_2(t), & \frac{1}{3} \le t < \frac{2}{3},\\
                 \alpha_3(t), & \frac{2}{3} < t \le 1.
                 \end{cases}
\end{align*} 
  \begin{center}
\psset{unit=1cm}
\psset{linecolor=gdarkgray}
\psset{fillcolor=glightgray}
\begin{pspicture}(0.5,0.5)(3.5,3.2)
 %\psgrid

 \psline[fillstyle=solid,%
 	     linestyle=none]%
 	     (1,1)(3,1)(2,3)(1,1)
 
 \psline[arrows=->](2,1)(3,1)(2.5,2)
 \psline[arrows=->](2.5,2)(2,3)(1.5,2)
 \psline[arrows=->](1.5,2)(1,1)(2,1)

 \rput[t](2,0.9){\color{gdarkgray}$\alpha_1$}
 \rput[t](2.8,2.4){\color{gdarkgray}$\alpha_2$}
 \rput[t](1.2,2.4){\color{gdarkgray}$\alpha_3$}
\end{pspicture}
\end{center}
Da $\gamma$ nur an endlich vielen Stellen nicht $C^1$ ist, kann $\gamma$
durch eine geschickte Parametertransformation auf ganz $D$ zu $C^1$
transformiert werden. Sei dazu $\beta_1(t) = \alpha_1\left(\frac{1}{3}p(3t)
\right)$, dann folgt
\begin{align*}
\int\limits_{\beta_1}f(z)\dz = \int\limits_0^\frac{1}{3}
f\left(\alpha_1\left(\frac{1}{3}p(3t)\right)
\right)\alpha_1'\left(\frac{1}{3}p(3t)\right)p'(3t)\dt,
\end{align*}
Substituiere $s = \frac{1}{3}p(3t) \Rightarrow \ds = p'(3t)\dt$,
\begin{align*}
\ldots = \int\limits_0^\frac{1}{3} f\left(\alpha_1(s) \right)\alpha_1'(s)\ds
= \int\limits_{\alpha_1} f(z)\dz.
\end{align*}
Das Gleiche gilt für 
\begin{align*}
\beta_2(t) = \alpha_2\left(\frac{1}{3} + \frac{1}{3}p(3t-\frac{1}{3}) \right),\\
\beta_3(t) = \alpha_3\left(\frac{2}{3} + \frac{1}{3}p(3t-\frac{2}{3}) \right),
\end{align*}
damit erhalten wir
\begin{align*}
\tilde{\gamma}(t) = \begin{cases}
                    \beta_1(t), & 0\le t\le \frac{1}{3},\\
                    \beta_2(t), & \frac{1}{3}< t\le \frac{2}{3},\\
                    \beta_3(t), & \frac{2}{3}< t\le 1,
                    \end{cases}
\end{align*}
und damit ist $\tilde{\gamma}(t)\in C^1([0,1]\to O)$.

$\gamma$ und $\tilde{\gamma}$ sind homotop, also folgt mit \ref{prop:2.25.3}

\begin{align*}
\int\limits_\gamma f(z)\dz =\int\limits_{\tilde{\gamma}} f(z)\dz = 0.
\end{align*}

Um zu prüfen, ob $\tilde{\gamma}$ auch Nullhomotop ist, muss bei der Angabe der
Homotopie darauf geachtet werden, dass die ``Zwischenkurven'' den
Definitionsbereich $D$ nicht verlassen. Wählen wir unsere Standardhomotopie,
\begin{align*}
\phi(t,s) = (1-s)\tilde{\gamma}(t) + s\gamma(t),
\end{align*}
dann ist jedes $\phi(t,s)\in D$, da $D$ konvex ist, also ist
$\tilde{\gamma}\sim\tilde{\gamma}(0)$.\qedhere
\end{enumerate}
\end{proof}

\begin{prop}[Satz von Morera]
\label{prop:2.42}
Sei $f: O\to\C$ stetig und für jede abgeschlossene Dreiecksfläche $D\subseteq O$
mit geschlossener stückweiser $C^1$-Randkurve $\gamma$ sei
\begin{align*}
\int\limits_\gamma f(z)\dz=0,
\end{align*}
dann ist $f$ differenzierbar in $O$.\fishhere
\end{prop}
\begin{proof}
Sei $z_0\in O$ und $\overline{K_r(z_0)}\subseteq O$. Wir müssen zeigen, dass $f$
eine Stammfunktion $F$ auf $\overline{K_r(z_0)}\subseteq O$ besitzt.
Dann ist $F$ differenzierbar und mit \ref{prop:2.25.5} beliebig oft
differenzierbar. Also ist auch $f$ differenzierbar und die Behauptung folgt.

\begin{center}
\psset{unit=1cm}
\psset{linecolor=gdarkgray}
\psset{fillcolor=glightgray}
\begin{pspicture}(-2.2,-2.2)(2.2,2.2)
 %\psgrid

 \pscircle[linecolor=gdarkgray,%
 linestyle=dotted,%
 fillstyle=solid
 ]%
 (0,0){2}
 
 \psdot(0,0)
 \psdot(-0.8,0.8)
 \psdot(0.6,1)
 
 \psline[linecolor=darkblue,arrows=->](-0.8,0.8)(0.6,1)
 \psline[linecolor=darkblue,arrows=->](0,0)(-0.8,0.8)
 \psline[linecolor=darkblue,arrows=->](0,0)(0.6,1)
 
 \psline[arrows=->](0,0)(-1.41,-1.41)
 
 \rput[lb](0.2,0){\color{gdarkgray}$z_0$}
 \rput[rt](-0.5,0.5){\color{darkblue}$\gamma_{z_0,z}$}
 \rput[lt](-0.4,1.3){\color{darkblue}$\gamma_{z,w}$}
 \rput[lt](0.6,0.6){\color{darkblue}$\gamma_{z_0,w}$}
 \rput[rb](-0.8,0.9){\color{gdarkgray}$z$}
 \rput(0.8,1.2){\color{gdarkgray}$w$}
 
 \rput(-1.4,-1){\color{gdarkgray}$r$}
\end{pspicture}
\end{center}
Sei $\gamma_{z,w}(t) = z+t(w-z), 0\le t\le 1$. Aus der Voraussetzung folgt
\begin{align*}
\int\limits_{\gamma_{z_0,w}} f(\zeta)\dzeta - \int\limits_{\gamma_{z,w}}
f(\zeta)\dzeta - \int\limits_{\gamma_{z_0,z}} f(\zeta)\dzeta = 0.
\end{align*}
Setze $F(z) := \int\limits_{\gamma_{z_0,z}} f(\zeta)\dzeta$, dann ist
\begin{align*}
\abs{F(w)-F(z)} = \abs{\int\limits_{\gamma_{z_0,w}} f(\zeta)\dzeta -
\int\limits_{\gamma_{z_0,z}} f(\zeta)\dzeta}
= \abs{\int\limits_{\gamma_{z,w}} f(\zeta)\dzeta}.
\end{align*}
Damit können wir den Differenzenquotienten abschätzen,
\begin{align*}
\abs{\frac{F(w)-F(z)}{w-z} - f(z)} &= \frac{1}{\abs{w-z}}
\abs{\int\limits_{\gamma_{z,w}} f(\zeta) - f(z)\dzeta}
 \\ &\le \frac{1}{\abs{w-z}}\max_{\zeta\in\im\gamma_{z,w}}
 \abs{f(\zeta)-f(z)} \underbrace{L(\gamma_{z,w})}_{\abs{w-z}} < \ep,
\end{align*}
für $0 < \abs{w-z} < \delta$, da $f$ stetig ist.\qedhere
\end{proof}

\begin{proof}[Teilbeweis \ref{prop:2.25}]
\ref{prop:2.25.3} $\Rightarrow$ \ref{prop:2.25.1}: Die Voraussetzungen für den
Satz von Morera sind erfüllt und damit folgt \ref{prop:2.25.1}.\qedhere
\end{proof}

\begin{bem}[Bemerkungen]
\label{bem:2.43}
\begin{enumerate}
  \item Wie wir im Beweis von Satz \ref{prop:2.42} gesehen haben, hat eine
  Funktion $f: O\to\C$ genau dann eine Stammfunktion, wenn sie differenzierbar
  ist. Im reellen Fall war Stetigkeit bereits mehr als genug.
  \item Die Existenz einer lokalen Stammfunktion von $f$ impliziert nicht die
  Existenz einer Stammfunktion auf ganz $O$ (``globale Stammfunktion'').\maphere
\end{enumerate}
\end{bem}
 \begin{bspn}
  $f(z)=z^{-1}$ auf $O = \C\setminus\{0\}$ ist holomorph, besitzt aber keine
  Stammfunktion auf $O$. Später werden wir sehen, dass in der ``geschlitzten''
  komplexen Ebene
  \begin{align*}
  \C \setminus(-\infty,0],
  \end{align*}
   $\ln z$ eine globale Stammfunktion von $z^{-1}$ ist.\bsphere
  \end{bspn}

  \begin{prop}
  \label{prop:2.44}
  Sei $f:O\to\C$, $z_0 = x_0+iy_0\in O$, dann sind folgende Aussagen äquivalent
  \begin{enumerate}[label=(\roman{*})]
    \item\label{prop:2.44.1} $f$ ist differenzierbar in $z_0$.
    \item\label{prop:2.44.2} $(u,v): \tilde{O}\to\R^2$ ist in $(x_0,y_0)$
    differenzierbar und es gelten die Cauchy-Riemannschen Differentialgleichungen in $(x_0,y_0)$
    \begin{align*}
    u_x = v_y, u_y = -v_x.
    \end{align*}
  \end{enumerate}
  Sind \ref{prop:2.44.1} und \ref{prop:2.44.2} erfüllt, so gilt
  \begin{align*}
  &u_x(x_0,y_0) = \Re f'(z_0),\\
  &v_y(x_0,y_0) = -\Im f'(z_0).\fishhere
  \end{align*}
  \end{prop}
\begin{proof}
Sei $f$ differenzierbar in $z_0$, dann gilt
\begin{align*}
f(z) = f(z_0) + f'(z_0)(z-z_0) + o(z-z_0).
\end{align*}
Dies ist äquivalent mit
\begin{align*}
\Re f(z) &= \Re f(z_0) + \Re f'(z_0)\Re (z-z_0) - \Im f'(z_0)\Im (z-z_0) +
o(z-z_0),\\
\Im f(z) &= \Im f(z_0) + \Im f'(z_0)\Re (z-z_0) + \Re f'(z_0)\Im (z-z_0) -
o(z-z_0),
\end{align*}
als Vektoren geschrieben
\begin{align*}
\begin{pmatrix}
\Re f(x+iy)\\
\Im f(x+iy)
\end{pmatrix}
&=
\begin{pmatrix}
\Re f(z_0)\\
\Im f(z_0)
\end{pmatrix}
+
\begin{pmatrix}
\Re f'(z_0) & -\Im f'(z_0)\\
\Im f'(z_0) & \Re f'(z_0)
\end{pmatrix}
\begin{pmatrix}
x-x_0\\
y-y_0
\end{pmatrix}
\\ &+
o
\begin{pmatrix}
\abs{
\begin{pmatrix}
x\\y
\end{pmatrix}
-
\begin{pmatrix}
x_0\\y_0
\end{pmatrix}
}
\end{pmatrix},
\end{align*}
oder äquivalent
\begin{align*}
\begin{pmatrix}
u\\v
\end{pmatrix}(x,y)
&=
\begin{pmatrix}
u\\v
\end{pmatrix}(x_0,y_0)
+
\begin{pmatrix}
\Re f'(z_0) & -\Im f'(z_0)\\
\Im f'(z_0) & \Re f'(z_0)
\end{pmatrix}
\begin{pmatrix}
x-x_0\\
y-y_0
\end{pmatrix}
\\ &+
o
\begin{pmatrix}
\abs{
\begin{pmatrix}
x\\y
\end{pmatrix}
-
\begin{pmatrix}
x_0\\y_0
\end{pmatrix}
}
\end{pmatrix}.
\end{align*}
Aber dies ist gerade äquivalent mit $(u,v)$ ist differenzierbar in $(x_0,y_0)$
und die Jacobimatrix ist gegeben als
\begin{align*}
\begin{pmatrix}
u_x & u_y\\ v_x  & v_y 
\end{pmatrix}
=
\begin{pmatrix}
\Re f'(z_0) & -\Im f'(z_0)\\
\Im f'(z_0) & \Re f'(z_0)
\end{pmatrix}\qedhere
\end{align*}
\end{proof}

\begin{proof}[Restbeweis von \ref{prop:2.25}]
\ref{prop:2.25.6} $\Rightarrow$ \ref{prop:2.25.1} Siehe lezter Beweis,\\
\ref{prop:2.25.1} $\Rightarrow$ \ref{prop:2.25.6} Die Differenzierbarkeit von
$f$ impliziert die von $(u,v)$ und da $f'$ stetig ist, sind es auch
$u_x,u_y,v_x,v_y$.\qedhere
\end{proof}

\begin{bsp}
\label{bsp:2.45}
Sei $f(z) = e^z$, dann ist $e^z = e^x(\cos y + i\sin y)$.
\begin{align*}
u(x,y) = e^x\cos y,\\
v(x,y) = e^x\sin y,
\end{align*}
\begin{align*}
&u_x = e^x\cos y,&&u_y = -e^x\sin y\\
&v_y = e^x\cos y, && v_x = e^x\sin y.
\end{align*}
Offensichtlich gilt hier Cauchy Riemann.\bsphere
\end{bsp}

\begin{defn}
\label{defn:2.46}
\begin{enumerate}
  \item $M\subseteq\C$ heißt \emph{(weg-)zusammenhängend}, falls zu je zwei
  Elementen $z_1,z_2\in M$ eine $C^1$ Kurve in $M$ existiert, mit $z_1$ als Anfangs- und
  $z_2$ als Endpunkt.
  \item $G\subseteq\C$ heißt \emph{Gebiet}, falls $G$ offen und
  zusammenhängend ist.\fishhere
\end{enumerate}
\end{defn}

\begin{prop}
\label{prop:2.47}
Sei $G\subseteq \C$ Gebiet, $f: \C\opento\C$ holomorph auf $G$ und $f'=0$ auf
$G$, dann ist $f$ auf $G$ konstant.\fishhere
\end{prop}
\begin{proof}
Sei $z_0\in G$. Zu $z\in G$ sei $\gamma$ $C^1$ Kurve von $z_0$ nach $z$, dann
gilt nach Voraussetzung,
\begin{align*}
0 = \int\limits_\gamma f'(\zeta)\dzeta = f(z) - f(z_0),
\end{align*}
also ist $f(z) = f(z_0)$ für $z\in G$.\qedhere
\end{proof}

\begin{prop}
\label{prop:2.48}
Sei $G\subseteq\C$ ein Gebiet, $f:\C\opento\C$ holomorph in $G$. Existiert ein
$z_0\in G$, so dass $f^{(n)}(z_0) = 0$ für alle $n\in \N$, dann ist $f$ auf $G$
konstant.\fishhere
\end{prop}
\begin{proof}
Erfülle $z_0\in G$ die Voraussetzung. Zu $z\in G$ sei $\gamma\in C^1([0,1]\to
G)$ mit $\gamma(0) = z_0$ und $\gamma(1) = z$.

\begin{center}
\psset{unit=1cm}
\psset{linecolor=gdarkgray}
\psset{fillcolor=glightgray}
\begin{pspicture}(0,0)(5,4)

 %\psgrid

 \psline[fillstyle=none,%
 linestyle=none]%
 (-1,1)(-4,1)(-4,3)(-1,3)(-1,1)
%  
%  \psline[arrows=<-](-1,2)(-1,1)(-2.5,1)
%  \psline[arrows=<-](-2.5,1)(-4,1)(-4,2)
%  \psline[arrows=<-](-4,2)(-4,3)(-2.5,3)
%  \psline[linecolor=gdarkgray,arrows=<-](-2.5,3)(-1,3)(-1,2)
% 
 % Kartoffel
 \psccurve[fillstyle=solid,%
 linestyle=dotted]%
 (1,0.5)(1,3)(3,2.5)(4.5,3)(4.5,1)(3,1)
 
 \pscircle(1.6,1.8){1.06}
 
 \psdot(1.6,1.8)
 \psdot(4.2,1.8)

 
 \psbezier[linecolor=darkblue,arrows=->]%
 	(1.6,1.8)(2.2,2.6)%
 	(3.2,1)(4.2,1.8)

 \psbezier[linecolor=gdarkgray]%
 	(2,2.6)(2.2,2.8)%
 	(2,3)(2.2,3.2)
 	
  \psdot(3,1.67) 	
 	
 \rput(1.8,1.6){\color{gdarkgray}$z_0$}
 \rput(4.4,1.8){\color{gdarkgray}$z$}
 \rput[lt](3.4,1.4){\color{darkblue}$\gamma$}
 
 \rput[l](2.2,3.2){\color{gdarkgray}$f=\text{const}$}
 \rput(3.2,2){\color{gdarkgray}$\gamma(T)$}
%  
%  \psbezier[linecolor=darkblue,arrows=->]%
% 	(-1.4,3.2)(-0.5,3.5)%
% 	(0.5,3.5)(1.2,2.1)
% 
% \pscustom[linecolor=gdarkgray]{%
%  \psbezier[liftpen=2]%
% 	(1,1)(1,1.5)(2,1.5)(2,2)
%  \psbezier[liftpen=2]%
% 	(2,2)(2.25,2.3)(2.75,1.8)(3,1.9)
%  \psbezier[liftpen=2]%
% 	(3,1.9)(3,1.4)(2,1.4)(2,0.9)
%  \psbezier[liftpen=2]%
% 	(2,0.9)(1.75,0.8)(1.25,1.3)(1,1)
% }
% 
%  \rput(0,3.8){\color{gdarkgray}$\ph$}
%  \rput(1.2,1.8){\color{gdarkgray}$\ph\circ\gamma$}
%  \rput(-2.5,2){\color{gdarkgray}$R$}
%  \rput(5.2,3){\color{gdarkgray}$O$}
%  \rput[l](-0.8,1){\color{gdarkgray}$\gamma$}
\end{pspicture}
\end{center}

\begin{enumerate}[label=\arabic{*}.)]
  \item\label{proof:2.46.1} Nach Satz \ref{prop:2.25} hat
  $f$ in $z_0$ postiven Konvergenzradius $R$, also gilt
  \begin{align}
  f(z) = \sum\limits_{n=0}^\infty a_n (z-z_0)^n = \sum\limits_{n=0}^\infty
  \frac{f^{(n)}(z_0)}{n!} (z-z_0)^n = f(z_0),
  \end{align}
  für $\abs{z-z_0}<R$.
  
  Wählen wir nun $z\in G$ beliebig aber fest und $\gamma:=\gamma_{z_0,z}$. Da
  $\gamma$ stetig ist, existiert ein $\delta > 0$, so dass für $0\le t\le
  \delta$ gilt $\abs{\gamma(t)-z_0}<R$. Da $f$ ebenfalls stetig ist, folgt $f\circ\gamma(t) = f(z_0)$ für $0\le t\le \delta$.
  \item Sei $T:=\sup\setdef{t\in[0,1]}{f\circ\gamma(\tau) = f(z_0),\;
  0\le\tau\le t}$. Aus \ref{proof:2.46.1} wissen wir, $T\ge\delta$.
  \item Ist $T=1$, dann folgt $f(z) = \lim\limits_{t\to 1} f\circ\gamma(t) =
f(z_0)$, also ist $f(z) = f(z_0)$ und wir sind fertig.
    
Nun müssen wir $T<1$ zu einem Widerspruch führen.

Wir wissen bereits $f\circ\gamma(T) = f(z_0)$. Gilt nun $f^{(n)}\circ\gamma(t)
= 0$ für $0\le t\le T$ und alle $n\in\N$, dann liefert dasselbe Argument wie in
\ref{proof:2.46.1} ein $\delta > 0$ für das gilt
\begin{align*}
f\circ\gamma(t) = f\circ\gamma(T) = f(z_0),\quad\text{für } T\le t \le
T+\delta,
\end{align*}
was ein Widerspruch zur Maximalität von $T$ wäre, also kann $T$ nicht kleiner 1
sein.

Wir zeigen nun $f^{(n)}\circ\gamma(t)
= 0$ für $0\le t\le T$ per Induktion.

Sei also $t_0\in[0,T]$ beliebig.

Induktionsanfang:
Sei $\abs{\gamma(t)-z_0} < R$, dann ist
$f'\circ\gamma(t) = 0$, da $f$ auf $K_R(z_0)$ konstant ist.

Sei $\abs{\gamma(t)-z_0}\ge R$. Da $t\mapsto \abs{\gamma(t)-\gamma(t_0)}$ stetig
ist und
\begin{align*}
\abs{\gamma(0)-\gamma(t_0)} = \abs{z_0-\gamma(t_0)}\ge R \text{ sowie }
\abs{\gamma(t_0)-\gamma(t_0)} = 0,
\end{align*}
können wir den Zwischenwertsatz anwenden und eine Folge $(t_n)$ in
$[0,t_0]$ mit folgenden Eigenschaften konstruieren 
  \begin{align*}
  \abs{\gamma(t_n)-\gamma(t_0)} = \frac{R}{n}\quad
  \begin{cases}
  \to 0, & \text{für } n\to \infty\\
  \neq 0, & \text{für alle } n\in\N.
  \end{cases}
  \end{align*}
Da $f$ differenzierbar ist, stimmen folgende Grenzwerte überein:
\begin{align*}
f'\circ\gamma(t_0) &= \lim\limits_{z\to\gamma(t_0)}
\frac{f(z)-f\circ\gamma(t_0)}{z-\gamma(t_0)}
= \lim\limits_{n\to\infty}
\frac{f\circ\gamma(t_n)-f\circ\gamma(t_0)}{\gamma(t_n)-\gamma(t_0)} \\ &= 0,
\end{align*}
also ist $f'\circ\gamma(t_0) = 0$.

Induktionsschritt: Wir argumentieren genau wie im Induktionsanfang und
ersetzten $f$ durch $f^{(n)}$. Da $f$ holomorph in $G$ existiert der
Differenzenquotient auch für $f^{(n)}$.\qedhere
\end{enumerate}
\end{proof}

\subsection{Nullstellen}

In diesem Abschnitt werden wir Nullstellen holomorpher Funktionen betrachten,
die ebenfalls überraschende Eigenschaften haben. Ist nämlich $z_0$ Nullstelle
der Ordnung $K$ von $f: \C\opento\C$, so verhalten sich $f$
und ihre ``Umkehrfunktionen'' lokal wie die Abbildungen $z\mapsto (z-z_0)^K$ und
deren ``Umkehrfunktionen''.

\begin{defn}
\label{defn:2.49}
Sei $f:O\to\C$ holomorph, $z_0\in O$, $f(z_0) = 0$. Existiert ein $k\in\N$
mit $f^{(k)}(z_0) \neq 0$, dann nennt man
\begin{align*}
K = \min\setdef{k\in\N}{f^{(k)}(z_0)\neq 0},
\end{align*}
die \emph{Ordnung} oder \emph{Vielfachheit} der Nullstelle $z_0$
andernfalls heißt die Vielfachheit $\infty$.\fishhere
\end{defn}

Betrachtet man $f: z\mapsto z^2,\; re^{i\ph} \mapsto r^2e^{2i\ph}$,
\begin{center}
\psset{unit=1cm}
\psset{linecolor=gdarkgray}
\psset{tickcolor=gdarkgray}
\psset{fillcolor=glightgray}
\begin{pspicture}(-3,-3)(9,3)
 %\psgrid

 \psaxes[labels=none,ticks=none]{->}%
 (0,0)(-2.5,-2.5)(2.5,2.5)[\color{gdarkgray}$Re$,-90][\color{gdarkgray}$Im$,0]
 
 \psaxes[labels=none,ticks=none]{->}%
 (6,0)(3.5,-2.5)(8.5,2.5)[\color{gdarkgray}$Re$,-90][\color{gdarkgray}$Im$,0]
 
 \psarc[linecolor=yellow,arrows=->](0,0){2}{170}{360}
 \psarc[linecolor=darkblue,arrows=->](0,0){2}{0}{180}
 
 \psarc[linecolor=yellow,arrows=->](6,0){2.03}{5}{365}
 \psarc[linecolor=darkblue,arrows=->](6,0){2}{0}{360}
 
 \psbezier[arrows=->](2,1.4)(2.6,2)(3.4,2)(4,1.4)
 
 \rput[b](3,2){\color{gdarkgray}$f$}
 \rput(1.7,-1.7){\color{gdarkgray}$z$}
 \rput(7.7,-1.7){\color{gdarkgray}$z^2$}
\end{pspicture}
\end{center}
dann ist $z^2$ offensichtlich nicht injektiv, denn zu jedem
$w\in\C\setminus\{0\}$ gibt es zwei Urbilder.

Durch Einführung der ``Riemannschen Flächen'' lässt sich dieses Problem
umgehen. Dabei legt man zwei $\C\setminus\{0\}$ Ebenen übereinander, schneidet
sie jeweils entlang der positiven reellen Halbachse auf, verklebt den Rand für
$\Im z \uparrow 0$ der unteren Ebene mit dem Rand für $\Im z\downarrow 0$ der
oberen Ebene und dann die anderen beiden Ränder miteinander.\\
Dies ist in unserer Vorstellung nur dann möglich, wenn sich die Ebenen
durchdringen.

Betrachten wir $f$ nun auf dieser Fläche, als
\begin{align*}
f: \C\setminus\{0\} \to 
\setdef{re^{i\ph}}{r>0,\;\ph\in\R,\;e^{i(\ph+2\pi)}\neq e^{i\ph},\;
e^{i(\ph+4\pi)} = e^{i\ph}},
\end{align*}
dann ist $f$ bijektiv und holomorph. Dabei ist es nicht von Bedeutung, entlang
welcher Halbgeraden der Schnitt erfolgt ist.
Für $z\mapsto z^2$ hat die Riemannsche Fläche $2$ Blätter, für $z\mapsto z^n$
respektive $n$. Für den $\ln$ hat sie sogar unendlich viele Blätter.

\begin{prop}[Hilfssatz]
\label{prop:2.50}
Sei $0\le\ph_0\le 2\pi,$ $\C_{\ph_0} =
\setdef{\C\setminus\{re^{i\ph_0-\pi}\}}{r\ge0}$.
\begin{center}
\psset{unit=1cm}
\psset{linecolor=gdarkgray}
\psset{tickcolor=gdarkgray}
\psset{fillcolor=glightgray}
\begin{pspicture}(-3,-3)(3,3)

 %\psgrid

 \psline[fillstyle=solid,linestyle=none]%
 (-2.5,-2.5)(2.5,-2.5)(2.5,2.5)(-2.5,2.5)(-2.5,-2.5)
 \psaxes[labels=none,ticks=none]{->}%
 (0,0)(-2.5,-2.5)(2.5,2.5)[\color{gdarkgray}$Re$,-90][\color{gdarkgray}$Im$,0]
 
 \psxTick[linecolor=gdarkgray](2){\color{gdarkgray}$1$}
 \psyTick[linecolor=gdarkgray](2){\color{gdarkgray}$1$}
 
 \psline[linecolor=darkblue,arrows=*-](0,0)(-2.5,-2.5)
 
 \psarcn(0,0){1}{-135}{-180}
 
 \rput(-0.6,-0.2){\color{gdarkgray}$\ph_0$}
 \rput(2,2){\color{gdarkgray}$\C_{\ph_0}$}
\end{pspicture}
\end{center}
\begin{align*}
\arg_{\ph_0}(z) := \begin{cases}
\arg(z), & \text{für } -\pi + \ph_0 < \arg z < \pi,\\
\arg(z)+2\pi, & \text{für } -\pi \le \arg z < -\pi +\ph_0,
\end{cases}
\end{align*}
dann ist $\arg_{\ph_0} : \C_{\ph_0} \to (-\pi+\ph_0,\pi+\ph_0)$ stetig und
surjektiv.\fishhere
\end{prop}
\begin{proof}
$z\mapsto \abs{z}$ ist stetig und daher ist
\begin{align*}
\arg_{\ph_0}(z) = \begin{cases}
\arccos\frac{ \Re z}{\abs{z}},& \Im z > 0,\\
\arcsin \frac{\Im z}{\abs{z}},& \Re z > 0,\\
\arccos \frac{\Re z}{\abs{z}},& \Im z < 0,
\end{cases}
\end{align*}
mit geeigneter Addition von $\pm 2\pi$, ebenfalls stetig und surjektiv.\qedhere 
\end{proof}

\begin{prop}
\label{prop:2.51}
Sei $0\le\ph_0<2\pi$, $k\in\N$ und $z^{1/k} =
\abs{z}^{1/k}e^{i1/k\arg_{\ph_0}(z)}$, dann ist
\begin{align*}
\sqrt[k]{\cdot}: \C_{\ph_0} \to \C, 
\end{align*}
holomorph, und es gilt
\begin{align*}
\left(\sqrt[k]{z}\right)^k  = z,\;\left(\sqrt[k]{z}\right)' =
\frac{1}{k\left(\sqrt[k]{z}\right)^{k-1}}.\fishhere
\end{align*}
\end{prop}
\begin{proof}
\begin{enumerate}
  \item $(\sqrt[k]{z})^k = \abs{z} e^{i\arg_{\ph_0}(z)} = z$.
  \item Seien $f(z) = z^k,\;g(z) = z^{1/k}$, dann ist $f(z)$ holomorph und
  $g(z)$ nach Satz \ref{prop:2.50} stetig. Offensichtlich ist $f\circ g (z) =
  z$ und damit gilt
  \begin{align*}
  g'(z_0) &= \lim\limits_{\atop{z\to z_0,}{z\neq z_0}} \frac{g(z)-g(z_0)}{z-z_0}
  = \lim\limits_{\atop{z\to z_0,}{z\neq z_0}}
  \frac{1}{\frac{f(g(z))-f(g(z_0))}{g(z)-g(z_0)}} \\ &= \frac{1}{f'(g(z_0))} =
  \frac{1}{k g(z_0)^{k-1}} = \frac{1}{k\left(\sqrt[k]{z_0}\right)^{k-1}}.
  \end{align*}
Andere Zweige sind $z^{1/k} = \abs{z}^{1/k}e^{i1/k(\arg_{\ph_0}(z)+2\pi n)}$,
$n= 0,1, \ldots,k-1$.\qedhere
\end{enumerate}
\end{proof}

\begin{prop}
\label{prop:2.52}
Sei $f:O\to\C$ holomorph, $z_0\in O$ Nullstelle der Ordnung $k\in\N$.
Dann existiert ein $r> 0$ und eine holomorphe Funktion $h: K_r(z_0)\to\C$
mit $h(z_0)=0$, $h'(z_0)\neq 0$ und $f(z) = \left(h(z)\right)^k$ in
$K_r(z_0)$.\fishhere
\end{prop}

\begin{center}
\psset{unit=1cm}
\psset{linecolor=gdarkgray}
\psset{tickcolor=gdarkgray}
\psset{fillcolor=glightgray}
\begin{pspicture}(-3,-2)(3,4)

 %\psgrid

 \psaxes[labels=none,ticks=none]{->}%
 (0,0)(-2.5,-1.5)(2.5,3.5)[\color{gdarkgray}$Re$,-90][\color{gdarkgray}$Im$,0]
 
 \pscircle[linestyle=dotted](1,2){1.4}
 
 \psarc(0,0){0.4}{0}{62}
 \psarcn(0,0){0.4}{-118}{-180}
 
 \psline[linecolor=darkblue,arrows=-*](-0.5,-1)(1,2)
 
 \rput[lt](1,2){\color{gdarkgray}$g(0)$}
 %\rput[lt](-1,-2){\color{gdarkgray}$-g(0)$}
 \rput[rt](-0.5,-0.3){\color{gdarkgray}$\ph_0$}
 \rput[lt](0.5,0.4){\color{gdarkgray}$\arg g(0)$}
 
\end{pspicture}
\end{center}
\begin{proof}
Wir können ohne Einschränkung annehmen, dass $z_0 = 0$. Da $f$ holomorph,
existiert ein $R>0$, so dass gilt
\begin{align*}
f(z) = \sum\limits_{n=k}^\infty a_n z^n = z^k \underbrace{\left(a_k +
\sum\limits_{n=k+1}^\infty a_n z^{n-k}\right)}_{:= g(z)},\quad\text{für }
\abs{z} < R.
\end{align*}
Dann ist $g$ holomorph auf $K_R(z_0)$ und $g(0) = a_k \neq
0$. Wähle $r > 0$ mit
\begin{align*}
\abs{g(z)-g(0)} <\frac{\abs{g(0)}}{2},\quad\text{für }\abs{z} <r,
\end{align*}
dann ist $g(z)\in \C_{\ph_0}$ mit $\ph_0 = \arg(-g(0))$.

Sei $h(z) = \sqrt[k]{g(z)}z$, dann folgt
\begin{align*}
&h(z)^k = g(z) z^k = f(z),\\
&h(0) = 0\\
&h'(0) = \frac{1}{k\left(\sqrt[k]{g(0)}\right)^{k-1}}\cdot 0 +
\sqrt[k]{g(0)}\cdot 1 \neq 0,
\end{align*}
und $h$ ist, als Produkt und Verkettung holomorpher Funktionen,
holomorph.
\qedhere
\end{proof}

\begin{bsp}
\label{bsp:2.53}
Verschwindet die Ableitung einer Funktion $\R\to\R$ nirgends, ist sie
injektiv, bei Funktionen $\C\to\C$ gilt dies nicht mehr, denn für 
$f:\C\to\C,\; z\mapsto e^z$, ist $\abs{f'(z)}
= e^{\Re z} \neq 0$ für $z\in\C$ aber $f$ ist nicht injektiv, denn $e^{z+2\pi i} = e^z$.\bsphere
\end{bsp}

\begin{prop}
\label{prop:2.54}
Sei $f:O\to\C$ holomorph, $z_0\in O$, $f'(z_0)\neq 0$. Dann existiert ein
$r>0$, so dass $f\big|_{K_r(z_0)}$ injektiv ist.

In diesem Fall ist $f(K_r(z_0))$ offen, die Umkehrfunktion $f^{-1}:
f(K_r(z_0))\to K_r(z_0)$ holomorph und es gilt
\begin{align*}
\left(f^{-1}\right)'(w) = \frac{1}{f'\circ f^{-1}(w)}.
\end{align*}
$f^{-1}$ heißt \emph{lokale Umkehrfunktion}.\fishhere
\end{prop}
\begin{proof}
Löse die Gleichung $f(z) = w = w_1+iw_2$ mit
\begin{align*}
u(x,y) &= \Re f(x+iy),\\
v(x,y) &= \Im f(x+iy),
\end{align*}
was äquivalent zu folgender Vektorgleichung ist
\begin{align*}
\begin{pmatrix}
g_1(x,y,w_1,w_2)\\
g_2(x,y,w_1,w_2)
\end{pmatrix}
= \begin{pmatrix}
  u(x,y)-w_1\\
  v(x,y)-w_2
  \end{pmatrix}
= \begin{pmatrix}
  0\\0
  \end{pmatrix}.
\end{align*}
Dann gilt $g_i\in C^1(\overline{K_r(z_0)}\to\R)$ und
\begin{align*}
\begin{vmatrix}
\partial_x g_1 & \partial_y g_1\\
\partial_x g_2 & \partial_y g_2\\
\end{vmatrix}
=
\begin{vmatrix}
u_x & u_y\\
v_x & v_y\\
\end{vmatrix}
= \begin{vmatrix}
\Re f' & -\Im f'\\
\Im f' & \Re f'\\
\end{vmatrix}
= \abs{f'}^2 \neq 0,
\end{align*}
für $z=z_0$. Außerdem gilt
\begin{align*}
&g_1(x_0,y_0,\Re f(z_0), \Im f(z_0)) = 0,\\
&g_2(x_0,y_0,\Re f(z_0), \Im f(z_0)) = 0.
\end{align*}
Nun können wir den Satz über implizite Funktionen anwenden, der die Existenz
eine Umgebung $\tilde{U}\subseteq \R^2$ von $\left(u(x_0,y_0),v(x_0,y_0)\right)$
und eine eindeutige Auflösung
\begin{align*}
x &= \ph_1(w_1,w_2),\\
y &= \ph_2(w_1,w_2),
\end{align*}
mit $\ph_1,\ph_2\in C^1\left(\tilde{U}\to\R\right)$ liefert. Damit ist
\begin{align*}
g_i(\ph_1(w_1,w_2),\ph_2(w_1,w_2),w_1,w_2) = 0,
\end{align*}
und $f$ lokaler Diffeomorphismus in $\tilde{U}$.\\
Insbesondere gilt $f^{-1}(w) := \ph_1(w_1,w_2)+i\ph_2(w_1,w_2)$ ist Umkehrfunktion, $f^{-1}$ ist
stetig und $f$ ist injektiv auf
$f^{-1}\left(\setdef{x+iy}{(x,y)\in\tilde{U}}\right)$.
\begin{enumerate}
  \item $f$ ist stetig $\Rightarrow$ $\exists r > 0 : K_r(z_0)\subseteq U =
  \setdef{x+iy}{(x,y)\in \hat{U}}$, insbesondere ist $f\big|_{K_r(z_0)}$
  injektiv.
  \item $f^{-1}$ ist stetig $\Rightarrow$ Urbilder offener Mengen sind offen
  $\Rightarrow$ $f(K_r(z_0)) = $ Urbild von $K_r(z_0)$ bezüglich $f^{-1}$ ist
  offen.
  \item $f^{-1}$ ist holomorph, denn
  \begin{align*}
  \lim\limits_{\atop{w\to w_0}{w\neq w_0}} \frac{f^{-1}(w)-f^{-1}(w_0)}{w-w_0}
  &= \lim\limits_{\atop{w\to w_0}{w\neq w_0}} \frac{z-z_0}{f(z)-f(z_0)}
  \\ &= \lim\limits_{\atop{z\to z_0}{z\neq z_0}}
  \frac{1}{\frac{f(z)-f(z_0)}{z-z_0}} = \frac{1}{f'(f^{-1}(w_0))}.\qedhere
  \end{align*}
\end{enumerate}
\end{proof}

\begin{prop}[Blätterzahl einer Nullstelle]
\label{prop:2.55}
Sei $f:O\to\C$ holomorph, $z_0\in O$ eine $k$-fache Nullstelle. Zu jedem
hinreichend kleinen $\ep > 0$ existiert eine offene Umgebung $O_{\ep}(z_0)$ mit
$f\left(O_{\ep}\right) = K_{\ep}(0)$.

$f\big|_{O_{\ep}}$ nimmt jeden Wert $w$ mit $0<\abs{w}<\ep$ genau $k$-mal an,
$w=0$ genau einmal.\fishhere
\end{prop}
\begin{proof}
Nach Satz \ref{prop:2.52} existiert ein $r>0$, so dass $f(z) = (h(z))^k$ mit $h$
holomorph auf $K_r(z_0)$ und $h(z_0) = 0,\;h'(z_0)\neq 0$.

Satz \ref{prop:2.54}: Wähle $\ep >0$ mit $K_{\ep}(0)\subseteq
f\left(K_r(z_0)\right)$ offen, dann ist $O_{\ep} :=
f^{-1}\left(K_{\ep}(0)\right)$.\qedhere
\begin{center}
\psset{unit=1cm}
\psset{linecolor=gdarkgray}
\psset{fillcolor=glightgray}
\begin{pspicture}(0,0)(12,4)

 %\psgrid

 \psline[fillstyle=none,%
 linestyle=none]%
 (-1,1)(-4,1)(-4,3)(-1,3)(-1,1)
%  
%  \psline[arrows=<-](-1,2)(-1,1)(-2.5,1)
%  \psline[arrows=<-](-2.5,1)(-4,1)(-4,2)
%  \psline[arrows=<-](-4,2)(-4,3)(-2.5,3)
%  \psline[linecolor=gdarkgray,arrows=<-](-2.5,3)(-1,3)(-1,2)
% 
 % Kartoffel
 \psccurve[fillstyle=none,%
 linestyle=dotted]%
 (1,0.5)(1,3)(3,2.5)(4.5,3)(4.5,1)(3,1)
 
 \pscircle[linestyle=dotted](7,1.8){1}
 \pscircle[linestyle=dotted](10.5,1.8){1.3}
 
 \psdot(4.2,1.8)
 \psdot(7,1.8)
 \psdot(10.5,1.8)
 
 \psline[arrows=->](7,1.8)(7.6,1)
 \psline[arrows=->](10.5,1.8)(11.43,0.87)
 	
 \rput(1.2,2.6){\color{gdarkgray}$O_\ep$}
 \rput(4.5,1.8){\color{gdarkgray}$z_0$}
 \rput(7.3,1.8){\color{gdarkgray}$0$}
 \rput(10.8,1.8){\color{gdarkgray}$0$}
 \rput(5.7,2.8){\color{gdarkgray}$h$}
 \rput[b](8.8,2.8){\color{gdarkgray}$z\mapsto z^k$}
 
 \psbezier[arrows=->,linecolor=darkblue]%
 (4.4,2)(4.8,2.8)(6.4,2.8)(6.8,2)
 \psbezier[arrows=->,linecolor=darkblue]%
 (7.2,2)(7.8,3)(9.7,3)(10.3,2)
 
 %\rput[lt](3.4,1.4){\color{darkblue}$\gamma$}
 
 %\rput[l](2.2,3.2){\color{gdarkgray}$f=\text{const}$}
 %\rput(3.2,2){\color{gdarkgray}$\gamma(T)$}
%  
%  \psbezier[linecolor=darkblue,arrows=->]%
% 	(-1.4,3.2)(-0.5,3.5)%
% 	(0.5,3.5)(1.2,2.1)
% 
% \pscustom[linecolor=gdarkgray]{%
%  \psbezier[liftpen=2]%
% 	(1,1)(1,1.5)(2,1.5)(2,2)
%  \psbezier[liftpen=2]%
% 	(2,2)(2.25,2.3)(2.75,1.8)(3,1.9)
%  \psbezier[liftpen=2]%
% 	(3,1.9)(3,1.4)(2,1.4)(2,0.9)
%  \psbezier[liftpen=2]%
% 	(2,0.9)(1.75,0.8)(1.25,1.3)(1,1)
% }
% 
%  \rput(0,3.8){\color{gdarkgray}$\ph$}
%  \rput(1.2,1.8){\color{gdarkgray}$\ph\circ\gamma$}
%  \rput(-2.5,2){\color{gdarkgray}$R$}
%  \rput(5.2,3){\color{gdarkgray}$O$}
%  \rput[l](-0.8,1){\color{gdarkgray}$\gamma$}
\end{pspicture}
\end{center}
\end{proof}

\begin{cor}
\label{prop:2.56}
Nullstellen endlicher Ordnung sind isoliert.\fishhere
\end{cor}

\begin{prop}[Satz von der inversen Abbildung]
\label{prop:2.57}
Seien $O_1,O_2\subseteq \C$ offen, $f: O_1\to O_2$ holomorph und bijektiv. Dann
gilt $f'(z)\neq 0$ für $z\in O_1$, $f^{-1}$ ist holomorph auf $O_2$ und
\begin{align*}
\left(f^{-1}\right)'(w) = \frac{1}{f'\circ f^{-1}(w)}.\fishhere
\end{align*}
\end{prop}
\begin{proof}
\begin{enumerate}[label=\arabic{*}.)]
  \item Wir zeigen $f' \neq 0$ mit einem Widerspruchsargument.
  
  Angenommen $\exists z_0\in O_1 : f'(z_0) = 0$. Betrachte $g(z) =
  f(z)-f(z_0)$, dann ist $z_0$ Nullstelle von $g$ mindestens $2$. Ordnung.
  \begin{enumerate}[label=Fall \alph{*})]
    \item Ordnung ist $\infty$, dann gibt es ein $r > 0: g(z) = 0$ in
    $K_r(z_0)$, also $f(z) = f(z_0)$ für $\abs{z-z_0}<r$ und $f$ wäre nicht
    injektiv.
    \item Ordnung $k\ge 2$, dann existiert ein $\ep > 0$, so dass
    $g\big|_{O_{\ep}}$ jeden Wert $w$ mit $0<\abs{w}<\ep$ $k$-mal annimmt, dann
    ist aber $f(z)-f(z_0)$ nicht injektiv.
  \end{enumerate}
  
  $z_0$ kann also keine Nullstelle von $f'$ sein.
  \item Da $f$ bijektiv ist, existiert ein $f^{-1} : O_2\to O_1$. Der Rest
  folgt aus \ref{prop:2.54}, indem man den Satz auf jeden Punkt in $O_1$
  anwendet.\qedhere
    \end{enumerate}
\end{proof}

\begin{prop}[Identitätssatz]
\label{prop:2.58}
Sei $G\subseteq \C$ Gebiet, $f,g$ holomorph in $G$ und $(z_n)$ Folge in $G$ mit
$z_n\to z_0\in G$, $z_n\neq z_0$, sowie $f(z_n)=g(z_n)
\forall n\in\N$, dann ist $f=g$ auf $G$.\fishhere
\end{prop}
\begin{proof}
Sei $h(z) = f(z)-g(z)$, dann folgt $h(z_0) = h(z_n) = 0$.
\begin{enumerate}[label=Fall \alph{*})]
  \item $z_0$ ist Nullstelle von $h$ mit endlicher Vielfachheit, dann ist $z_0$
  isolierte Nullstelle \dipper, da $z_n\to z_0$.
  \item $z_0$ hat Vielfachheit $\infty$, dann ist $h$ konstant, also $h(z) =
  h(z_0) = 0$.\qedhere
\end{enumerate}
\end{proof}

\begin{prop}[Gebietstreue]
\label{prop:2.59}
Sei $G\subseteq \C$ Gebiet, $f: \C\opento\C$ holomorph in $G$. Ist $f$ nicht
konstant auf $G$, dann ist $f(G)$ Gebiet.\fishhere
\end{prop}
\begin{proof}
\begin{enumerate}[label=\arabic{*}.)]
  \item $f(G)$ ist offen, denn sei $w_0=f(z_0)\in f(G)$. Setze $g(z) =
  f(z)-f(z_0)$. \begin{enumerate}[label=Fall \alph{*})]
  \item $z_0$ ist Nullstelle der Ordnung $\infty$.
  
  Dann ist $g=0$ auf ganz $G$, also ist $f$ konstant, ein Widerspruch.
  \item $z_0$ ist Nullstelle der Ordnung $K\in\N$
  
  Dann existiert ein $O_\ep \subseteq G$ mit $g(O_\ep) = K_\ep(g(z_0)) =
  K_\ep(0)$. Es gilt also
  \begin{align*}
  f\left(O_\ep\right) = g\left(O_\ep\right) + f(z_0) = K_\ep(f(z_0)),
  \end{align*}
  und damit enthält $f(G)$ mit $w_0=f(z_0)$ auch eine offene Umgebung
  $K_\ep(w_0)$.
\end{enumerate}
\item $f(G)$ ist zusammenhängend, denn seien $w_j = f(z_j)\in f(G)$ mit
$j=1,2$, dann existiert eine $C^1$-Kurve $\gamma$ von $z_1$ nach $z_2$ in $G$
und damit ist auch $f\circ\gamma$ eine $C^1$-Kurve von $f(z_1)=w_1$ nach
$f(z_2)=w_2$ in $f(G)$.

\scalebox{0.8}{
\begin{pspicture}(0,-3.624844)(13.131146,3.6448438)
\psbezier[linestyle=dotted](0.6751052,-0.34484375)(1.1231707,-0.6703513)(2.5159776,-0.20307913)(3.4751053,-0.22484376)(4.4342327,-0.24660836)(5.3964148,-1.000268)(6.0151052,-0.52484375)(6.6337957,-0.049419545)(7.062335,1.3332928)(6.215105,1.9151562)(5.367875,2.4970198)(4.533017,1.7397419)(3.4951053,1.6951562)(2.4571931,1.6505706)(1.3902104,2.5131092)(0.6951052,1.8151562)(0.0,1.1172032)(0.22703972,-0.019336164)(0.6751052,-0.34484375)
\psbezier[linestyle=dotted](10.695106,3.0151563)(9.236911,3.0028481)(8.635105,2.3951561)(9.015105,1.3151562)(9.395105,0.23515625)(9.943622,0.31096286)(9.995105,-0.66484374)(10.046589,-1.6406504)(8.835105,-2.0448437)(9.435105,-2.8248436)(10.035105,-3.6048439)(11.959064,-3.2822642)(12.535105,-2.4648438)(13.111147,-1.6474233)(12.39263,-0.7962254)(12.155106,0.17515625)(11.917582,1.1465379)(12.153299,3.0274644)(10.695106,3.0151563)
\psbezier[arrows=->,linecolor=darkblue](5.9551053,2.5751562)(6.735105,3.3351562)(7.9351053,3.3751562)(8.635105,2.6151562)
\psbezier[arrows=*-*,linecolor=darkblue](0.9751052,0.67515624)(1.8873415,1.3351562)(2.6841855,1.0419565)(3.4789026,0.6351563)(4.27362,0.22835596)(5.075105,0.09515625)(5.695105,0.9351562)
\psbezier[arrows=*-*,linecolor=darkblue](9.995105,2.1151562)(9.675105,1.3551563)(10.175105,0.77515626)(10.855105,0.21515626)(11.535105,-0.34484375)(11.315105,-1.6048437)(10.635105,-2.1248438)

\rput(7.3465114,3.4651563){\color{gdarkgray}$f$}

\rput(6.736199,1.9451562){\color{gdarkgray}$G$}

\rput(12.41073,2.1851563){\color{gdarkgray}$f(G)$}

\rput(6.0540113,0.76515627){\color{gdarkgray}$z_2$}

\rput(0.9594802,0.38515624){\color{gdarkgray}$z_1$}

\rput(3.6565115,0.90515625){\color{gdarkgray}$\gamma$}

\rput(10.33198,2.2851562){\color{gdarkgray}$w_1$}

\rput(10.846512,-2.3748438){\color{gdarkgray}$w_2$}

\rput(11,0.72515625){\color{gdarkgray}$f\circ\gamma$}
\end{pspicture}
}

\hfill\qedhere
\end{enumerate}
\end{proof}

\begin{prop}[Maximumsprinzip I]
\label{prop:2.60}
Sei $G\subseteq \C$ Gebiet und $f: \C\opento\C$ holomorph in $G$. $f$ ist
konstant auf $G$, falls eine der folgenden Bedingungen erfüllt ist:
\begin{enumerate}
  \item $\exists z_1\in G : \forall z\in G : \abs{f(z)}\le \abs{f(z_1)}$,
  \item $\exists z_2\in G : \forall z\in G : \abs{f(z)}\ge
  \abs{f(z_2)}$ und $f(z_2)\neq 0$.\fishhere
\end{enumerate}
\end{prop}
\begin{proof}
Angenommen es existiert ein $z_1\in G$, so dass $\forall z\in G: \abs{f(z)}\le
\abs{f(z_1)}$. Ist $f$ konstant, sind wir fertig, andernfalls ist $f(G)$ nach
\ref{prop:2.59} offen. Aber $f(z_1)$ liegt auf dem Rand und
kann daher kein innerer Punkt von $f(G)$ sein, also wäre $f(G)$ nicht offen und
damit muss $f$ konstant sein.

\begin{pspicture}(0,-2.554844)(11,2.8)
 \psaxes[labels=none,ticks=none]{->}%
 (8,0)(5.5,-2.5)(10.5,2.5)[\color{gdarkgray}$\Re$,-90][\color{gdarkgray}$\Im$,0]
\psbezier[linestyle=dotted](1.5555631,1.8207812)(0.5737694,1.7787391)(0.0,0.56078124)(0.58,-0.0992187)(1.16,-0.75921863)(0.2,-1.4992187)(1.16,-2.0192187)(2.12,-2.5392187)(3.291799,-1.3174471)(2.88,-0.79921865)(2.468201,-0.2809903)(2.1630082,-0.051417217)(2.74,0.68078125)(3.3169918,1.4129797)(2.5373569,1.8628234)(1.5555631,1.8207812)
\psdots[dotsize=0.13,linecolor=darkblue](2.08,1.1607813)
\pscircle(8.0,-0.00437495){2.0}
\psbezier[linestyle=dotted](8.234167,1.7592187)(6.94,1.2607814)(7.26,0.40078124)(6.9141665,-0.06078135)(6.568333,-0.52234393)(6.1426973,-1.1301303)(7.3141665,-1.4807813)(8.485636,-1.8314325)(8.04,-0.8192187)(8.274167,-0.5407814)(8.508333,-0.262344)(8.927741,-0.6442705)(9.094167,-0.18078135)(9.2605915,0.28270775)(9.528334,2.2576559)(8.234167,1.7592187)
\psline{->}(8.0,-0.00437495)(6.8541665,-1.5807813)
\psbezier[linecolor=darkblue]{->}(2.96,1.7207813)(4.22,2.4407814)(5.7,2.3007812)(6.68,1.2207813)

\rput{50.8}(2.0945463,-5.84648){\rput(7.3240037,-0.6480363){\color{gdarkgray}$\abs{f(z1)}$}}

\rput(9.1770315,-0.8292186){\color{gdarkgray}$f(G)$}

\rput(9.497031,1.9707813){\color{gdarkgray}$f(z1)$}

\rput(0.58250004,1.7507813){\color{gdarkgray}$G$}

\rput(4.6728125,2.3907812){\color{gdarkgray}$f$}

\rput(2.2771873,0.92640615){\color{gdarkgray}$z_1$}
\psdots[dotsize=0.13,linecolor=darkblue](8.834167,1.7992187)
\end{pspicture}
\hfill\qedhere 
\end{proof}

\begin{prop}[Maximumsprinzip II]
\label{prop:2.61}
Sei $G\subseteq \C$ beschränktes Gebiet, $f: \C\opento\C$ holomorph in $G$ und
$f\in C(\overline{G}\to \C)$, dann nimmt $\abs{f}$ das Maximum auf
dem Rand an.
\begin{align*}
\exists z_0\in \partial G : \forall z\in G : \abs{f(z)}\le
\abs{f(z_0)},
\end{align*}
Ist außerdem $\abs{f}>0$ auf $G$, so existiert
auch ein $z_1\in \partial G$ mit $\abs{f(z)}\ge\abs{f(z_1)},\;\forall z\in G$.\fishhere
\end{prop}
\begin{proof}
$\overline{G}$  kompakt, also nimmt die reellwertige Funktion $\abs{f(z)}$
ihr Minimum und Maximum an. Es gilt also
\begin{align*}
\exists z_1,z_2\in \overline{G} : \forall z\in G :
\abs{f(z_1)}\le\abs{f(z)}\le\abs{f(z_2)}.
\end{align*}
Ist $z_1\in\partial G$, sind wir fertig, ist andererseits $z_1\in G$, dann ist
$f$ auf $G$ konstant und da $f$ auf $\overline{G}$ stetig ist, ist $f$ auch
auf $\overline{G}$ konstant und man kann ein $\tilde{z}_1$ auf dem Rand
$\partial G$ wählen.

Analog verfährt man mit $z_2$.\qedhere
\end{proof}

\begin{bsp}
\label{bsp:2.62}
Sei $f(z) = e^z$ und $G=K_2(1+2i)$ das zu untersuchende Gebiet. Es gilt
$\abs{f(z)} = e^{\Re z}$, also ist $\abs{f}$ maximal bei $3+2i$ und minimal bei
$-1+2i$, also $e^{-1}\le \abs{f(z)}\le e^3$.
\begin{center}
\psset{unit=1cm}
\psset{linecolor=gdarkgray}
\psset{tickcolor=gdarkgray}
\psset{fillcolor=glightgray}
\begin{pspicture}(-2,-1)(4,4.5)
 %\psgrid
 \pscircle[linestyle=dotted,fillstyle=solid](1,2){1.98}
 
 \psaxes[labels=none,ticks=none]{->}%
 (0,0)(-1.5,-0.5)(3.5,4)[\color{gdarkgray}$Re$,-90][\color{gdarkgray}$Im$,0]
 
 \psdot(1,2)
 \psdot[linecolor=darkblue](3,2)
 \psdot[linecolor=darkblue](-1,2)
 
 \rput[lt](1.1,2){\color{gdarkgray}$1+2i$}
 \rput[lb](2,1){\color{gdarkgray}$G$}
\end{pspicture}
\end{center}

\end{bsp}

\begin{bsp}
Sei $f: \R\to\R,\;x\mapsto c\abs{x-x_0}^4$.

Für $c > 0$ ist
\begin{align*}
f_+&:=f\big|_{[x_0, \infty)} \text{ injektiv und damit}\\
f_+^{-1} &: [0,\infty)\to [x_0,\infty),\; y\mapsto 
 \left(\frac{y}{c}\right)^\frac{1}{4}+x_0.
\end{align*}

Für $c<0$ ist
\begin{align*}
f_-&:=f\big|_{(-\infty, x_0]} \text{ injektiv und damit}\\
f_-^{-1} &: [0,\infty)\to (-\infty,x_0],\; y\mapsto 
 x_0 - \left(\frac{y}{c}\right)^\frac{1}{4}.\bsphere
\end{align*}
\begin{center}
\begin{pspicture}(-1,-1)(3,3)
 %\psgrid

 \psaxes[labels=none,ticks=none]{->}%
 (0,0)(-0.5,-0.5)(2.5,2.5)[\color{gdarkgray}$\Re$,-90][\color{gdarkgray}$\Im$,0]
 
 \psplot[linecolor=darkblue,algebraic=true]{-0.2}{2.2}{(x-1)^4}
 \psxTick(1){\color{gdarkgray}x_0}
  
 \rput(1.8,2.8){\color{gdarkgray}$y=c(x-x_0)^4$}
 
\end{pspicture}
\begin{pspicture}(-1,-3)(3,1)
 %\psgrid

 \psaxes[labels=none,ticks=none]{->}%
 (0,0)(-0.5,-2.5)(2.5,0.5)[\color{gdarkgray}$\Re$,-90][\color{gdarkgray}$\Im$,0]
 
 \psplot[linecolor=darkblue,algebraic=true]{-0.2}{2.2}{-(x-1)^4}
 
 \psxTick(1){\color{gdarkgray}x_0}
 
 \rput(1.8,0.8){\color{gdarkgray}$y=c(x-x_0)^4$}
 
\end{pspicture}
\end{center}
\end{bsp}

\begin{prop}
\label{prop:2.64}
Sei $f:(a,b)\to\R$, $x_0\in(a,b)$ und $f$ reell analytisch in $x_0$, d.h.
\begin{align*}
f(x) = y_0+\sum\limits_{n=K}^\infty a_n(x-x_0)^n,\quad\text{für } \abs{x-x_0}<r,
\end{align*}
mit $r>0,\;a_n\in\R,\;a_K\neq 0,\;K\ge 1$. Dann existiert ein $\ep>0$, so dass
\begin{align*}
f_+ := f\big|_{\left[x_0,x_0+\ep \right]}\text{ und }
f_- := f\big|_{\left[x_0-\ep,x_0 \right]},
\end{align*}
injektiv und $f_+^{-1}$, $f_-^{-1}$ als \emph{Puiseux-Reihen}
darstellbar sind,
\begin{align*}
f_{\pm}^{-1}(y) = x_0 + \sum\limits_{n=1}^\infty b_n
\left(\pm\abs{y-y_0}^{1/K}\right)^n,\quad\text{für }
y\in\begin{cases}
    f_+\left(\left[x_0,x_0+\ep\right] \right),\\
    f_-\left(\left[x_0-\ep,x_0\right] \right),
    \end{cases}
\end{align*}
mit geeignetem $b_n\in\R$ und $b_1 = \frac{1}{\abs{a_K}^{1/K}}\neq0$.\fishhere
\end{prop}
\begin{proof}
Sei $g(z):= \left(\sign a_K\right) \left(\sum\limits_{n=K}^\infty a_n(z-z_0)^n 
\right)$ für $z\in \C,\;\abs{z-z_0}<r$, dann ist $g$ holomorph in $K_r(x_0)$,
$z=x_0$ ist Nullstelle der Ordnung $K$ und
\begin{align*}
y_0 + \left(\sign a_K \right)g(x) = f(x),\quad x_0-r<x<x_0+r.
\end{align*}
Nach Satz \ref{prop:2.52} existiert ein $\ep >0$, so dass $g(z) = (h(z))^K$ für
$\abs{z-z_0}<\ep$ und
\begin{align*}
h(z) &= (z-x_0)\left(\underbrace{\sum\limits_{n=K}^\infty \left(\sign
a_K\right)a_K (z-x_0)^{n-K}}_{\left(\sign a_K\right) a_K = \abs{a_K} > 0\text{
für } z=x_0} \right)^{1/K},
\end{align*}
wobei $\left(\cdot\right)^{1/K}: \C\setminus(-\infty,0)\to \C$. Daraus folgt
\begin{itemize}
  \item $h'(x)\neq0$ also folgt mit \ref{prop:2.54}, dass $h\big|_{K_\ep(x_0)}$
  injektiv ist und damit $h^{-1}$ holomorph.
  \item $h(x)\in\R$ für $x\in\R\cap K_\ep(x_0)$.
  \item $h'(x_0) = 1\cdot\left(a_K\right)^{1/K}+0 = \left(a_K\right)^{1/K} > 0$.
  \item $h$ ist in $\R\cap K_\ep(x_0)$ streng monoton wachsend, also auch
  $h^{-1}$.
  \item $h^{-1}(z) = x_0 + \sum\limits_{n=1}^\infty b_n z^n$.
  \item $b_1 = \frac{(h^{-1})'(0)}{1} = \frac{1}{h'\circ h^{-1}(0)} =
  \frac{1}{h'(x_0)} = \frac{1}{\abs{a_K}^{1/K}}$.
  \item $b_n = \frac{\left(h^{-1}\right)^{(n)}(0)}{n!}\in\R$, da
  $h^{-1}(\R)\subseteq\R$.
\end{itemize}
Die Gleichung $f(x) = y$ ist äquivalent zu
\begin{align*}
y - y_0 = \left( \sign a_K \right) g(x) = \left( \sign a_K \right) (h(x))^K
\Leftrightarrow \frac{y-y_0}{\sign a_K} = h(x)^K,
\end{align*}
und da $h$ streng monoton wächst, gilt für $x\ge x_0$, dass $h(x)\ge h(x_0)$,
\begin{align*}
 \Rightarrow &h(x) = \abs{\frac{y-y_0}{\sign a_K}}^{1/K}
\\ \Rightarrow &x = h^{-1}\left(\abs{y-y_0}^{1/K} \right) = x_0 +
\sum\limits_{n=1}^\infty b_n \abs{y-y_0}^{1/K},
\end{align*}
analog dazu gilt für $x\le x_0$, dass $h(x)\le h(x_0)$,
\begin{align*}
 \Rightarrow &h(x) = -\abs{\frac{y-y_0}{\sign a_K}}^{1/K}
\\ \Rightarrow &x = h^{-1}\left(-\abs{y-y_0}^{1/K} \right) = x_0 -
\sum\limits_{n=1}^\infty b_n \abs{y-y_0}^{1/K}.\qedhere
\end{align*}
\end{proof}

Mit der Funktionentheorie können also auch sehr elegant Sätze über reelle
Funktionen bewiesen werden, obiger Beweis wäre ohne die hier entwickelte
Theorie nur mit erheblichem Aufwand möglich.
\subsection{Analytische Fortsetzung}

\begin{bsp}
\label{bsp:2.65}
Die Funktion $f(z) = \sum\limits_{n=0}^\infty (-z^2)^n$ ist holomorph in
$K_1(0)$ mit $f(z) = \frac{1}{1+z^2}$.

\begin{center}
\psset{unit=1cm}
\begin{pspicture}(-7,-4)(3.2,4.2)
 %\psgrid
 
 \psaxes[labels=none,ticks=none]{->}%
 (0,0)(-7,-4)(3,4)[\color{gdarkgray}$\Re$,-90][\color{gdarkgray}$\Im$,0]
 
 \pscircle[linestyle=dotted](0,0){2}
 
 \pscircle[linestyle=dotted](-1,0){2.23}
 
 \pscircle[linestyle=dotted](-3,0){3.60}
 
 \psdot[linecolor=darkblue](0,0)
 \psdot[linecolor=darkblue](-1,0)
 \psdot[linecolor=darkblue](-3,0)
 
 \rput(1.686875,1.69){\color{gdarkgray}$1$}
 \rput(-2.653125,1.93){\color{gdarkgray}$\frac{\sqrt{5}}{2}$}
 \rput(-5.673125,2.89){\color{gdarkgray}$\frac{\sqrt{13}}{2}$}
 
 
 \rput(0.177031,-0.27){\color{gdarkgray}$0$}
 \rput(-0.802969,-0.27){\color{gdarkgray}$-\frac{1}{2}$}
 \rput(-2.802969,-0.27){\color{gdarkgray}$-\frac{3}{2}$}
\end{pspicture}
\end{center}
Entwicklung von $f$ um $z=-\frac{1}{2}$ in eine Potenzreihe ergibt
\begin{align*}
f_1(z) = \sum\limits_{n=0}^\infty
a_n\left(z+\frac{1}{2}\right)^n,\quad\text{in }
K_{\sqrt{5}/2}\left(-\frac{1}{2}\right),
\end{align*}
mit $f=f_1$ in $K_1(0)\cap K_{\sqrt{5}/2}\left(-\frac{1}{2}\right)$.

Entwicklung von $f$ um $z=-\frac{3}{2}$ in eine Potenzreihe ergibt
\begin{align*}
f_2(z) = \sum\limits_{n=0}^\infty
a_n\left(z+\frac{3}{2}\right)^n,\quad\text{in }
K_{\sqrt{13}/2}\left(-\frac{3}{2}\right),
\end{align*}
mit $f_1=f_2$ in $K_{\sqrt{5}/2}\left(-\frac{1}{2}\right)\cap
K_{\sqrt{13}/2}\left(-\frac{3}{2}\right)$.\bsphere
\end{bsp}

\begin{defn}
\label{defn:2.66}
Ein Tupel $(K_0,\ldots,K_n)\equiv K$ offener Kreisscheiben $K_j = K_{r_j}(z_j)$
heißt \emph{Kreiskette}, falls
\begin{align*}
z_j\in K_{j-1} \text{ und } z_{j-1}\in K_j,\quad\text{für }j=1,\ldots,n.
\end{align*}
Gibt es holomorphe Funktionen $f_j: K_j\to\C$, mit
\begin{align*}
f_j\big|_{K_{j-1}\cap K_j} = f_{j-1}\big|_{K_{j-1}\cap K_j},\quad\text{für
}j=1,\ldots,n,
\end{align*}
so heißt $f$ \emph{analytisch fortsetzbar längs $K$}. $f_n$ heißt
\emph{analytische Fortsetzung von $f_0$ längs $K$}.\fishhere
\end{defn}

\begin{bem}
\label{bem:2.67}
Nach dem Identitätssatz ist $f_n$ eindeutig und damit sind es auch auch
$f_2,\ldots,f_{n-1}$.\maphere
\end{bem}

Nun stellt sich die Frage, ob für zwei Kreisketten $(K_0,\ldots,K_n)$ und
$(\tilde{K}_0,\ldots,\tilde{K}_n)$ mit $K_0 = \tilde{K}_0$ und $K_n =
\tilde{K}_n$ auch $f_n = \tilde{f}_n$ gilt. Wir werden sehen, dass dies im
Allgemeinen falsch ist, wir jedoch mit zusätzlichen Einschränkungen die
Gleichheit erzielen können.

\begin{defn}
\label{defn:2.68}
Sei $\gamma\in C\left([t_0,t_1]\to\C \right)$. Eine Kreiskette
$K=(K_0,\ldots,K_n)$ \emph{verläuft längs $\gamma$}, falls es eine Unterteilung
$t_0=\tau_0<\tau_1<\ldots<\tau_n=t_1$ gibt, so dass
\begin{align*}
&\gamma(\tau_j) = \text{Mittelpunkt von }K_j,\\
&\gamma\left([\tau_{j-1},\tau_j]\right)\subseteq K_{j-1}\cap K_j.\fishhere
\end{align*}
\end{defn}

\begin{prop}
\label{prop:2.69}
Sei $\gamma\in C\left([t_0,t_1]\to\C\right)$ und $K,\;K'$ offene Kreisscheiben
um $\gamma(t_0)$ bzw. $\gamma(t_1)$
\begin{center}
\psset{unit=1cm} 
\begin{pspicture}(0,-1.73)(8.389688,1.73)

\psbezier[linecolor=darkblue,arrows=*-*](0.84,-0.89)(1.62,1.03)(4.94,-1.53)(6.5,0.17)

\pscircle[linestyle=dotted](0.84,-0.89){0.82}
\pscircle[linestyle=dotted](6.5,0.17){1.56}

\rput(1.8671875,-0.83){\color{gdarkgray}$K$}
\rput(8.25375,0.65){\color{gdarkgray}$K'$}
\rput(3.1214063,0.04){\color{gdarkgray}$\gamma$}

\end{pspicture}
\end{center}
Sei $f$ holomorph in $K$ und $g,\;\tilde{g}$ seien in $K'$ holomorphe
analytische Fortsetzungen von $f$ längs Kreisketten, die längs $\gamma$
verlaufen, dann gilt $g=\tilde{g}$.\fishhere
\end{prop}
\begin{proof}
Sei $\tau_0<\ldots<\tau_n$ die Unterteilung und $f_0,\ldots,f_n$ die Funktionen
der ersten Kreiskette $K=(K_0,\ldots,K_n)$.

Für $t\in[\tau_{j-1},\tau_j]$ sei
\begin{align*}
p_t(z) = \sum\limits_{n=0}^\infty a_n(t) (z-\gamma(t))^n,\quad\text{in }
K_{r(t)}(\gamma(t)),
\end{align*}
die Potenzreihenentwicklung von $f_{j-1}$ um $\gamma(t)$.
% \begin{center}
% \begin{pspicture}(0,-1.91)(6.34,1.98)
% 
% \psbezier[linecolor=darkblue](0.0,-1.89)(1.58,0.01)(4.28,0.99)(6.32,0.89)
% 
% \psdots[linecolor=darkblue](1.24,-0.73)
% \psdots[linecolor=darkblue](2.94,0.23)
% \psdots[linecolor=darkblue](4.04,0.62)
% \pscircle[linestyle=dotted](2.94,0.23){1.72}
% 
% \rput(2.8214064,-0.09){\color{gdarkgray}$\gamma(t)$}
% \rput(4.2171874,0.37){\color{gdarkgray}$\gamma(\tau_j)$}
% \rput(0.6296875,-0.49){\color{gdarkgray}$\gamma(\tau_{j-1})$}
% 
% \rput(6.181406,1.15){\color{gdarkgray}$\gamma$}
% \end{pspicture} 
% \end{center}
$p_{\tau_j}$ ist doppelt definiert, nämlich durch
\begin{align*}
p_{\tau_j} &= \text{Entwicklung von }f_{j-1}\text{ um }\gamma(\tau_j)\text{
und}\\
p_{\tau_j} &= \text{Entwicklung von }f_{j}\text{ um }\gamma(\tau_j).
\end{align*}
Die Definitionen stimmen aber überein, denn $f_j = f_{j-1}$ im $K_j\cap
K_{j-1}$ und $\gamma(\tau_j)\in K_j\cap K_{j-1}$. Außerdem ist
$p_{\tau_j}\big|_{K_j} = f_j$. Es gilt aber noch mehr, denn da $\gamma$ stetig
ist, gibt es ein $\delta > 0$, so dass $\gamma(t')$ in $K_{r(t)}(\gamma(t))$
liegt für $\abs{t-t'}<\delta$.
Man erhält somit $p_{t'}$ durch Entwicklung von $p_t$ um
$\gamma(t')$.
\begin{align*}
p_t = f_{j},\quad\text{in } K_{j-1}\cap K_j \cap K_{r(t)}(\gamma(t)).
\end{align*}
Der Identitätssatz besagt nun, dass diese Entwicklung mit unserer Definition
von $p_{t'}$ übereinstimmt, denn $p_t = f_j$ in $K_j\cap
K_{r(t)}(\gamma(t))$. Diese Argumentation kann man auch auf weitere
Kreisscheiben fortsetzten, falls $t'$ ``zu weit'' von $t$ entfernt ist. In
einer kleinen Umgebung vom $t$ sind also alle Potenzreihen $p_t$ identisch.
(lokale Verträglichkeit)
% 
% Damit ist $p(t) = f_{j+1}$ in
% $K_{j+1}\cap K_{r(t)}(\gamma(t))$.

Entsprechend erhält man $\tilde{p}_t$ für die andere Kreiskette.

Sei nun $M = \setdef{t\in[t_0,t_1]}{p_t = \tilde{p}_t}$, dann folgt
\begin{itemize}
  \item $M\neq\varnothing$, denn $[t_0,\min\{\tau_1,\tilde{\tau}_1\}]\subseteq
  M$.
  \item $M$ ist relativ offen in $[t_0,t_1]$, denn falls $p_t = \tilde{p}_t$
  existiert ein $\delta > 0$, so dass $p_s = \tilde{p}_s$ für $t\le s <t+\delta$.
  \item $M$ ist abgeschlossen, denn sei $t'$ ein Häufungspunkt von $M$, dann
  stimmen $p_t$ und $\tilde{p}_t$ aufgrund der lokalen Verträglichkeit in jeder
  Umgebung von $t'$ überein und aufrgund des Identitätssatzes dann auch in $t'$.
%   \item $M$ ist abgeschlossen, denn sei $(s_n)$ Folge in $M$ mit $s_n\to s$ und
%   $s_n\neq s_m$, dann gilt
%   $p_{s_m} = \tilde{p}_{s_m} \Rightarrow p_s(s_m) = p_{s_m}(s) =
%   \tilde{p}_{s_m}(s) = \tilde{p}_s(s_m)$
%   und mit dem Identitätssatz folgt $p_s = \tilde{p}_s$.
\end{itemize}
$M$ ist also nichtleer und sowohl offen als auch abgeschlossen in der
Spurtopologie bezüglich $[t_0,t_1]$. Nun ist $[t_0,t_1]$ zusammenhängend und
daher gilt $M=[t_0,t_1]$. Damit ist $g = p_{t_1} = \tilde{p}_{t_1} = \tilde{g}$.\qedhere
\end{proof}

\begin{defn}
\label{defn:2.70}
Sei $\gamma\in C([t_0,t_1]\to\C)$, $K,K'$ offene Kreisscheiben um
$\gamma(t_0)$ bzw. $\gamma(t_1)$, $f$ holomorph in $K$, $g$ holomorph in $K'$.
Dann heißt $g$ \emph{analytische Fortsetzung von $f$ längs $\gamma$}, falls es
eine längs $\gamma$ verlaufende Kreiskette gibt, so dass $g$ analytische
Fortsetzung von $f$ längs dieser Kreiskette ist.\fishhere
\end{defn}

\begin{bsp}
\label{bsp:2.71}
Sei $N\in\N,\;N\ge 2$ und
\begin{align*}
f(z) &= \abs{z}^{1/N}e^{i1/N\arg z},\;z\in\C\setminus(-\infty,0],\\
\gamma(t) &= e^{i2\pi t},\;0\le t\le N.
\end{align*}
Die Funktion
\begin{align*}
g: \setdef{re^{i\ph}}{r > 0,\;-\frac{\pi}{N}<\ph<\frac{\pi}{N}} \to
\C\setminus(-\infty,0],\;z\mapsto z^N,
\end{align*}
ist holomorph und bijektiv und damit ist auch $f$ holomorph, denn $f = g^{-1}$.

Betrachten wir die Kreiskette $K_j = K_1\left(\gamma(\tau_j)\right)$ für
$j=0,\ldots,N$ und $\tau_j = \frac{j}{8}$, dann ist $f_0 = f_1 = f_2 = f$.
Für $f_3$ setzten wir $\tilde{f}(z) :=
\abs{z}^{1/N}e^{i\frac{1}{N}\arg_\pi(z)}$ mit
\begin{align*}
\arg_\pi = \begin{cases}
           \arg z, &\text{für } \arg z <\pi,\\
           \arg z + 2\pi, &\text{für } -\pi \le \arg z < 0,
           \end{cases}
\end{align*}
dann ist $\tilde{f} = g^{-1}$ für $g: \setdef{re^{i\ph}}{r > 0,\; 0<\ph<2\pi}\to
\C,\; z\mapsto z^N$ holomorph und
\begin{align*}
f_3 = \tilde{f}\big|_{K_3},\;f_4 = \tilde{f}\big|_{K_4},\;f_5 =
\tilde{f}\big|_{K_5},\;f_6 = \tilde{f}\big|_{K_6},
\end{align*}
Für $f_7$ muss beachtet werden, dass $\arg_\pi z$ stetig an $f$ anschließt.

Für $f_7$ wählt man $\tilde{\tilde{f}}:= f\cdot e^{i2\pi/N}$ und damit ist
\begin{align*}
f_7 = \tilde{\tilde{f}}\big|_{K_7},\;f_8 = \tilde{\tilde{f}}\big|_{K_8}.
\end{align*}
Nun ist $K_8 = K_0$ aber $f_8 = f\cdot e^{i2\pi/N}\neq f$. Nach $N$ Umläufen
stimmen die Funktionen auf den Kreisen wieder überein.\bsphere
\end{bsp}

\begin{lem}
\label{prop:2.72}
Sei $\gamma\in C([t_0,t_1]\to\C)$, $f_1$ analytische Fortsetzung von $f_0$
längs $\gamma$ und $\ph: [s_0,s_1]\to[t_0,t_1]$ stetig, streng monoton wachsend
mit $\ph(s_j)=t_j$. Dann ist $f_1$ auch analytische Fortsetzung von $f_0$ längs
$\gamma\circ\ph$. Insbesondere kann immer $s_0 = 0, s_1=1$ gewählt werden.
\end{lem}
\begin{proof}
$f_1$ ist analytische Fortsetzung längs einer Kreiskette
\begin{align*}
K=(K_{r_0}(\gamma(\tau_1)),\ldots,K_{r_n}(\gamma(\tau_n))).
\end{align*}
$K$ verläuft auch längs $\gamma\circ\ph$ und daher ist
\begin{align*}
\sigma_j = \ph^{-1}(\tau_j) \Rightarrow K =
(K_{r_0}(\gamma\circ\ph(\sigma_0)),\ldots,K_{r_n}(\gamma\circ\ph(\sigma_n))).\qedhere
\end{align*}
\end{proof}
\begin{prop}[Monodromiesatz]
Seien $\gamma,\tilde{\gamma}\in C([0,1]\to C)$ und $\gamma(0) =
\tilde{\gamma}(0)$, $\gamma(1)=\tilde{\gamma}(1)$ und $\phi\in
C([0,1]\times[0,1]\to\C)$ eine stetige Homotopie zwischen $\gamma $ und
$\tilde{\gamma}$, d.h.
\begin{align*}
\phi(\cdot,0) &= \gamma,\;\phi(\cdot,1) = \tilde{\gamma},\\
\phi(0,s) &= \phi(0,0),\;\phi(1,s) = \phi(1,0), \text{ für } 0\le s\le 1.
\end{align*}
Ist $f_0$ holomorph in $K_r(\gamma(0))$ und lässt sich $f_0$ längs jedem Weg
$\gamma_s:=\phi(\cdot,s)$ für $0\le s\le 1$ analytisch fortsetzen, so stimmen
die Fortsetzungen längs $\gamma$ und $\tilde{\gamma}$ überein.\fishhere
\end{prop}
\begin{proof}
Ohne Beweis.\qedhere
\end{proof}

%\begin{bspn}
%TODO: Beispiel mit Bildchen
%\end{bspn}

\begin{defn}
\label{defn:2.74}
Ein Gebiet  $G\subseteq \C$ heißt \emph{einfach zusammenhängend}, falls jede
geschlossene Kurve in $G$ nullhomotop ist oder äquivalent, falls je zwei Kurven
$\gamma$ und $\tilde{\gamma}$ mit gemeinsamem Anfangs- und Endpunkt homotop
sind.\fishhere
\end{defn}

\begin{bemn}[Erläuterung.]
Die Äquivalenz der Definition kann man Anhand folgender Homotopie sehen:
\begin{center}
\begin{pspicture}(-0.2,-1.8)(1.8228126,1.8)

\psbezier[arrows=*-*](0.909375,1.7)(0.429375,1.26)(0.069375,-1.32)(0.909375,-1.7)
\psbezier[arrows=*-*](0.909375,-1.7)(1.669375,-0.9)(1.389375,0.96)(0.909375,1.7)

\psline(0.289375,-0.18)(0.409375,-0.06)(0.529375,-0.18)
\psline(1.249375,-0.18)(1.369375,-0.06)(1.489375,-0.18)

\rput(1.6907812,0.01){\color{gdarkgray}$\gamma$}
\rput(0.07078125,-0.01){\color{gdarkgray}$\tgamma$}
\end{pspicture}
\begin{pspicture}(-0.2,-2.32)(5,1.8)

\psbezier(0.909375,1.44)(0.429375,1.0)(0.069375,-1.58)(0.909375,-1.96)
\psbezier(0.909375,-1.96)(1.669375,-1.16)(1.389375,0.7)(0.909375,1.44)
\psline(0.289375,-0.44)(0.409375,-0.32)(0.529375,-0.44)
\psline(1.4844714,-0.24753284)(1.377836,-0.37955132)(1.2458175,-0.27291584)


\psbezier[linecolor=darkblue,arrows=*-](0.909375,-1.96)(1.849375,-1.28)(2.829375,-2.3)(3.649375,-1.6)
\psbezier[linecolor=darkblue,arrows=*-](0.909375,1.46)(2.429375,2.3)(2.809375,-1.56)(3.649375,-1.6)

\psbezier[arrows=*-*](2.249375,0.82)(1.9171801,0.45502338)(1.369375,-1.4050668)(1.909375,-1.72)
\psbezier(1.909375,-1.72)(2.4107308,-1.03668)(2.5379937,0.24617568)(2.249375,0.84)
\psline(1.655818,-0.5421098)(1.7567927,-0.4612693)(1.8176286,-0.57541853)
\psline(2.4315078,-0.6658258)(2.3320737,-0.7485537)(2.2690985,-0.6355704)

\psdot[linecolor=yellow](3.649375,-1.6)

\rput(0.07078125,-0.27){\color{gdarkgray}$\gamma_1$}

\rput(2.3107812,-1.5){\color{gdarkgray}$\gamma_s$}

\rput(2.2707813,1.31){\color{darkblue}$\gamma_2$}

\rput[l](2.7107813,0.45){\color{darkblue}$\phi(\frac{1}{2},s)$}

\rput[l](2,-2.05){\color{darkblue}$\phi(0,s)=\phi(1,s)$}

\rput[l](3.8707812,-1.55){\color{yellow}$\phi(\cdot,1)$}
\end{pspicture} 
\end{center}

Hierbei entspricht $\gamma_1$, zunächst $\gamma$ vorwärts und dann $\tgamma$
rückwärts durchlaufen.

Für die Konstruktion der Homotopie gehe zu festem $s$ entlang $\phi(t,s)$ für
$0\le t\le s$, dann $\phi(s,t-s)$ für $s\le t\le s+\frac{1}{2}$, dann
$\phi(2s+\frac{1}{2}-t,\frac{1}{2})$ für $s+\frac{1}{2} \le t \le
2s+\frac{1}{2}$. Eventuell muss umparametrisiert werden, so dass $t$ zwischen
$0$ und $1$ läuft.\maphere
\end{bemn}

\begin{bspn}
\begin{enumerate}
  \item $\C\setminus\{0\}$ ist nicht einfach zusammenhängend.
  \item $\C\setminus(-\infty,0]$ ist einfach zusammenhängend.
  \item Jede Kreisscheibe $K_r(z_0)$ mit $z_0\in\C$ ist
  einfach zusammenhängend.\bsphere
\end{enumerate}
\end{bspn}

\begin{cor}
\label{prop:2.75}
Sei $G\subseteq \C$ einfach zusammenhängendes Gebiet, $K_r(z_0)\subseteq G$ und
$f_0$ holomorph in $K_r(z_0)$, lässt sich $f_0$ längs jeder stetigen Kurve
in $G$ mit Anfangspunkt $z_0$ analytisch fortsetzen. Dann gibt es genau eine in
$G$ holomorphe Funktion $f$ mit $f\big|_{K_r(z_0)} = f_0$.\fishhere
\end{cor}
\begin{proof}
\begin{bemn}[Eindeutigkeit.]
Folgt aus dem Identitätssatz  und $f\big|_{K_r(z_0)} = f_0$.
\end{bemn}
\begin{bemn}[Existenz.]
Wähle zu jedem $z_1\in G$ eine Kurve $\gamma$ in $G$ von $z_0$ nach
$z_1$, dann ist $f_0$ längs $\gamma$ analytisch fortsetzbar zu $f_1$ und $f_1$
ist holomorph in $K_{r_1}(z_1)$. Definiere $f(z_1) := f_1(z_1)$. Der
Monodromiesatz besagt nun, dass $f_1(z_1)$ unabhängig von der Wahl der Kurve
$\gamma$ und damit $f$ in jedem $z_1\in G$ eindeutig definiert ist.

Nun ist noch zu zeigen, dass $f$ holomorph ist. Sei $z_1\in G$
fest und $f_1$, die soeben konstruierte Fortsetzung von $f_0$, holomorph in
$K_{r_1}(z_1)\subseteq G$.

Zeige $f\big|_{K_{\frac{r_1}{3}}(z_1)} = f_1\big|_{K_{\frac{r_1}{3}}(z_1)}$,
dann ist $f$ holomorph in $K_{\frac{r_1}{3}}(z_1)$.

Sei $z\in K_{\frac{r_1}{3}}(z_1)$, dann ist $\gamma_z$ stetige Kurve von $z_0$
nach $z$. Ergänze die Kreiskette längs $\gamma$ durch $K_{\frac{2r_1}{3}}(z)$
zu Kreiskette längs $\gamma_z$.

Dann ist $f(z)$ analytische Fortsetzung von $f_0$ längs $\gamma_z$ zu $f_z$ und
$f_z = f_1$ in $K_{\frac{2}{3}r_1}(z)$ und $f(z) = f_z(z) = f_1(z)$.\qedhere
\end{bemn}
\end{proof}

\addtocounter{prop}{2}

\begin{bsp}
\label{bsp:2.78}
Sei $f_0(z) = \abs{z}^{\frac{1}{N}}e^{i\frac{1}{N}\arg z}$ in $K_1(2)$ und
$G=\C_{\ph_0} = \C\setminus\setdef{re^{i(\ph_0-\pi)}}{r>0}$, dann ist 
$f(z) = \abs{z}^{\frac{1}{N}}e^{i\frac{1}{N}\arg_{\ph_0} z}$ in $G$.

Man nennt $f$ einen \emph{Zweig} der ``mehrdeutigen'' Umkehrfunktion $f$ von
$g(z) = z^N$. Weitere Zweige sind $f(z) =
\abs{z}^{\frac{1}{N}}e^{i\frac{1}{N}\arg_{\ph_0} z + 2k\pi}$.\bsphere
\end{bsp}

\subsection{Integrale längs geschlossener Kurven}

\begin{bsp}
\label{bsp:2.79}
Seien $f(z) = \sum\limits_{n=0}^\infty a_n z^n$ für $\abs{z}<R$ und $g(z) =
\sum\limits_{n=1}^\infty b_n z^n$ für $\abs{z}<r$ mit $\frac{1}{r}<R$, dann gilt
$h(z) = f(z) + g(\frac{1}{z})$ ist holomorph für $\frac{1}{r}<\abs{z}<R$ mit
\begin{align*}
h(z) &= \sum\limits_{n=0}^\infty a_n z^n + \sum\limits_{n=1}^\infty b_nz^{-n}
\\& = \sum\limits_{n=-\infty}^\infty c_nz^n\text{ mit } c_n :=\begin{cases}
a_n,& n\ge 0,\\
b_{-n},& n< 0
\end{cases}
\end{align*}
Diese Form von Reihe nennt man ``Laurent-Reihe''.\bsphere
\end{bsp}

\begin{prop}[Laurent-Entwicklung]
\label{prop:2.80}
Sei $0\le r<R$ und $K_{r,R} :=\setdef{z\in\C}{r<\abs{z-z_0}<R}$ und $f$
holomorph in $K_{r,R}(z_0)$. Dann gilt
\begin{align*}
f(z) = \sum\limits_{n=-\infty}^\infty a_n (z-z_0)^n, \text{ für }
r<\abs{z-z_0}<R,
\end{align*}
wobei
\begin{align*}
a_n = \frac{1}{2\pi i}\int\limits_{\abs{z-z_0}=\rho}
\frac{f(z)}{(z-z_0)^{n+1}}\dz, \text{ für } n\in\Z, r<\rho<R,
\end{align*}
\emph{Cauchy-Formel für Laurent Koeffizienten}.\fishhere
\end{prop}
\begin{bem}
\label{bem:2.81}
\begin{enumerate}[label=\arabic{*}.)]
  \item $H(z) = \sum\limits_{n=-\infty}^{-1} a_n (z-z_0)^n$ nennt man den
  \emph{Hauptteil}, $N(z) = \sum\limits_{n=0}^\infty a_n (z-z_0)^n$ den \emph{Nebenteil} der
  Laurentreihe.
  \item $r=0$ ist erlaubt. Dann  hat $f$ in $z_0$ eine isolierte Singularität.
  In diesem Fall konvergiert $H(z)$ für alle $z\in\C\setminus\{z_0\}$. $N(z)$
  konvergiert immer in $K_R(z_0)$.
  \item Falls $f$ holomorph in $K_R(z_0)$, gilt
  \begin{align*}
  a_n = \frac{1}{2\pi i}\int\limits_{\abs{z-z_0}=\rho} f(z)(z-z_0)^{-n-1}\dz =
  0, \text{für } -n\in\N.\maphere
  \end{align*}
\end{enumerate}
\end{bem}
\begin{proof}[Beweis Satz \ref{prop:2.80}]
Ohne Einschränkung können wir $z_0=0$ wählen. Sei $z\in K_{r,R}(0)$ fest, $\ep
= \min\left\{\frac{R-\abs{z}}{2},\frac{\abs{z}-r}{2}\right\}$.

\begin{center}
\scalebox{0.95}{
\begin{pspicture}(0,-2.0)(4.0,2.0)

\pscircle(2.0,0.0){2.0}
\pscircle(2.0,0.0){1.0}
\pscircle[linecolor=darkblue](2.64,1.34){0.38}
\psline[arrows=->](1.98,0.0)(0.94,-1.66)
\psline[arrows=->](1.98,0.0)(1.12,0.46)
\psdots[linecolor=darkblue](2.64,1.34)
\psdots(1.98,0.0)

\rput(2.7248437,1.185){\color{gdarkgray}\small $z$}
\rput(1.56,0.49){\color{gdarkgray}\small $r$}
\rput(1.3985938,-1.37){\color{gdarkgray}\small $R$}
\rput(2.1414063,1.73){\color{gdarkgray}\small $\gamma_1$}

\end{pspicture}
\begin{pspicture}(0,-2.0)(4.0,2.0)

\pscircle(2.0,0.0){2.0}
\pscircle(2.0,0.0){1.0}
\psline[arrows=->](1.98,0.0)(0.94,-1.66)
\psline[arrows=->](1.98,0.02)(1.12,0.46)
\psdots[linecolor=darkblue](2.64,1.34)
\psdots(1.98,0.0)

\psarc[linecolor=darkblue](2.0,0.0){1.1}{45.0}{85.0}
\psarc[linecolor=darkblue](2.0,0.0){1.88}{45.0}{85.0}
\psline[linecolor=darkblue](2.1,1.08)(2.18,1.88)
\psline[linecolor=darkblue](2.76,0.78)(3.32,1.34)

\rput(2.7248437,1.185){\color{gdarkgray}\small $z$}
\rput(1.56,0.49){\color{gdarkgray}\small $r$}
\rput(1.3985938,-1.37){\color{gdarkgray}\small $R$}
\rput(1.9214063,1.71){\color{gdarkgray}\small $\gamma_2$}

\end{pspicture} 
\begin{pspicture}(0,-2.0)(4.0,2.0)
\pscircle(2.0,0.0){2.0}
\pscircle(2.0,0.0){1.0}
\psline[arrows=->](1.98,0.0)(1.04,-1.6)
\psline[arrows=->](1.98,0.02)(0.98,0.36)
\psdots[linecolor=darkblue](2.64,1.34)
\psdots(1.98,0.0)

\psarc[linecolor=darkblue](2.0,0.0){1.1}{0.0}{358.0}
\psarc[linecolor=darkblue](2.0,0.0){1.88}{0.0}{358.0}
\psline[linecolor=darkblue](3.1,-0.06)(3.9,-0.06)
\psline[linecolor=darkblue](3.08,0.0)(3.9,0.0)

\psline[linecolor=darkblue](3.72,0.0)(3.56,0.1)
\psline[linecolor=darkblue](3.32,-0.06)(3.48,-0.14)
\psline[linecolor=darkblue](2.0,-1.1)(2.16,-1.18)
\psline[linecolor=darkblue](0.92,1.54)(1.12,1.54)

\rput(2.7248437,1.185){\color{gdarkgray}\small $z$}

\rput(1.54,0.37){\color{gdarkgray}\small $\rho_1$}

\rput(1.3985938,-1.45){\color{gdarkgray}\small $\rho_2$}

\rput(1.6614063,1.69){\color{gdarkgray}\small $\gamma_3$}

\end{pspicture} 
}
\end{center}

$\gamma_1\sim\gamma_2\sim\gamma_3$ in $K_{r,R}(z_0)\setminus\{z\}$.

Aus dem Cauchy Integralsatz und der gleichmäßigen Konvergenz der geometrichen
Reihe folgt,
\begin{align*}
2\pi if(z) &= \int\limits_{\abs{\zeta-z}=\ep} \frac{f(\zeta)}{\zeta -z} \dzeta
= \int\limits_{\gamma} \frac{f(\zeta)}{\zeta -z} \dzeta
\\&= \int\limits_{\abs{\zeta}=\rho_2} \frac{f(\zeta)}{\zeta -z} \dzeta
- \int\limits_{\abs{\zeta}=\rho_1} \frac{f(\zeta)}{\zeta -z} \dzeta
\\&= \int\limits_{\abs{\zeta}=\rho_2}
\frac{f(\zeta)}{\zeta}\frac{1}{1-\frac{z}{\zeta}}
\dzeta
-\int\limits_{\abs{\zeta}=\rho_1}
\frac{f(\zeta)}{-z}\frac{1}{1-\frac{\zeta}{z}}
\dzeta
\\&= \int\limits_{\abs{\zeta}=\rho_2}
\frac{f(\zeta)}{\zeta}\sum\limits_{n=0}^\infty \left( \frac{z}{\zeta}\right)^n
\dzeta
-\int\limits_{\abs{\zeta}=\rho_1}
\frac{f(\zeta)}{-z}\sum\limits_{n=0}^\infty \left( \frac{\zeta}{z}\right)^n
\dzeta
\\&= \sum\limits_{n=0}^\infty
\int\limits_{\abs{\zeta}=\rho_2}
\frac{f(\zeta)}{\zeta^{n+1}}
\dzeta z^n
+\sum\limits_{n=0}^\infty
\int\limits_{\abs{\zeta}=\rho_1}
\frac{f(\zeta)}{z^{n+1}}\zeta^n
\dzeta\\
&= \sum\limits_{n=0}^\infty
\underbrace{\int\limits_{\abs{\zeta}=\rho_2}
\frac{f(\zeta)}{\zeta^{n+1}}
\dzeta}_{:= a_n \text{ für } n\ge0} z^n
+
\sum\limits_{n=-\infty}^{-1}
\underbrace{\int\limits_{\abs{\zeta}=\rho_1}
\frac{f(\zeta)}{\zeta^{n+1}}\dzeta}_{:=a_n\text{ für }n<0} z^n.\qedhere
\end{align*}
\end{proof}

\begin{defn}
\label{defn:2.82}
\begin{enumerate}[label=\arabic{*}.)]
  \item Sei $f$ holomorph in $O$, dann hat $f$ in $z_0\in \C\setminus O$ eine
  \emph{isolierte Singularität}, falls gilt $\exists R>0 :
  K_{0,R}(z_0)\subseteq O$.
  \item $f$ habe in $z_0$ eine isolierte Singularität und sei als Laurent-Reihe
  darstellbar
  \begin{align*}
  f(z) = \sum\limits_{n=-\infty}^\infty a_n (z-z_0)^{n}, \text{ in }
  K_{0,R}(z_0),
  \end{align*}
  dann unterscheiden wir drei Fälle
\begin{enumerate}[label=(\roman{*})]
  \item Falls $a_n=0$ für $n<0$,  heißt die Singularität \emph{hebbar}
  (vgl. Riemannscher Hebbarkeitssatz).
  \item Falls $\exists N\in \Z\setminus\N_0 \forall n < N : a_n = 0$,  heißt
  $z_0$ \emph{Polstelle} von $f$. $f$ hat in $z_0$ einen \emph{Pol der Ordnung $m$},
  falls $a_{-m}\neq 0$ und $a_{-n}=0$ für $n>m$.
  \item Falls $\forall N\in\Z\setminus\N_0 : \exists n< N : a_n \neq 0$,  hat
  $f$ in $z_0$ eine \emph{wesentliche Singularität}.\fishhere
\end{enumerate}
\end{enumerate}
\end{defn}

\begin{bsp}
\label{bsp:2.83}
\begin{enumerate}
  \item Standardbeispiel für eine Funktion mit wesentlicher Singularität:
  \begin{align*}
  f(z) = e^{1/z} = \sum\limits_{n=0}^\infty \frac{1}{n!}z^{-n} = 1 + H(z),
  \end{align*} 
hat eine wesentliche Singularität in $z_0 = 0$.
\item $f(z) =\frac{p(z)}{q(z)}$ mit $p,q$ Polynome hat isolierte
Singularitäten in $z_1,\ldots,z_n$, diese sind genau die Nullstellen von $q$.

Vereinfachungen:
\begin{align*}
&\deg p < \deg q,\\
&p(z_j)\neq 0, j = 1,\ldots,n.
\end{align*}

Partialbruchzerlegung. Sei $m$ die Ordnung der Nullstelle $z_1$, dann gilt
\begin{align*}
\frac{p(z)}{q(z)} = \sum\limits_{j=1}^m \frac{c_j}{(z-z_1)^j} +
\underbrace{\frac{\tilde{p}(z)}{\tilde{q}(z)}}_{\text{holomorph in }
K_\ep(z_1)},
\end{align*}
mit $\tilde{p},\tilde{q}$ Polynome und $q(z) = (z-z_1)^m\tilde{q}(z),
\tilde{q}(z_1)\neq 0$ und $\deg \tilde{p} < \deg \tilde{q}$.

Der zweite Summand $\frac{\tilde{p}}{\tilde{q}}$ ist holomorph und Nebenteil der
Laurentreihe, der erste Summand ist Hauptteil und endlich.
$f$ hat also an der Stelle $z_1$ einen Pol der Ordnung $m$.\bsphere
\end{enumerate}
\end{bsp}

\begin{cor}
\label{prop:2.84}
$f: O\setminus\{z_0\}\to\C$ hat genau dann einen Pol der Ordnung $m$ in $z_0$,
falls
\begin{align*}
f(z) = \sum\limits_{n=-m}^\infty a_n (z-z_0)^n =
\frac{1}{(z-z_0)^m}\underbrace{\sum\limits_{n=-m}^\infty a_n (z-z_0)^{n+m}}_{:=
g(z)},
\end{align*}
und $a_{-m}\neq 0$. Also eine in $O$ holomorphe Abbildung $g$ existiert,
mit $g(z_0)\neq 0$ und
\begin{align*}
f(z) = \frac{g(z)}{(z-z_0)^m}.\fishhere
\end{align*} 
\end{cor}

\begin{propn}[Satz von Casorati-Weierstraß]
Sei $z_0$ eine wesentliche Singularität von $f$, dann liegt für jedes
 $\ep > 0$ das Bild $f(K_{\ep}(z_0))$ dicht in $\C$.\fishhere
\end{propn}

\begin{defn}
\label{defn:2.85}
Sei $\gamma$ geschlossene $C^1$-Kurve in $\C$ und
$z_0\in\C\setminus\im\gamma$, dann heißt
\begin{align*}
\nu(\gamma,z_0) = \frac{1}{2\pi i}\int\limits_\gamma \frac{1}{z-z_0}\dz,
\end{align*}
die \emph{Umlaufzahl} von $\gamma$ um $z_0$.\fishhere
\end{defn}

\begin{bsp}
\label{bsp:2.86}
\begin{enumerate}
  \item Sei $\gamma_N : [0,N]\to\C,\;t\mapsto e^{i2\pi t}$ und
  $\tilde{\gamma}_N\sim\gamma_N$,
  dann gilt
  \begin{align*}
  \int\limits_{\gamma_N} \frac{1}{z-z_0}\dz =
  \int\limits_{\tilde{\gamma}_N}\frac{1}{z-z_0}\dz = 2\pi iN,
  \end{align*}
also ist $\nu(\gamma_N,z_0) = N$.

\begin{pspicture}(-0.4,-2)(3.4,1.8)
%\psgrid
\rput(1.2,0.15){\psaxes[labels=none,ticks=none]{->}(0,0)(-1.4,-2)(2,1.6)}
\psbezier(2.3458889,0.05)(2.4,-1.63)(0.0,-1.55)(0.0,0.05)(0.0,1.65)(2.4,1.85)(2.4,0.05)(2.4,-1.75)(0.04,-1.47)(0.06588889,0.07)(0.09177778,1.61)(2.2917778,1.73)(2.3458889,0.05)
\psbezier(2.7341666,-0.7358889)(2.76,-2.05)(0.8,-1.9944444)(0.8,-0.75)(0.8,0.4944444)(2.8,0.65)(2.8,-0.75)(2.8,-2.15)(0.86,-1.9388888)(0.8458889,-0.75)(0.83177775,0.4388888)(2.7083333,0.5782222)(2.7341666,-0.7358889)
\psdots[linecolor=darkblue](1.6,-0.35)

\rput(1.6,-0.6){\color{gdarkgray}$z_0$}
\end{pspicture} 

Für $\abs{z_0}>1$ gilt $\nu(\gamma_N,z_0)=0$, denn $\gamma_N$ ist
Nullhomotop in $\C\setminus\{z_0\}$.
\item Einfach getwistete Kreislinie:

\begin{center}
\begin{pspicture}(0,-1.4432487)(6.081561,1.4232488)
\psbezier(4.6615615,-0.80324876)(3.2615614,-0.80324876)(2.923123,1.4032488)(1.4615614,1.3967513)(0.0,1.3902538)(0.06156141,-0.80324876)(1.4615614,-0.80324876)(2.8615613,-0.80324876)(3.2615614,1.3967513)(4.6615615,1.3967513)(6.0615616,1.3967513)(6.0615616,-0.80324876)(4.6615615,-0.80324876)
\psdots[linecolor=darkblue](3.1415615,-1.2032487)
\psdots[linecolor=darkblue](1.3415614,0.23675126)
\psdots[linecolor=darkblue](4.6415615,0.25675124)

\rput(1.6215614,0.106751256){\color{gdarkgray}$z_0$}
\rput(4.95,0.106751256){\color{gdarkgray}$z_1$}
\rput(3.4204676,-1.2932488){\color{gdarkgray}$z_2$}
\rput(5.7429676,1.3067513){\color{gdarkgray}$\gamma$}

\psline(5.6615615,0.63675123)(5.6215615,0.49675125)
\psline(5.6615615,0.63675123)(5.7815614,0.5367513)
\psline(0.8215614,1.1967512)(0.78156143,1.0767512)
\psline(0.8215614,1.1967512)(0.6815614,1.2167512)
\end{pspicture}
\end{center}

\begin{align*}
\nu(\gamma,z_j) = \begin{cases}
                  -1,& z_0,\\
                  1,& z_1,\\
                  0,& z_2.\bsphere
                  \end{cases}
\end{align*}
\end{enumerate}
\end{bsp}

\begin{bem}
\label{bem:2.87}
Die Umlaufzahl $\nu(\gamma,z_0)$ ist eine ganze Zahl (Ohne Beweis). Sind
$\gamma$ und $\tilde{\gamma}$ homotope Kurven in $\C\setminus\{z_0\}$, dann ist
auch $\nu(\gamma,z_0)=\nu(\tilde{\gamma},z_0)$.\maphere
\end{bem}

\begin{prop}
\label{prop:2.88}
Sei $f(z)=\sum\limits_{n=-\infty}^\infty a_n (z-z_0)^n$ in $K_{0,R}(z_0)$ und
$\gamma$ geschlossene $C^1$-Kurve in $K_{0,R}(z_0)$, dann gilt
\begin{align*}
\int\limits_\gamma f(z)\dz = 2\pi i a_{-1} \nu(\gamma,z_0).\fishhere
\end{align*}
\end{prop}
\begin{proof}
Sei $g(z) = f(z) - \frac{a_{-1}}{z-z_0} =
\sum\limits_{\atop{n=-\infty}{n\neq -1}}^\infty a_n (z-z_0)^n$,
dann hat $g$ folgende Stammfunktion in $K_{0,R}(z_0)$
\begin{align*}
G(z) = \sum\limits_{\atop{n=-\infty}{n\neq -1}}^\infty \frac{a_n}{n+1}
(z-z_0)^{n+1},
\end{align*}
und es gilt
\begin{align*}
\int\limits_\gamma g(z) \dz = G(\gamma(1)) - G(\gamma(0)) = 0,
\end{align*}
da $\gamma$ geschlossen und damit folgt,
\begin{align*}
\int\limits_\gamma f(z)\dz = \int\limits_\gamma g(z)\dz + \int\limits_\gamma
\frac{a_{-1}}{z-z_0}\dz = 0 + a_{-1}2\pi i \nu(\gamma,z_0).\qedhere
\end{align*}
\end{proof}
\begin{defn}
\label{defn:2.89}
Sei $f$ holomorph in $O$ mit isolierter Singularität in $z_0$ und $\gamma$
Kurve in $O$ mit $\nu(\gamma,z_0)\neq 0$, dann heißt
\begin{align*}
\Res(f,z_0) := a_{-1} = \left(\int\limits_\gamma f(z)\dz \right)\frac{1}{2\pi i
\nu(\gamma,z_0)} = \frac{1}{2\pi i}\int\limits_{\abs{z-z_0}=r}f(z)\dz,
\end{align*}
das \emph{Residuum} von $f$ in $z_0$.\fishhere
\end{defn}
\begin{bsp}
\label{bsp:2.90}
\begin{enumerate}
  \item Sei 
\begin{align*}
f(z) = \frac{\sin z}{z^2} = \sum\limits_{n=0}^\infty (-1)^n
\frac{z^{2n-1}}{(2n+1)!},
\end{align*}
dann ist $\Res(f,0) = 1$.
\item Sei $g(z) = \frac{\sin z}{z^4}$, dann ist $\Res(g,0) = -\frac{1}{6}$.
\item Sei $h(z) = e^{\frac{1}{z^2}} = \sum\limits_{n=0}^\infty \frac{1}{n!}
z^{-2n}$, dann ist $\Res(h,0) = 0$.\bsphere
\end{enumerate}
\end{bsp}

\begin{prop}[Residuensatz]
\label{prop:2.91}
Sei $f:O\setminus S\to \C$ holomorph und jedes $s\in S$ sei eine isolierte
Singularität von $f$. $\gamma$ sei $C^1$ nullhomotop in $O$ und $\im \gamma\cap
S = \varnothing$, dann gilt
\begin{align*}
\int\limits_{\gamma} f(z)\dz = \sum\limits_{z\in S} 2\pi
i\Res(f,z)\nu(\gamma,z).\fishhere
\end{align*}
\end{prop}
\begin{proof}
\begin{enumerate}[label=\arabic{*}.)]
  \item\label{proof:2.91:1} Sei $\phi$ eine $C^1$ Homotopie zwischen $\gamma$
  und einer konstanten Kurve.
  
  Die Funktion $g(z) = \frac{1}{z-z_0}$ ist holomorph in $\C\setminus\{z_0\}$
  und $\im \phi\subseteq \C\setminus\{z_0\}$, also ist $\gamma$ $C^1$
  nullhomotop und mit \ref{prop:2.25} folgt, dass
  \begin{align*}
  \int\limits_\gamma g(z)\dz = 0 \Rightarrow \nu(\gamma, z_0) = \frac{1}{2\pi
  i}\int\limits_\gamma g(z)\dz = 0.
  \end{align*}
\item\label{proof:2.91:2} Behauptung: die Summe im Residuensatz ist endlich.

\begin{enumerate}[label=(\alph{*})]
  \item Für alle hebbaren Singularitäten $z\in S$ gilt $\Res(f,z) = 0$.
  \item Angenommen, es gibt eine Folge $(z_n)$ von Polen bzw.
  wesentlichen Singularitäten von $f$ mit $\nu(\gamma,z)\neq 0$ und $z_n\neq
  z_m$ für $m\neq n$.
  
  Da $\nu(\gamma,z_n)\neq 0$, ist
  $z_n\in\im\phi$ und $\im\phi = \phi([0,1]\times[0,1])$ ist kompakt. Daher hat
  $z_n$ eine konvergente Teilfolge $z_{n_k}\to\zeta\in\im\phi$.
  
  Die $z_{n_k}$ sind nicht hebbare Singularitäten von $f$, also gilt
  \begin{align*}
  \forall k\in\N : \exists \zeta_k\in O : \abs{\zeta_k-z_{n_k}} < \frac{1}{k}
  \text{ und } \abs{f(\zeta_k)} > k,
  \end{align*}
denn ist $z_{n_k}$ ein Pol so, gilt $\abs{f(z)} =
\abs{\frac{g(z)}{(z-z_{n_k})^n}} \to\infty$ für $z\to z_{n_k}$ und
ist $z_{n_k}$ wesentlich, können wir den Satz von Casorati Weierstraß
anwenden und $f(K_{\frac{1}{k}}(z_{n_k}))$ liegt dicht in $\C$.

Nun gilt $\zeta_k\to\zeta$ und $f(\zeta_k)\to\infty$, also ist $f$ nicht
holomorph in $\zeta$.

Also ist $\zeta$ eine nicht hebbare Singularitätsstelle aber $z_{n_k}\to
\zeta$ ist ein Widerspruch, da $\zeta$ nach Voraussetzung eine isolierte
Singularitätsstelle von $f$ sein müsste.

Eine solche Folge $z_n$ kann somit nicht existieren und daher gilt,
\begin{align*}
\card \setdef{z\in S}{\Res(f,z)\cdot \nu(\gamma,z)\neq 0}<\infty.
\end{align*}
\end{enumerate}
\item\label{proof:2.91:3}
Aus \ref{proof:2.91:1} und \ref{proof:2.91:2} folgt, dass
$\setdef{z\in S}{\Res(f,z)\nu(\gamma,z)\neq 0} = \{z_1,\ldots,z_M\}$,  mit
$M\in\N$.
Setzt man
\begin{align*}
H(z_j,z) = \sum\limits_{n=-\infty}^{-1} a_n(z_j) (z-z_j)^n,\quad\text{für }
z\in \C\setminus\{z_j\},
\end{align*} 
und $g(z) = f(z)-\sum\limits_{j=1}^M H(z_j,z)$, dann
ist der Riemannsche Hebbarkeitssatz auf $g$ anwendbar und $g$ ist holomorph in
$\im\phi$.

Da $\gamma$ nullhomotop ist, folgt $\int\limits_\gamma g(z)\dz = 0$ und damit
ist
\begin{align*}
\int\limits_\gamma f(z)\dz
&= \int\limits_\gamma g(z)\dz + \sum\limits_{j=1}^M \int\limits_\gamma
H(z_j,z)\dz \\ &= 0 + \sum\limits_{j=1}^M 2\pi i\ a_{-1}\ \nu
(\gamma,z_j).\qedhere
\end{align*}
\end{enumerate}
\end{proof}

\begin{prop}[Residuenberechnung]
\label{prop:2.92}
\begin{enumerate}[label=\arabic{*}.)]
  \item $f(z) = \sum\limits_{n=-k}^\infty a_n (z-z_0)^n$ hat eine Polstelle der
  Ordnung $k$ bei $z_0$.
\begin{enumerate}[label=(\alph{*})]
  \item Ist $k=1$, dann ist
  \begin{align*}
  &(z-z_0)f(z) = \sum\limits_{n=-1}^\infty a_n(z-z_0)^{n+1}\\
  \Rightarrow\; &a_{-1}  = \lim\limits_{z\to z_0} (z-z_0)f(z).
  \end{align*}
  \item Ist $1<k\in\N$, dann ist
  \begin{align*}
  &\frac{\partial^{k-1}}{\partial z^{k-1}} (z-z_0)^k f(z)
  = \frac{\partial^{k-1}}{\partial z^{k-1}} \sum\limits_{n=-k}^\infty a_n
  (z-z_0)^{n+k}
  \\ &= \sum\limits_{n=-1}^\infty a_n \frac{(n+k)!}{(n+1)!} (z-z_0)^{n+1}\\
  &\overset{z\to z_0}{\rightarrow} a_{-1} (-1+k)(-1+(k-1))\cdots(1) =
  a_{-1}(k-1)!
  \end{align*}
\begin{align*}
  \Rightarrow\; a_{-1} &= \lim\limits_{z\to z_0}
  \frac{1}{(k-1)!}\frac{\partial^{k-1}}{\partial z^{k-1}} \left((z-z_0)^k
 f(z) \right) \\ &= \frac{1}{(k-1)!}\frac{\partial^{k-1}}{\partial z^{k-1}}
 \left( (z-z_0)^k f(z) \right)\big|_{z=z_0}.
  \end{align*}
\end{enumerate}
\item Falls $f=\frac{g}{h}$ mit $g,h$ holomorph in $z_0$ und $h(z_0) = 0,\;
h'(z_0)\neq 0$ und $g(z_0)\neq 0$, dann ist
\begin{align*}
\ph(z) = \begin{cases}
         \frac{h(z)-h(z_0)}{z-z_0}, & \text{ für } z\neq z_0,\\
         h'(z_0)\neq 0,&\text{für } z=z_0,
         \end{cases}
\end{align*}
holomorph in $z_0$ und damit hat $f(z) = \frac{1}{z-z_0}\frac{g(z)}{\ph(z)}$
einen Pol 1. Ordnung in $z_0$ und es gilt $\Res (f,z_0) =
\frac{g(z_0)}{h'(z_0)}$.
\end{enumerate}
\end{prop} 

\begin{bsp}
\label{bsp:2.93}
\begin{enumerate}[label=\arabic{*}.)]
  \item $f(z) = \frac{e^{iz}}{1+z^2}$ dann ist
  \begin{align*}
  &\Res(f,i) = \frac{e^{ii}}{2i} =
  \frac{e^{-1}}{2i},\\
  &\Res(f,-i) = \frac{e^{-ii}}{-2i} = -\frac{e}{2i}.
  \end{align*}
\begin{center}
\begin{pspicture}(-2.7,-2.7)(2.7,2.7)
 \psaxes[labels=none,ticks=none]{->}%
 (0,0)(-2.5,-2.5)(2.5,2.5)[\color{gdarkgray}$\Re$,-90][\color{gdarkgray}$\Im$,0]
 
 \psarc(0,0.05){2}{0.0}{180.0}
 \psline(-2,0.05)(2,0.05)
 
 \psarc(0,-0.05){2}{180}{360}
 \psline(-2,-0.05)(2,-0.05)
 
 \psyTick(1){\color{gdarkgray}1}
 \psyTick(-1){\color{gdarkgray}-1}
 
 \psdots[linecolor=darkblue](0,1)(0,-1)
 
 \rput(-2.25,-0.2){\color{gdarkgray}$-R$}
 \rput(2.18,-0.2){\color{gdarkgray}$R$}
%  
%  \psline[linestyle=dotted](3.5,0)(3.5,1.5)
%  \psline[linestyle=dotted](0,1.5)(3.5,1.5)
%  \psline[linecolor=darkblue,arrows=-*](0,0)(3.5,1.5)
%  \psarc[arrows=->,linestyle=dashed](0,0){1.1}{0}{21.6}
%  
%  \rput(1.3,0.2){\color{gdarkgray}$\ph$}
%  \rput[b]{29.7}(1.8,0.9){\color{gdarkgray}$\abs{z}$}
%  
\end{pspicture}
\end{center}

\begin{align*}
\int\limits_{\gamma_1} f(z)\dz &= 2\pi i
\left(\Res(f,i)\underbrace{\nu(\gamma_1,i)}_{=1} +
\Res(f,-i)\underbrace{\nu(\gamma_1,-i)}_{=0}\right) \\ &= 2\pi i\Res(f,i) =
\frac{\pi}{e}.
\end{align*}
\begin{align*}
\int\limits_{\gamma_2}f(z)\dz = \pi e.
\end{align*}
  \item Mit Hilfe der Funktionentheorie lassen sich auch elegant reelle 
 uneigentliche Integrale berechnen.
 
 Das Integral 
 $\int\limits_{-\infty}^\infty
 \frac{\cos x}{(1+x^2)^2}\dx$ konvergiert. Sei nun 
 \begin{align*}f(z) =
 \frac{e^{iz}}{(1+z^2)^2} \Rightarrow \Re(f(x+i\cdot 0)) = \frac{\cos
 x}{(1+x^2)^2}.
 \end{align*}
  
  Tipp: Die Funktion $e^{iz}$ ist in $\setdef{z\in\C}{\Im z \ge 0}$ beschränkt,
  deshalb wählt man $\gamma_1$ in der oberen Halbebenen.
  
  $z_0 = i$ ist Polstelle 2. Ordnung, für das Residuum gilt also:
  \begin{align*}
  \Res (f,i) &= \frac{\diffd}{\dz}(z-i)^2\frac{e^{iz}}{(1+z^2)^2}\Huge|_{z=i}
  = \frac{\diffd}{\dz}\frac{e^{iz}}{(z+i)^2}\big|_{z=i}
  \\ &= \frac{ie^{iz}(z+i)^2 - 2e^{iz}(z+i)}{(z+i)^4}\big|_{z=i}
  = \frac{e^{iz}(i(z+i)-2)}{(z+i)^3}\big|_{z=i}\\ &
  = -\frac{4e^{-1}}{-8i} = \frac{1}{2ie}.
  \end{align*}
  Das Integral über $\gamma$ hat also den Wert
  \begin{align*}
  \int\limits_\gamma f(z)\dz = 2\pi i \Res(f,i)\nu(\gamma,i),
  = 2\pi i \frac{1}{2ie} = \frac{\pi}{e}.
  \end{align*}

Den Wert des Kurvenstücks  in der oberen Halbebenen ($\Im z > 0$) können wir
für $R\to\infty$ wie folgt abschätzen,
\begin{align*}
\abs{f(z)} = \frac{\abs{e^{iz}}}{\abs{1+z^2}^2} = \frac{\abs{e^{-\Im
z}}}{\abs{1+z^2}^2} \le \frac{1}{\abs{1+z^2}^2} \le \frac{1}{(\abs{z}^2-1)^2},
\end{align*}
also ist
\begin{align*}
\abs{\int\limits_\gamma f(z)\dz } \le L(\gamma) \max\limits_{\gamma} \abs{f}
\le \pi R \frac{1}{(R^2-1)^2} \to 0.
\end{align*}
Der Grenzwert des Integral ist also
\begin{align*}
\pi e^{-1} = \int\limits_{-R}^R f(x)\dx +
\underbrace{\int\limits_{\atop{\abs{z}=R}{\Im z > 0}} f(z)\dz}_{\to 0} \to
\int\limits_{-\infty}^\infty \frac{e^{ix}}{(1+x^2)^2}\dx.
\end{align*}
Bildet man den Realteil, erhält man das reelle Integral,
\begin{align*}
\pi e^{-1} = \int\limits_{-\infty}^\infty \frac{\cos x}{(1+x^2)^2}\dx.\bsphere
\end{align*}
\end{enumerate}
\end{bsp}

\begin{defn}
\label{defn:2.94}
Sei $G\subseteq \C$ Gebiet. Eine geschlossene $C^1$-Kurve $\gamma$
\emph{berandet $G$}, falls
\begin{align*}
\nu(\gamma,z) = \begin{cases}
                1, & z\in G,\\
                0, & z\in \C\setminus\overline{G}.\fishhere
                \end{cases}
\end{align*}
\end{defn}

\begin{bsp}
\label{bsp:2.95}
$G=K_1(0), \gamma_n(t) = e^{i2\pi nt}$ für $0\le t\le 1$ und $n\in \Z$.

$\gamma_1$ berandet $G$.

$\gamma_n$ für $n\neq 1$ berandet $G$ nicht, auch nicht für $n=-1$.\bsphere
\end{bsp}

\begin{defn}
\label{defn:2.96}
Ist $f$ in $O$ bis auf Pole holomorph, so heißt $f$
\emph{meromorph}.\fishhere
\end{defn}

\begin{prop}[Null- und Polstellen Zählen I]
\label{prop:2.97}
Sei $f$ meromorph in $O$ und $G\subseteq O$ Gebiet, $f$ nicht
konstant Null auf $G$, $\gamma$ eine $C^1$ Kurve in $O$, die $G$
berandet und $\gamma$ geht durch keine Null- oder Polstellen, dann gilt
\begin{align*}
\frac{1}{2\pi i}\int\limits_{\gamma} \frac{f'(z)}{f(z)}\dz = N_G-P_G,
\end{align*}
wobei $N_G$ die Anzahl der Nullstellen in $G$ und
$P_G$ die Anzahl der Polstellen in $G$ jeweils mit Vielfachheit gezählt
ist.\fishhere
\end{prop}
\begin{proof}
Sei $S$ die Menge der Null- und Polstellen von $f$, dann besteht $S$ nur aus
isolierten Punkten (vgl. \ref{prop:2.56}).

$\frac{f'}{f}$ ist holomorph in $O\setminus S$ und daraus folgt mit dem
Residuensatz, dass
\begin{align*}
\frac{1}{2\pi}\int\limits_\gamma \frac{f'(z)}{f(z)}\dz = \sum\limits_{z\in S}
\Res\left(\frac{f'}{f},z\right)\nu(\gamma,z) = \sum\limits_{z\in S}
\Res\left(\frac{f'}{f},z\right).
\end{align*}

Behauptung: $\Res\left(\frac{f'}{f},z\right) = \begin{cases}
                                               k,&\text{falls }z_0\text{ eine }
                                               k\text{-fache Nullstelle},\\
                                               -k,&\text{falls }z_0\text{ eine
                                               }k\text{-fache Polstelle}.
                                               \end{cases}$

Sei nun $f(z) = \sum\limits_{n=k}^\infty a_n(z-z_0)^n = (z-z_0)^k g(z)$. Ist
$k\in\N$, dann ist $z_0$ eine Nullstelle der Ordnung $k$, ist $-k\in\N$, dann
ist $z_0$ eine Polstelle der Ordnung $k$.

Insbesondere ist $g$ holomorph in $z_0$ und $g(z_0)\neq 0$, also gilt
\begin{align*}
\frac{f'(z)}{f(z)} &= \frac{k(z-z_0)^{k-1}g(z)+(z-z_0)^kg'(z)}{(z-z_0)^kg(z)}
 \\ &=\frac{kg(z)}{(z-z_0)g(z)} + \frac{(z-z_0)g'(z)}{(z-z_0)g(z)}
 = \frac{k}{z-z_0} + \frac{g'(z)}{g(z)}.
\end{align*}
Also ist $z_0$ ein Pol 1. Ordnung von $\frac{f'}{f}$ und es gilt
\begin{align*}
\Res\left(\frac{f'}{f},z_0\right)&=\lim\limits_{z\to z_0}
\left((z-z_0)\frac{f'(z)}{f(z)}\right) \\&= k = \begin{cases}
                                             \text{Ordnung der Nullstelle,
                                             falls $z_0$ Nullstelle},\\
                                             \text{-Ordnung des Pols, falls
                                             $z_0$ Polstelle}.\qedhere
                                             \end{cases}
\end{align*}
\end{proof}

\begin{bsp}
\label{bsp:2.98}
\begin{align*}
&f(z) = \frac{(z-2)(z-3)}{(z-1)^2},\\
&\frac{1}{2\pi i}\int\limits_{\gamma_j} \frac{f'(z)}{f(z)}\dz = \begin{cases}
                                                               -2,& j=1\\
                                                               1, & j=2,3\\
                                                               0, & j=4.\bsphere
                                                               \end{cases}
\end{align*}
\end{bsp}

\begin{prop}[Null- und Polstellen Zählen II]
\label{prop:2.99}
Seien die Voraussetzungen wie in \ref{prop:2.97}, dann gilt $N_G-P_G =
\nu(f\circ\gamma,0)$.\fishhere
\end{prop}
\begin{proof}
\begin{align*}
2\pi i(N_G-P_G) &= \int\limits_{\gamma} \frac{f'(z)}{f(z)}\dz =
\int\limits_{t_0}^{t_1}
\frac{f'\circ\gamma(t)}{f\circ\gamma(t)}\dot{\gamma}(t)\dt
\\ &= \int\limits_{t_0}^{t_1}
\frac{1}{f\circ\gamma(t)}\frac{\diffd}{\dt}(f\circ\gamma)(t) \dt
= \int\limits_{f\circ\gamma} \frac{1}{z}\dz = \nu(f\circ\gamma,0)2\pi i.\qedhere 
\end{align*}
\end{proof}

\begin{prop}[Satz von Rouché]
\label{prop:2.100}
Seien $f,g$ holomorph in $O$ und $G\subseteq O$ Gebiet berandet von einer
$C^1$-Kurve $\gamma$ in $O$, so dass
\begin{align*}
\abs{g(z)}<\abs{f(z)}, \text{ für } z\in\im \gamma,
\end{align*}
dann haben $f$ und $f+g$ gleich viele Nullstellen in $G$.\fishhere
\end{prop}
\begin{proof}
Aus \ref{prop:2.99} folgt, dass
\begin{align*}N_G(f) &= \nu(f\circ\gamma,0) =
\frac{1}{2\pi i}\int\limits_{f\circ\gamma} \frac{1}{z}\dz\\
N_G(f+g) &= \nu((f+g)\circ\gamma,0) = \frac{1}{2\pi
i}\int\limits_{(f+g)\circ\gamma} \frac{1}{z}\dz.
\end{align*}
Zeige, dass $f\circ\gamma\sim (f+g)\circ \gamma$ in $\C\setminus\{0\}$.
\begin{align*}
\phi(t,s) &:= (f\circ\gamma)(t) + s(g\circ\gamma)(t)\\
\phi(t,0) &= (f\circ\gamma)(t),\\
\phi(t,1) &= ((f+g)\circ\gamma)(t),\\
\abs{\phi(t,s)} &= \abs{f\circ\gamma(t) + s g\circ\gamma(t)} \\ &\ge
\abs{f\circ\gamma(t)} - s\abs{g\circ\gamma(t)} \ge
\abs{f\circ\gamma(t)}-\abs{g\circ\gamma(t)} > 0.
\end{align*}
Also ist $f\circ\gamma \sim (f+g)\circ\gamma$ in $\C\setminus\{0\}$.\qedhere
\end{proof}

\begin{prop}[Fundamentalsatz der Algebra]
\label{prop:2.101}
Sei $p(z) = z^n + \sum\limits_{j=0}^{n-1} a_j z^j$ und $a_j\in\C$, dann hat $p$
in $\C$ genau $n$-Nullstellen mit Vielfachheit gezählt und alle liegen im
$\overline{K_R(0)}$ mit $R=\max\left\{
1,\sum\limits_{j=0}^{n-1}\abs{a_j}\right\}$.
\end{prop}
\begin{proof}
\begin{enumerate}[label=\arabic{*}.)]
\item Für $\abs{z} > R$ gilt
\begin{align*}
\abs{\sum\limits_{j=0}^{n-1} a_j z^j} \le \sum\limits_{j=0}^{n-1} \abs{a_j}
\abs{z}^j \le \abs{z}^{n-1}\underbrace{\sum\limits_{j=0}^{n-1} \abs{a_j}}_{\le
R < \abs{z}} < \abs{z}^n,
\end{align*}
also ist $p(z)\neq 0$ für $z\notin\overline{K_R(0)}$.
\item
Sei $\ep > 0$, dann hat $f(z) = z^n$ $n$ Nullstellen (mit
Vielfachheit gezählt) in $K_{R+\ep}(0)$. Sei
\begin{align*}
g(z) =\sum\limits_{j=0}^{n-1} a_j z^j,
\end{align*}
dann gilt für $\abs{z} = R+\ep$, dass $\abs{g(z)}<\abs{f(z)}$, also haben $f(z)$
und $p(z) = f(z)+g(z)$ gleich viele Nullsten in $K_{R+\ep}(0)$, also
$n$.\qedhere
\end{enumerate}
\end{proof}

\begin{bem}
\label{bem:2.102}
Holomorphe Funktionen sind allgemeiner als man denken könnte. Der Riemannsche
Abbildungssatz besagt, dass für jedes einfach zusammenhängende Gebiet
$G\varsubsetneq  \C$ eine bijektive holomorphe Abbildung $f:G\to K_1(0)$
existiert.\maphere
\end{bem}