\section{Einführung in Integrations und Maßtheorie}

\subsection{Falsche Erwartungen}

Wir wissen bereits, dass $C^1([a,b]\to\R)$ bezüglich der Supremumsnorm
vollständig ist, es existiert jedoch kein Skalarprodukt, das die Supremumsnorm
induziert, also für das gilt $\norm{f} = \sqrt{\lin{f,f}}$. Für die
durch ein Skalarprodukt induzierte $L_2$-Norm
\begin{align*}
\norm{f}_2 := \sqrt{\lin{f,f}} = \sqrt{\int\limits_a^b \left(\abs{f}^2\right)\dx},
\end{align*}
ist $C^1([a,b]\to\R)$ hingegen nicht mehr vollständig. Nun sind sowohl
Orthogonalität als auch Vollständigkeit wichtige Konzepte in der Analysis, auf
die wir auch in Funktionenräumen nicht verzichten wollen. Die Intuition rät,
den $C^1([a,b]\to\R)$ bezüglich der $L_2$-Norm ``zu
vervollständigen'', wie wir $\Q$ zu $\R$ bezüglich der euklidischen Norm
vervollständigt haben. Dies ist mit Hilfe des \emph{Lebesgue-Integrals} möglich.

Bis dahin müssen wir jedoch noch etwas arbeiten. Zuerst wollen wir uns damit
beschäftigen Mengen zu ``messen''. Wir kennen für einen Quader
$Q=X_{j=1}^n [a_j,b_j]$ im $\R^n$ das Volumen $V(Q) = \prod_{j=1}^n
\abs{b_j-a_j}$.\\
Zunächst wollen wir die Frage klären, ob der
Volumenbegriff für beliebige Mengen $M\subseteq\R^n$ Sinn macht.\\
Für einen allgemeinen Volumenbegriff sollte gelten, dass eine Translation,
Spiegelung oder Rotation der Menge das Volumen nicht ändert. Außerdem sollte das
Volumen des Einheitsquaders stets 1 sein. Und betrachtet man die Vereinigung zweier
disjunkter Mengen $M_1$ und $M_2$, dann sollte das Volumen der Vereinigung der
Summe der Einzelvolumina entsprechen.

\begin{defn}[Erster Versuch]
\label{defn:3.1}
Seien $A,B\subseteq\R^n$. Eine Abbildung $\mu: \PP(\R^n)\to[0,\infty]$ heißt
\emph{Inhalt}, falls sie den folgenden Eigenschaften genügt.
\begin{enumerate}[label=(\roman{*})]
  \item\label{defn:3.1:1} $\mu(A\dot{\cup} B) = \mu(A)+\mu(B)$.
  \item Sei $\beta: \R^n\to\R^n$ eine Translation, Spiegelung oder Rotation,
  dann ist $\mu(\beta(A)) = \mu(A)$.
  \item $\mu((0,1]^n)=1$.
  \par
$\mu$ heißt \emph{Maß}, falls statt \ref{defn:3.1:1} gilt
\item $\mu(\dot{\bigcup}_{j\in\N} A_j) = \sum_{j=1}^\infty \mu(A_j)$.\fishhere
\end{enumerate}
\end{defn}

\begin{bsp}
\label{bsp:3.2}
Gegenbeispiel (Vitali 1905): Es gibt kein Maß $\mu: \PP(\R)\to[0,\infty]$ auf
$\R$.

Zerlege $(0,1]$ in abzählbar viele gleichgroße disjunkte Mengen. Sei $\sim$
eine Äquivalenzrelation auf $(0,1]$ und
\begin{align*}
x\sim y\Leftrightarrow x-y\in\Q.
\end{align*}
Sei $(q_j)_{j\in\N}$ eine Abzählung von $\Q\cap(0,1]$.

Sei $A_0\subseteq(0,1]$ so definiert, dass $A_0$ aus jeder
Äquivalenzklasse genau ein Element enthält\footnote{Dass das funktioniert,
garantiert das Auswahlaxiom}.

$A_j:= (A_0+q_j)\mod 1$.
\begin{enumerate}[label=(\alph{*})]
  \item Es gilt $\bigcup_{j=1}^\infty A_j = (0,1]$.
  
  ``$\subseteq$'' gilt per
  definitionem. Sei nun $x\in(0,1]$, dann gibt es ein $y\in A_0: x\in[y]$, also
  ist $x-y\in\Q\cap(-1,1)$.
  
  Falls $x=y$, dann ist $x\in A_0\subseteq A_0 + 1\mod 1$.
  
  Falls $x-y\in\Q\cap (0,1)$, existiert ein $j\in\N : x-y=q_j$, also ist $x\in
  (A_0 + q_j)\mod 1$.
  
  Falls $x-y\in\Q\cap(-1,0)$, existiert ein $j\in\N : x-y + 1 = q_j$, also ist
  $x\in (A_0+q_j)\mod 1$.
  
  Also ist $(0,1]\subseteq \bigcup_{j=1}^n A_j$.
  \item Es ist $A_j\cap A_k = \varnothing$, falls $j\neq k$.
  
Sei $x\in A_j\cap A_k$ für $j\neq k$, dann gilt
\begin{align*}
x &= (y_1+q_j)\mod 1, && y_1\in A_0,\\
x &= (y_2+q_k)\mod 1, && y_2\in A_0,\\
\Rightarrow &= y_1-y_2\in\Q,
\end{align*}  
also ist $[y_1]=[y_2]$ und daher auch $y_1=y_2$, da $A_0$ genau einen Vertreter
jeder Äquivalenzklasse enthält und damit ist $q_j=q_k\mod 1$ und $A_j=A_k$, ein
Widerspruch, also ist $A_j\cap A_k = \varnothing$.
\item Angenommen $\mu$ sei ein Maß, dann gilt für jedes $j\in\N$,
\begin{align*}
\mu(A_j) &= \mu\left((A_0 + q_j)\cap (0,1]\right) + \mu\left((A_0+q_j)\cap (1,2]
-1 \right) \\
&= \mu\left((A_0 + q_j)\cap (0,1]\right) + \mu\left((A_0+q_j)\cap (1,2]
\right) \\
&= \mu\left(((A_0 + q_j)\cap (0,1]) \cup ((A_0+q_j)\cap (1,2])
\right) \\
&= \mu\left((A_0+q_j)\cap (0,2]\right) = \mu(A_0 + q_j) = \mu(A_0).
\end{align*}
Aber es ist $\mu((0,1]) = \sum\limits_{j=1}^\infty \mu(A_j) =
\sum\limits_{j=1}^\infty \mu(A_0)\neq 1$, ein Widerspruch.\bsphere
\end{enumerate}
\end{bsp}

\begin{bem}
\label{bem:3.3}
\begin{enumerate}[label=\arabic{*}.)]
  \item Hausdorff (1914): Man kann keinen Inhalt auf dem $\R^n$ für $n\ge3$
  definieren.
  \item Banach-Tarski (1924): Sei $n\ge 3$, dann existieren
  $C_1,\ldots,C_k\subseteq\R^n$ und Bewegungen $\beta_1,\ldots,\beta_k$, sodass
  $K_1(0) = \dot{\bigcup}_{j=1}^k C_j$ und
  $\dot{\bigcup}_{j=1}^k \beta_j(C_j) = K_1(0) \cup K_1(3)$.
  \item Banach (1923): Auf $\R$ und $\R^2$ existieren verschiedene
  Inhalte.\maphere
\end{enumerate}
\end{bem}

\subsection{Messbare Mengen}

\begin{defn}
\label{defn:3.4}
Sei $\Omega$ Menge.
\begin{enumerate}[label=\arabic{*}.)]
  \item $h\subseteq \PP(\Omega)$ heißt \emph{Halbring}, falls
  \begin{enumerate}[label=(\roman{*})]
    \item $\varnothing\in h$,
    \item $A,B\in h \Rightarrow A\cap B\in h$,
    \item $A,B\in h,\;A\subseteq B \Rightarrow \exists k\in \N \exists
    C_1,\ldots,C_k\in h : B\setminus A = \dot{\bigcup}_{i=1}^k C_i$.
  \end{enumerate}
  \item $\Sigma\subseteq\PP(\Omega)$ heißt \emph{$\sigma$-Algebra}, falls
  \begin{enumerate}[label=(\roman{*})]
    \item $\varnothing\in \Sigma$,
    \item $A\in\Sigma\Rightarrow A^c\in\Sigma$,
    \item $A_j\in\Sigma \Rightarrow \bigcup_{j\in\N} A_j \in \Sigma$.
  \end{enumerate}
  $(\Omega,\Sigma)$ heißt \emph{messbarer Raum}, $\Sigma$ \emph{Menge der
messbaren Mengen}.
\item Zu $M\subseteq\PP(\Omega)$ ist $\sigma(M) := \bigcap_{\Sigma\supseteq M}
\Sigma$, eine $\sigma$-Algebra, denn Schnitte von $\sigma$-Algebren sind
$\sigma$-Algebren. Sie heißt die \emph{von $M$ erzeugte $\sigma$-Algebra}.
\item Sei $X$ Menge, dann heißt $\OO_X\subseteq\PP(X)$ \emph{Topologie} auf
$X$, falls es den folgenden Eigenschaften genügt:
\begin{enumerate}[label=(\roman{*})]
  \item $\varnothing,X\in\OO_X$,
  \item $A,B\in\OO_X\Rightarrow A\cap B\in\OO_X$,
  \item $\TT\subseteq \OO_X \Rightarrow \bigcup_{O\in\TT} O \in \OO_X$.
\end{enumerate}
$\BB(X) := \sigma(\OO_X)$ heißt \emph{Borel $\sigma$-Algebra}.\fishhere
\end{enumerate}
\end{defn}

\begin{cor}
\label{prop:3.5}
\begin{enumerate}[label=\arabic{*}.)]
  \item Sei $\Sigma$ eine $\sigma$-Algebra und für $j\in\N$ sei $A_j\in\Sigma$,
  dann ist,
  \begin{align*}
  \bigcap_{j\in\N} A_j \in \Sigma.
  \end{align*}
  \item Ist $\OO_X$ Topologie auf $X$, $\AA_X=\setdef{O^c}{O\in\OO_X}$ das
  System aller abgeschlossener Mengen in $X$, so gilt $\sigma(\OO_X) =
  \sigma(\AA_X)$.\fishhere
\end{enumerate}
\end{cor}
\begin{proof}
\begin{enumerate}[label=\arabic{*}.)]
  \item Wir wissen, dass Komplemente und abzählbare Vereinigungen in $\Sigma$
  sind,
  \begin{align*}
  A_j\in\Sigma \Rightarrow A_j^c \in\Sigma \Rightarrow \bigcup_{j\in\N}
  A_j^c\in\Sigma \Rightarrow \bigcap_{j\in\N} A_j = \left(\bigcup_{j\in\N}
  A_j^c\right)^c \in \Sigma.
  \end{align*}
\item Ist $M\subseteq \Sigma$, dann ist auch $\sigma(M)\subseteq\Sigma$, also
gilt
\begin{align*}
A\in\AA_X &\Rightarrow \exists O\in\OO_X : A=X\setminus O \Rightarrow
A\in\sigma(\OO_X) \\ &\Rightarrow \AA_X\subseteq \sigma(\OO_X) \Rightarrow
\sigma(\AA_X)\subseteq \sigma(\OO_X).
\end{align*}
Analog folgt $\sigma(\OO_X)\subseteq\sigma(\AA_X)$.\qedhere
\end{enumerate}
\end{proof}

\begin{prop}
\label{prop:3.6}
\begin{enumerate}[label=\arabic{*}.)]
  \item Die Borel $\sigma$-Algebra $\BB(\R)$ wird von jeder der folgenden Mengen
erzeugt,
\begin{align*}
&M_1 = \setdef{A\subseteq\R}{A\in\AA_\R},\\
&M_2 = \setdef{O\subseteq\R}{O\in\OO_\R},\\
&M_3 = \setdef{(a,b)\subseteq\R}{a,b\in\R},\\
&M_4 = \setdef{(a,b]}{a,b\in\R},\\
&M_5 = \setdef{[a,b)}{a,b\in\R},\\
&M_6 = \setdef{[a,b]}{a,b\in\R},\\
&M_7 = \setdef{[a,\infty)}{a\in\R},\\
&M_8 = \setdef{(a,\infty)}{a\in\R},\\
&M_9 = \setdef{(-\infty,a]}{a\in\R},\\
&M_{10} = \setdef{(-\infty,a)}{a\in\R},\\
&M_{11} = \setdef{(a,b)}{a\in\Q}.
\end{align*}
Entsprechendes gilt für $\BB(\R^n)$. Insbesondere wird $\BB(\R^n)$ von
folgendem Halbring erzeugt
\begin{align*}
h = \setdef{X_{i=1}^\infty (a_i,b_i]}{a_i,b_i\in\R}.
\end{align*}
\item
$\BB(\R)$ enthält alle endlichen und abzählbaren Teilmengen von $\R$.\fishhere
\end{enumerate}
\end{prop}

\begin{proof}
\begin{enumerate}[label=\arabic{*}.)]
  \item $(a,b] = \underbrace{(a,\infty)}_{\in\OO_\R}\cap
  \underbrace{(b,\infty)^c}_{\in\AA_\R} \in\BB(\R)$, also ist
  $M_4\subseteq\BB(\R)$ und damit ist $\sigma(M_4)\subseteq \BB(\R)$.
  \item Sei $O\subseteq\R$ offen, dann ist $O$ abzählbare Vereinigung offener
  Intervalle, also
  \begin{align*}
  &O = \bigcup_{j=1}^\infty (a_j,b_j),\\
  &(a_j,b_j) = \bigcup_{k=1}^\infty \left(a_j,b_j -
  \frac{1}{k}\right]\in\sigma(M_4),
  \end{align*}
und damit ist $O\in\sigma(M_4)$.\qedhere
\end{enumerate}
\end{proof}

\begin{prop}
\label{prop:3.7}
Sei $M\subseteq\PP(\Omega')$ und $f:\Omega\to\Omega'$, dann ist
\begin{align*}
f^{-1}(\sigma(M)) = \setdef{f^{-1}(A)}{A\in\sigma(M)},
\end{align*}
eine $\sigma$-Algebra und wird von $f^{-1}(M)$ erzeugt.\fishhere
\end{prop}
\begin{proof}
\begin{enumerate}[label=(\roman{*})]
  \item\label{proof:3.7:1} $\varnothing = f^{-1}(\varnothing) \in
  f^{-1}(\sigma(M))$,
  \item\label{proof:3.7:2} $f^{-1}(A^c) = f^{-1}(\Omega'\setminus A) = \Omega
  \setminus f^{-1}(A)$. Sei also $B\in f^{-1}(\sigma(M))$, dann existiert ein $A\in
  \sigma(M) : B = f^{-1}(A)$. Damit gilt $B^c = f^{-1}(A)^c =
  f^{-1}(A^c) \in f^{-1}(\sigma(M))$.
  \item\label{proof:3.7:3} Die ``Urbildabbildung'' ist inklusionserhaltend, also
  gilt
  \begin{align*}
  f^{-1}\left(\bigcup_{j\in\N} A_j\right)  = \bigcup_{j\in\N} f^{-1}(A_j),
  \end{align*}
  ist also $B_j\in f^{-1}(\sigma(M))$, dann folgt $\bigcup_{j\in\N} B_j \in
  f^{-1}(\sigma(M))$.
  \par
  aus \ref{proof:3.7:1} bis \ref{proof:3.7:3} folgt, $f^{-1}(\sigma(M))$ ist
$\sigma$-Algebra.
\item $f^{-1}(M) \subseteq f^{-1}(\sigma(M))$ also ist auch
$\sigma(f^{-1}(M))\subseteq f^{-1}(\sigma(M))$.

Betrachte $\Sigma = \setdef{A\in\PP(\Omega)}{f^{-1}(A)\in\sigma(f^{-1}(M))}$,
also ist $\Sigma$ eine $\sigma$-Algebra auf $\Omega$ (Beweis analog zu oben).
 
Offensichtlich ist $M\subseteq\Sigma$, also ist auch $\sigma(M)\subseteq
\Sigma$ und damit ist
\begin{align*}
f^{-1}(\sigma(M)) \subseteq f^{-1}(\Sigma) = \sigma(f^{-1}(M)).
\end{align*}
Also ist $f^{-1}(\sigma(M)) = \sigma(f^{-1}(M))$ die von $f^{-1}(M)$ erzeugte
$\sigma$-Algebra.\qedhere
\end{enumerate}
\end{proof}

\subsection{Maße}

Sei nun stets $h$ ein Halbring und $\Sigma$ eine $\sigma$-Algebra.

\begin{defn}
\label{defn:3.8}
$\mu: h\to[0,\infty]$ heißt \emph{Prämaß}, $\mu: \Sigma\to[0,\infty]$ heißt
\emph{Maß}, falls
\begin{enumerate}[label=(\roman{*})]
  \item $\mu(\varnothing) = 0$, (Definitheit)
  \item $\mu(A)\ge 0$, (Positivität)
  \item $\mu\left(\dot{\bigcup}_{j\in\N} A_j \right) = \sum\limits_{j=1}^\infty
  \mu(A_j)$, ($\sigma$-Additivität)
  
  falls $\dot{\bigcup}_{j\in\N} A_j\in h$ bei Prämaß. 
\end{enumerate}
$(\Omega,\Sigma,\mu)$ heißt \emph{Maßraum}.

Falls $\Omega=\bigcup_{n\in\N} A_n$ mit $\mu(A_n)<\infty$, dann heißt $\mu$
\emph{$\sigma$-endlich}.

Ist $\mu(\Omega) <\infty$, dann heißt $\mu$ \emph{endliches Maß}.

Ist $\mu(\Omega) = 1$, dann heißt $\mu$ \emph{Wahrscheinlichkeitsmaß}.\fishhere
\end{defn}

\begin{prop}[Forsetzungssatz]
\label{prop:3.9}
Ein Prämaß $\mu_0$ auf $h$ besitzt eine Fortsetzung $\mu$ zu einem Maß auf
$\sigma(h)$. Ist $\mu_0$ $\sigma$-endlich, dann ist $\mu$ eindeutig.\fishhere
\end{prop}
\begin{proof}
Elstrodt, Kapitel III, $\mathsection\mathsection$ 4.5.\qedhere
\end{proof}

\begin{cor}
\label{prop:3.10}
Es gibt genau ein Maß $\mu^{(n)}$ auf der Borel $\sigma$-Algebra $\BB(\R^n)$ mit
\begin{align*}
\mu(X_{j=1}^n (a_j,b_j]) = \prod_{j=1}^n \abs{b_j-a_j}.
\end{align*}
$\mu^{n}$ heißt \emph{Lebesgue-Borel-Maß} und ist
translationsinvariant.\fishhere
\end{cor}
\begin{proof}
Durch $\mu$ wird ein Prämaß auf dem Halbring
\begin{align*}
h = \setdef{X_{j=1}^n (a_j,b_j]}{a_j,b_j\in\R},
\end{align*}
definiert. $\mu$ ist $\sigma$-endlich, denn $\R^n = \bigcup_{j=1}^\infty
W(z_j)$, mit $(z_j)$ einer Abzählung von $\Z^n$ und $W(z_j)$ ein halboffener
Würfel mit Mittelpunkt $z_j$ und Seitenlänge $1$.

Dann ist $\mu(W(z_j)) = 1$ und die Fortsetzung eindeutig. $\mu$ ist auf $h$
translationsinvariant, also ist auch die Fortsetzung
translationsinvariant.\qedhere
\end{proof}

\begin{prop}[Weitere Eigenschaften]
\label{prop:3.11}
Sei $\mu$ Maß, $A_j\in\Sigma$, dann gilt
\begin{enumerate}[label=(\roman{*})]
  \item $A_1\subseteq A_2\Rightarrow \mu(A_1)\le \mu(A_2)$. (Monotonie)
  \item $A_1\subseteq A_2\subseteq A_3\subseteq \ldots,\;
  A:=\left(\bigcup_{j=1}^\infty A_j\right)\in\Sigma$, dann ist
  
  $\mu(A) = \lim\limits_{j\to\infty} \mu(A_j)$.
  \item $A_1\supseteq A_2\supseteq A_3\supseteq \ldots, A:=\left(
  \bigcap_{j=1}^\infty A_j \right)\in\Sigma$, dann ist
  
  $\mu(A) = \lim\limits_{j\to\infty} \mu(A_j)$.
  \item $\mu\left(\bigcup_{j=1}^\infty A_j\right)\le \sum\limits_{j=1}^\infty 
  \mu(A_j)$.\fishhere
\end{enumerate}
\end{prop}

\begin{proof}
\begin{enumerate}[label=(\roman{*})]
  \item $A_2 = A_1\dot{\cup} (A_2 \cap A_1^c) \Rightarrow \mu(A_2) = \mu(A_1) +
  \underbrace{\mu(A_2\cap A_1^c)}_{\ge 0}$.
  \item Schreibe $A$ als $A = A_1\dcup (A_2\setminus A_1) \dcup (A_3\setminus
  A_2) \dcup \ldots$,
\begin{align*}
\mu(A) &= \mu(A_1) + \sum\limits_{j=2}^\infty \mu(A_j \setminus A_{j-1})\\
&= \lim\limits_{k\to\infty} \mu(A_1) + \sum\limits_{j=2}^k \mu(A_j
\setminus A_{j-1})\\
&= \lim\limits_{k\to\infty} \mu\left(
A_1\dcup (A_2\setminus A_1) \dcup \ldots \dcup (A_k\setminus A_{k-1})
 \right)\\
&= \lim\limits_{k\to\infty} \mu(A_k).
\end{align*}
%   \begin{align*}
%   \mu(A) &= \mu(A_1) + \sum\limits_{n=2}^\infty \mu(A_j \setminus A_{j-1}) \\ &=
%   \mu(A_1) + \sum\limits_{n=2}^\infty \mu(A_j \cap A_{j-1}^c) =
%   \mu(A_1) + \lim\limits_{k\to \infty}\sum\limits_{n=2}^k \mu(A_j \cap
%   A_{j-1}^c) \\ &=
%   \lim\limits_{k\to \infty} \mu(A_1) + \sum\limits_{n=2}^k \mu(A_j \cap
%   A_{j-1}^c) \\ &=
%   \lim\limits_{k\to \infty} \mu(A_1\dcup (A_2\setminus A_1^c) \dcup \ldots
%   \dcup (A_k \setminus A_{k-1}^c) = \lim\limits_{k\to\infty} \mu(A_k). 
%   \end{align*}
  \item $A\subseteq A_1$, also ist
  \begin{align*}
  A_1 = A\dcup (A_1\setminus A) = A\dcup \left(A_1 \setminus
  \bigcap_{j=1}^\infty A_j\right) = A\dcup \bigcup_{j=1}^\infty A_1\setminus
  A_j.
  \end{align*}
 Für das Maß von $A_1$ gilt also
 \begin{align*}
 \mu(A_1) &= \mu(A) + \mu\left(\bigcup_{j=1}^\infty A_1\setminus
  A_j\right) = \mu(A) + \lim\limits_{j\to\infty} \mu(A_1\setminus A_j)
  \\ &= \mu(A) + \mu(A_1) - \lim\limits_{j\to\infty}  \mu(A_j)\\
  \Rightarrow \mu(A) &= \lim\limits_{j\to\infty}  \mu(A_j).
 \end{align*}
\item 
\begin{align*}
\bigcup_{j=1}^\infty A_j &= A_1 \dcup \dot{\bigcup}_{j=1}^\infty \left(A_j
\setminus \bigcup_{k=1}^{j-1} A_k\right),\\
& A_j \supseteq A_j
\setminus \bigcup_{k=1}^{j-1} A_k,
\end{align*}
also gilt für das Maß
\begin{align*}
\mu\left(\bigcup_{j=1}^\infty A_j \right) &= \mu(A_1) +
\sum\limits_{j=2}^\infty \mu\left(A_j
\setminus \bigcup_{k=1}^{j-1} A_k\right)
\le \mu(A_1) +
\sum\limits_{j=2}^\infty \mu(A_j) \\ &=\sum\limits_{j=1}^\infty
\mu(A_j).\qedhere
\end{align*}
\end{enumerate}
\end{proof}

\begin{defn}
\label{defn:3.12}
Sei $(\Omega,\Sigma,\mu)$ ein Maßraum.
\begin{enumerate}[label=\arabic{*}.)]
  \item $N\in\Sigma$ heißt \emph{Nullmenge}, falls $\mu(N) = 0$. Eine
  Eigenschaft $E(x)$ für $x\in\Omega$ gilt \emph{fast überall}, wenn es eine
  Nullmenge $N\subseteq\Omega$ gibt, sodass
  \begin{align*}
  \forall x\in \Omega\setminus N : E(x)\text{ ist wahr}.
  \end{align*}
  \item $(\Omega,\Sigma,\mu)$ heißt \emph{vollständiger Maßraum}, falls
  $\Sigma$ mit einer Nullmenge $N$ auch jede Teilmenge von $N$ 
  enthält.\fishhere
\end{enumerate}
\end{defn}
\begin{prop}[Vervollständigung eines Maßraums]
Sei $(\Omega,\Sigma,\mu)$ ein Maßraum. Setze
\begin{align*}
\Sigma^* = \setdef{B\subseteq\Omega}{\exists A,C\in\Sigma: (A\subseteq
B\subseteq C)\land \mu(C\setminus A) = 0},
\end{align*}
und für $B\in\Sigma^*$, $\mu^*(B) = \mu(A) (=\mu(C))$. Insbesondere ist dann
jede Teilmenge einer Nullmenge in $\Sigma^*$ enthalten.

Dann gilt
\begin{itemize}
  \item $\Sigma^*$ ist eine $\sigma$-Algebra.
  \item $\mu^*$ ist ein Maß auf $\Sigma^*$.
  \item $\mu^*$ ist eindeutige Fortsetzung von $\mu$.
  \item $(\Omega,\Sigma^*,\mu^*)$ ist vollständig.\fishhere
\end{itemize}
\end{prop}
\begin{proof}
$\Sigma^*$ ist eine $\sigma$-Algebra, denn
\begin{enumerate}[label=\arabic{*}.)]
  \item $\varnothing\in\Sigma\subseteq\Sigma^*$.
  \item Sei $B\in\Sigma^*$, und $A\subseteq B\subseteq C$, mit $C\setminus
  A = N$, dann gilt
  \begin{align*}
  &C^c \subseteq B^c \subseteq A^c,\\
  &C^c, A^c \in \Sigma,\\
  &A^c \setminus C^c = A^c \cap (C^c)^c = A^c \cap C = C\setminus A,\\
  \Rightarrow &\mu(A^c\setminus C^c) = 0,\\
  \Rightarrow &B^c\in \Sigma^*.
  \end{align*}
  \item Sei 
  $B_j\in\Sigma^*$ für jedes $j\in\N$, dann gibt es $A_j,B_j\in \Sigma$, sodass
  $A_j\subseteq B_j\subseteq C_j$ und $\mu(C_j\setminus A_j) = 0$. Es gilt dann
  \begin{align*}
  A = \bigcup_{j\in\N} A_j \subseteq B = \bigcup_{j\in\N} B_j  \subseteq C =
  \bigcup_{j\in\N} C_j,
  \end{align*}
  und für $A,C\in\Sigma$ folgt,
  \begin{align*}
  C\setminus A &= C\cap A^c = \left( \bigcup_{j\in\N} C_j \right) \cap
  \left(\bigcup_{j\in\N} A_j\right)^c =
  \left( \bigcup_{j\in\N} C_j \right) \cap
  \left(\bigcap_{j\in\N} A_j^c\right) \\ &
  \subseteq   \left( \bigcup_{j\in\N} C_j \cap A_j^c \right)\\
  \overset{\ref{prop:3.11}}{\Rightarrow} &\mu(C\setminus A) \le
  \sum\limits_{k=1}^\infty \mu(C_k\setminus A_k) = 0.
  \end{align*}
\item $\mu^*$ ist Maß: Übung.
\item $\mu^*$ ist eindeutig: $\mu^*$ ist Maß, also monoton,
\begin{align*}
&\underbrace{\mu^*(A)}_{=\mu(A)} \le \mu^*(B) \le
\underbrace{\mu^*(C)}_{=\mu(C)},\\
\Rightarrow\; & \mu^*(B) = \mu(A), 
\end{align*}
also ist $\mu^*$ eindeutig.\qedhere
\end{enumerate}
\end{proof}

\begin{defn}
\label{defn:3.14} Die Vervollständigung $\lambda^{(n)}$ des
Lebesgue-Borel-Maßes $\mu^{(n)}$ heißt \emph{Lebesguemaß} auf $\R^n$.\fishhere
\end{defn}

\begin{bsp}
\label{bsp:3.15}
\begin{enumerate}[label=\arabic{*}.)]
  \item\label{bsp:3.15:1} Sei $x\in\R^n$, dann ist $\lambda^{(n)}(\{x\}) = 0$,
  denn sei
  \begin{align*}
  W_j := X_{k=1}^n \left(x_k- \frac{1}{j},x_k\right] \Rightarrow \{x\} 
  \subseteq \bigcap_{j=1}^\infty W_j,\\
  \Rightarrow \lambda^{(n)}(\{x\}) = \lim\limits_{j\to\infty}
  \lambda^{(n)}(W_j) = \lim\limits_{j\to \infty} \frac{1}{j^n} = 0
  \end{align*}
  \item Aus \ref{bsp:3.15:1} folgt für $M\subseteq\R^n$ abzählbar, dass
  $\lambda^{(n)}(M) = 0$ ist.
  \item Die Cantor-Menge ist ein Beispiel für eine überabzählbare Nullmenge.
  \item Für $a\in\R$ ist eine Hyperebene im $\R^n$ gegeben durch
  \begin{align*}
  H := \setdef{x\in\R^n}{x_1 = a}.
  \end{align*}
Falls $n\ge 2$, ist $\lambda^{(n)}(H) = 0$, denn
\begin{align*}
H=\bigcup_{k=1}^\infty H_k,\; H_k = \{a\}\times(-k,k]\times\ldots\times(-k,k],
\end{align*}
offensichtlich ist $H_k\subseteq H_{k+1}$, also folgt mit \ref{prop:3.11}, dass
$\lambda^{(n)}(H) = \lim\limits_{k\to\infty} \lambda^{(n)}(H_k)$,
\begin{align*}
H_k &= \bigcap_{l=1}^\infty
\underbrace{(a-\frac{1}{l},a)\times(-k,k]\times\ldots\times(-k,k]}_{\lambda^{(n)}(\ldots)
= \frac{1}{l}(2k)^{n-1}\to 0},\\
\Rightarrow &\lambda^{(n)}(H_k) = 0,\\
\Rightarrow &\lambda^{(n)}(H) = 0.\bsphere
\end{align*}
\end{enumerate}
\end{bsp}

\subsection{Messbare Funktionen}
\begin{defn}
\label{defn:3.16}
Seien $(\Omega,\Sigma)$ und $(\Omega',\Sigma')$ messbare Räume. Dann heißt $f:
\Omega\to\Omega'$ \emph{messbar}, falls gilt
\begin{align*}
\forall B\in\Sigma' : f^{-1}(B)\in\Sigma,
\end{align*}
oder anders geschrieben $f^{-1}(\Sigma')\subseteq\Sigma$. Man schreibt $f:(\Omega,\Sigma)\to(\Omega',\Sigma')$ ist messbar.

Falls $\Sigma' = \BB(\Omega') = \sigma(\OO_{\Omega'})$ Borel-$\sigma$ Algebra auf
$\Omega'$ ist, heißt $f: (\Omega,\Sigma) \to \Omega'$
\emph{Borel-messbar}.\fishhere
\end{defn}

\begin{prop}[Vereinfachung]
\label{prop:3.17}
Ist $M'\subseteq \PP(\Omega')$ mit $\Sigma' = \sigma(M')$, dann sind äquivalent
\begin{enumerate}[label=(\roman{*})]
  \item $f: (\Omega,\Sigma)\to(\Omega',\Sigma')$ messbar.
  \item $\forall B\in M' : f^{-1}(B) \in \Sigma$.\fishhere
\end{enumerate}
\end{prop}
\begin{proof}
\begin{enumerate}[label=(\roman{*})]
  \item $\Rightarrow f^{-1}(M')\subseteq f^{-1}(\sigma(M')) =
  f^{-1}(\Sigma')\subseteq \Sigma$.
  \item $\Rightarrow f^{-1}(\Sigma') = f^{-1}(\sigma(M'))  =
  \sigma(f^{-1}(M'))\subseteq \sigma(\Sigma) = \Sigma$,
  
  da $\sigma(f^{-1}(M'))$ die kleinste $\sigma$-Algebra ist, die $f^{-1}(M')$
  als Teilmenge enthält.\qedhere
\end{enumerate}
\end{proof}

\begin{cor}
\label{prop:3.18}
\begin{enumerate}[label=\arabic{*}.)]
  \item Ist $f: X\to Y$ stetig, dann ist $f$ Borel-messbar.
  \item Sei $(\Omega,\Sigma)$ ein messbarer Raum. Für $A\subseteq\Omega$ ist
  \begin{align*}
  \chi_A : \Omega\to\R,\;x\mapsto \begin{cases}
                                       1, & x\in A,\\
                                       0, & x\notin A,
                                       \end{cases}
  \end{align*}
die \emph{charakteristische Funktion } von $A$.

Es gilt $A\in\Sigma$ genau dann, wenn $\chi_A : (\Omega,\Sigma)\to\R$
Borel-messbar ist.\fishhere
\end{enumerate}
\end{cor}
\begin{proof}
\begin{enumerate}[label=\arabic{*}.)]
  \item Die Borel-$\sigma$ Algebra auf $X$ ist $\Sigma = \sigma(\OO_X)$, die
  auf $Y$ ist $\Sigma' = \sigma(\OO_Y)$. Wendet man nun \ref{prop:3.17} an mit
  $M' = \OO_Y$ folgt aufgrund der Stetigkeit von $f$, dass $f^{-1}{O}\in\OO_X$
  für $O\in\OO_Y$.
  \item Wähle $\BB(\R) = \sigma(\setdef{(a,b]}{a,b\in\R})$.
  
  ``$\Rightarrow$'': Sei $A\in\Sigma$, dann ist
  \begin{align*}
  \chi_A^{-1}((a,b]) &= \begin{cases}
                       \varnothing, & 0,1\notin(a,b],\\
                       A, & 1\in (a,b], 0\notin (a,b],\\
                       \Omega\setminus A & 1\notin(a,b], 0\in (a,b],\\
                       \Omega, 0,1\in(a,b]
                       \end{cases}\\ &\in \Sigma
  \end{align*}

``$\Leftarrow$'': Sei $A=\chi_A^{-1}((\frac{1}{2},1])$, dann ist $A\in\Sigma$.
\end{enumerate}
\end{proof}

\begin{prop}[Erweiterung von $\R$]
\label{prop:3.19}
Seien $\overline{\R} = \R\cup\{\infty\}\cup\{-\infty\}$,
\begin{align*}
\BB(\overline{\R}) = \setdef{B\cup E}{B\in\BB(\R)\text{ und }
E\in\{\varnothing, \{\infty\}, \{-\infty\}, \{\infty,-\infty\}},
\end{align*}
dann ist $\BB(\overline{\R})$ eine $\sigma$-Algebra, die Borel-$\sigma$-Algebra
von $\overline{\R}$.

Eine Funktion $f: \Omega\to\overline{\R}$ heißt \emph{numerische
Funktion}.\fishhere
\end{prop}

\begin{cor}
\label{prop:3.20}
$f: (\Omega,\Sigma)\to\overline{\R}$ ist Borel-messbar genau dann, wenn eine
der Bedingungen erfüllt ist,
\begin{enumerate}[label=(\roman{*})]
  \item $f^{-1}((a,\infty])\in\Sigma, a\in\RA$,
  \item $f^{-1}((a,b])\in\Sigma,\; a,b\in\RA$.\fishhere
\end{enumerate}
\end{cor}
\begin{proof}
$\setdef{(a,\infty]}{a\in\RA}$ erzeugt $\BB(\RA)$.\qedhere
\end{proof}

\begin{prop}[Bildmaß]
\label{prop:3.21}
Sei $f:(\Omega,\Sigma)\to(\Omega',\Sigma')$ messbar und $\mu:
\Sigma\to[0,\infty]$ ein Maß auf $(\Omega,\Sigma)$. Dann ist $\nu:
\Sigma'\to[0,\infty]$ mit
\begin{align*}
\nu(B) = \mu(f^{-1}(B)) \text{ für } B\in\Sigma',
\end{align*}
ein Maß auf $(\Omega',\Sigma')$, das sogenannte \emph{Bildmaß}. Schreibe
$\nu:=f(\mu)$.\fishhere
\end{prop}
\begin{proof}
\begin{enumerate}[label=(\roman{*})]
  \item $\nu(\varnothing) = \mu(f^{-1}(\varnothing)) = \mu(\varnothing) = 0$.
  \item Sei $B\in\Sigma'$, dann ist $\nu(B) = \mu(f^{-1}(B)) \ge 0$, da
  $f^{-1}(B)\in\Sigma$.
  \item Sei $\dot{\bigcup}_{j=1}^n B_j$ disjunkte Vereinigung von
  $B_j\in\Sigma'$ für $1\le j\le n$, dann ist
  \begin{align*}
  \nu\left(\dot{\bigcup}_{j=1}^n B_j \right)
  &= \mu\left(f^{-1}\left(\dot{\bigcup}_{j=1}^n B_j \right)\right)
  = \mu\left(\dot{\bigcup}_{j=1}^n f^{-1}(B_j)\right)
  \\ &= \sum\limits_{j=1}^n \mu\left(f^{-1}(B_j)\right) = \sum\limits_{j=1}^n
  \nu(B_j),
  \end{align*}
da $f^{-1}$ inklusionserhalten ist und die Urbilder disjunkter Mengen ebenfalls
disjunkt sind.\qedhere
\end{enumerate}
\end{proof}

\begin{prop}
\label{prop:3.22}
Seien $f: (\Omega,\Sigma)\to(\Omega',\Sigma')$ und $g: (\Omega',\Sigma')\to
(\Omega'',\Sigma'')$ messbar, dann ist auch $g\circ f:
(\Omega,\Sigma)\to(\Omega'',\Sigma'')$ messbar.\fishhere
\end{prop}
\begin{proof}
Sei $C\in\Sigma''$, dann ist
\begin{align*}
(g\circ f)^{-1}(C) = (f^{-1}\circ g^{-1})(C) = f^{-1}\circ
\underbrace{g^{-1}(C)}_{\in\Sigma'}\in\Sigma.\qedhere
\end{align*}
\end{proof}

\begin{prop}
\label{prop:3.23}
\begin{enumerate}[label=\arabic{*}.)]
  \item Sei $f:\Omega\to\R^n$, $f=(f_1,\ldots,f_n)$, dann
  sind folgende Aussagen äquivalent \begin{enumerate}[label=(\roman{*})]
    \item\label{prop:3.23:1:1} $f:(\Omega,\Sigma)\to\R^n$ ist Borel-messbar.
    \item\label{prop:3.23:1:2} $f_1,\ldots,f_n: (\Omega,\Sigma)\to\R$ sind alle
    Borel-messbar.
\end{enumerate}
\item $f:(\Omega,\Sigma)\to\C$ ist Borel-messbar genau dann, wenn $\Re f, \Im f
: (\Omega,\Sigma)\to\R$ beide Borel-messbar sind.\fishhere
\end{enumerate}
\end{prop}
\begin{proof}
\begin{enumerate}[label=\arabic{*}.)]
  \item ``\ref{prop:3.23:1:2}$\Rightarrow$\ref{prop:3.23:1:1}'': Folgt direkt 
 aus $f^{-1}\left(X_{j=1}^n (a_j,b_j]\right) = \bigcap_{j=1}^n
  f_j^{-1}((a_j,b_j])$.
  \item ``\ref{prop:3.23:1:1}$\Rightarrow$\ref{prop:3.23:1:2}'': Wähle $\R$
  statt $(a_j,b_j]$ für $j=2,\ldots,n$ dann folgt
  \begin{align*}
  f^{-1}\left((a_1,b_1]\times X_{j=2}^n \R] \right)= f_1^{-1}((a_1,b_1]),
  \end{align*}
und die $(a_1,b_1]$ erzeugen $\BB(\R)$, also ist $f_1$ messbar.
\item $\C$ ist homöomorph zum $\R^2$.\qedhere
\end{enumerate}
\end{proof}

\begin{prop}
\label{prop:3.24}
\begin{enumerate}[label=\arabic{*}.)]
  \item Seien $f,g: (\Omega,\Sigma)\to\RA$ Borel-messbar, dann sind es auch
  \begin{enumerate}[label=(\roman{*})]
    \item $\lambda f$ für $\lambda\in\RA$.
    \item $f+g$, falls $\forall x\in\Omega : f(x) = \pm\infty \Rightarrow g(x)\neq
  \mp\infty$.
  \item  $f\cdot g$, wobei $0\cdot\infty = 0$ zu setzen ist.
  \item $f_+ := \max\{f,0\}$.
  \item $f_- := \max\{-f,0\}$.
  \item $\abs{f} = f_+ + f_-$.
  \end{enumerate}
  \item Sei $(f_n)$ eine Funktionenfolge und alle $f_n:(\Omega,\Sigma)\to\R^n$
  Borel-messbar, dann sind auch $\sup_{n} f_n,\;\inf_n
  f_n,\; \limsup_{n\to\infty} f_n,\; \liminf_{n\to\infty} f_n$ und falls der
  Grenzwert existiert $\lim_{n\to\infty} f_n$ messbar.\fishhere
  \end{enumerate}
\end{prop}

\begin{proof}
\begin{enumerate}[label=\arabic{*}.)]
  \item Wir zeigen die Behauptung nur für $f+g$ und $\max\{f,g\}$, die anderen
  Fälle folgen analog.
  
  Seien $\phi: \Omega\to\RA^2,\;x\mapsto (f(x),\ g(x))$ und  
$P: \RA^2\to
  \RA,\;(x,y)\mapsto x+y$. Offensichtlich ist $P$ stetig, also Borel-messbar und
  daher gilt
  \begin{align*}
  (f+g)^{-1}((0,\infty]) &= (f+g)^{-1}((0,\infty))\cup (f+g)^{-1}(\{\infty\})
  \\&= \phi^{-1}(P^{-1}(0,\infty))\cup g^{-1}(\{\infty\})\cup f^{-1}(\{\infty\})
   \in\Sigma.
  \end{align*}
% \\&= f^{-1}(P^{-1}(0,\infty))\cap g^{-1}(P^{-1}(0,\infty)) \\ &
%   \cup g^{-1}(\{\infty\})\cup f^{-1}(\{\infty\})
\begin{align*}
\max\{f,g\}^{-1}((a,\infty]) = f^{-1}((a,\infty])\cup
g^{-1}((a,\infty])\in\Sigma.
\end{align*}
\item Sei $\tilde{f}(x) := \sup \setdef{f_n(x)}{n\in\N}$ für $x\in\Omega$,
dann ist
\begin{align*}
\tilde{f}^{-1}((a,\infty]) = \bigcup_{n\in\N}
\underbrace{f_n^{-1}((a,\infty])}_{\in\Sigma}\in \Sigma,
\end{align*}
und daher ist
\begin{align*}
\limsup_{n\to\infty} f_n(x) = \inf_{n\ge 1}\sup_{k\ge n} f_k(x),
\end{align*}
messbar. Ist $\limsup_{n\to\infty} f_n = \liminf_{n\to\infty} f_n$, ist auch
$\lim_{n\to\infty} f_n$ messbar, dabei ist auch $\infty$ zugelassen.\qedhere
\end{enumerate}
\end{proof}

\subsection{Lebesgue Integral}

\begin{defn}
\label{defn:3.25} $f:\Omega\to\R$ heißt \emph{einfache Funktion}, falls $f$
Borell-messbar ist und nur endlich viele Werte annimmt, d.h. $f$ ist genau dann
einfach, wenn $f$ Linearkombination\footnote{Diese sind per definitionem
endlich.} von charakteristischen Funktionen messbarer Mengen ist.\fishhere
\end{defn}

\begin{defn}[Vorläufige Definition]
\label{defn:3.26}
Sei $(\Omega,\Sigma,\mu)$ Maßraum und $E\in\Sigma$. Für eine einfache Funktion 
\begin{align*}
f = \sum\limits_{j=1}^n \lambda_j \chi_{A_j},\quad \text{mit } A_j \text{
messbar},
\end{align*}
definieren wir
\begin{align*}
\int_E f\dmu := \sum\limits_{j=1}^n \lambda_j\mu(A_j\cap E),
\end{align*}
wobei wir im Fall $\lambda_j = 0$ und $\mu(A_j) =\infty$, $\lambda_j\cdot
\mu(A_j) = 0$ setzen.\fishhere
\end{defn}

\begin{bspn}
\begin{enumerate}[label=\arabic{*}.)]
  \item Sei
  \begin{align*}
  f: (\R,\BB(\R)) \to \R,\; x\mapsto \begin{cases}
                    1, & 0\le x\le 2,\\
                    0, & \text{sonst},
                    \end{cases}
  \end{align*}
  dann lässt sich $f$ schreiben als $f =
                    1\cdot\chi_{[0,2]} + 0\cdot\chi_{\R\setminus[0,2]}$ und
                    damit ist $\int_\R f \dmu = 2$.
\item Sei
\begin{align*}
g(x) = \begin{cases}
       0, & x <0,\\
       1, & x \ge 0,
       \end{cases}
\end{align*}
dann ist $\int_\R g \dmu = \infty$.
\item Sei
\begin{align*}
h(x) &= \begin{cases}
       1, & x\in \Q,\\
       0, & x\in\R\setminus\Q,
       \end{cases}\\
&= 1 \cdot \chi_\Q + 0 \cdot \chi_{\R\setminus\Q},
\end{align*}
dann ist $\int_\R h \dmu = 1\cdot\mu(\Q) + 0\cdot\mu (\R\setminus \Q) =
0$.\bsphere
\end{enumerate}
\end{bspn}

\begin{prop}
\label{prop:3.27}
Sei $f: (\Omega,\Sigma)\to\RA$ Borel-messbar und positiv\footnote{Wir sagen $x$
ist positiv, falls $x\ge 0$ und ist echt positiv, falls $x > 0$.}, dann
existiert eine Folge $(s_n)$ einfacher Funktionen mit den Eigenschaften
\begin{enumerate}[label=(\roman{*})]
  \item $0\le s_1\le s_2\le \ldots$,
  \item $\forall x\in\Omega: \lim_{n\to\infty} s_n(x) = f(x)$.
\end{enumerate}
Ist $f$ beschränkt, so konvergiert $s_n$ auf $\Omega$ gleichmäßig gegen
$f$.\fishhere
\end{prop}
\begin{proof}
Sei
\begin{align*}
E_{n,j} := \setdef{x\in\Omega}{\frac{j-1}{2^n}\le f(x)< \frac{j}{2^n}},
\end{align*}
für $j=1,\ldots,2^n\cdot n$, und
\begin{align*}
F_n := \setdef{x\in\Omega}{f(x)\ge n} = f^{-1}([n,\infty]),
\end{align*}
dann sind $F_n$ und $E_{n,j}$ messbar. Sei
\begin{align*}
s_n(x) &:= \begin{cases}
          \frac{j-1}{2^n}, & \text{für } x\in E_{n,j}, 1\le j\le 2^n,\\
          n, & \text{für } x\in F_n,
          \end{cases}\\
&= \sum\limits_{j=1}^{2^nn} \frac{j-1}{2^n}\chi_{E_{n,j}} + n\chi_{F_n},
\end{align*}
dann ist $s_n$ messbar und einfach. $s_n(x)$ ist außerdem monoton wachsend, denn
\begin{align*}
E_{n+1,2j} \dcup E_{n+1,2j+1} = E_{n,j},
\end{align*}
mit $s_{n+1}(x) = \dfrac{2j}{2^{n+1}} =
\dfrac{j}{2^n}$ und $s_n(x) = \dfrac{j-1}{2^n}$ auf $E_{n+1,2j+1}$.

Sei nun $x\in\Omega$ fest,
\begin{itemize}
\item falls $f(x) <\infty$, folgt $\abs{s_n(x) - f(x)} < \frac{1}{2^n}$ für 
$n>f(x)$,
\item falls $f(x) = \infty$, ist $s_n(x) = n\to\infty$ für $n\to\infty$.
\end{itemize}

Ist $f$ beschränkt, also $f(x) \le c\in\R$, dann existiert ein $N\in\N$ mit
$N\ge c$, so dass $\norm{f-s_n}_\infty\le \frac{1}{2^n}$ für $n\ge N$ und damit
konvergiert $s_n$ gleichmäßig gegen $f$.\qedhere
\end{proof}

\begin{cor}
\label{prop:3.28}
Sei $f: (\Omega,\Sigma)\to\RA$, dann sind folgende Aussagen äquivalent
\begin{enumerate}[label=(\roman{*})]
  \item\label{prop:3.28:1} $f$ ist Borel-messbar.
  \item\label{prop:3.28:2} $f$ ist punktweiser limes einfacher
  Funktionen.\fishhere
\end{enumerate}
\end{cor}
\begin{proof}
\ref{prop:3.28:1}$\Rightarrow$\ref{prop:3.28:2}: $f$ ist Borel-messbar, also
auch $f_+$ und $f_-$.  $f_+$ und $f_-$ sind positiv, also nach
\ref{prop:3.27} punktweiser limes einfacher Funktionen und  da $f = f_+ - f_-$
ist damit auch $f$ punktweiser Limes einfacher Funktionen.

\ref{prop:3.28:2}$\Rightarrow$\ref{prop:3.28:1}: Einfache Funktionen sind
messbar, und damit ist $f$ als Grenzwert messbarer Funktionen messbar.\qedhere
\end{proof}

\begin{defn}
\label{defn:3.29}
Sei $(\Omega,\Sigma,\mu)$ Maßraum und $E\in\Sigma$.
\begin{enumerate}[label=(\roman{*})]
  \item Sei $f:(\Omega,\Sigma)\to\RA$ Borel-messbar und positiv, dann ist
  \begin{align*}
  \int_E f\dmu = \sup\setdef{\int_E s \dmu}{s \text{ einfach und }\forall
  x\in\Omega : 0\le s(x)\le f(x)},
  \end{align*}
\item Sei $f:(\Omega,\Sigma)\to\RA$ Borel-messbar mit $f = f_+ - f_-$.\\
Falls $\int_E f_+\dmu = \int_E f_-\dmu = \infty$, dann ist $\int_E f \dmu$
nicht definiert. Andernfalls ist
\begin{align*}
\int_E f\dmu = \int_E f_+ \dmu - \int_E f_- \dmu.
\end{align*}
\item
Falls $\int_E f_+\dmu, \int_E f_- \dmu <\infty$, was äquivalent zu $\int_E
\abs{f} \dmu <\infty$ ist, dann heißt $f$ \emph{Lebesgue integrierbar} und man
schreibt $f\in L_1(\Omega,\Sigma,\mu)$.\fishhere
\end{enumerate}
\end{defn}


\newpage
\begin{prop}[Eigenschaften des Lebesgue Integrals]
\label{prop:3.30}
Seien $(\Omega,\Sigma,\mu)$ Maßraum, $f,g: \Omega\to\RA$ Borel-messbar und
$A\in\Sigma$, dann gilt
\begin{enumerate}[label=(\roman{*})]
  \item\label{prop:3.30:1} $f\in L_1(\Omega,\Sigma,\mu)\Leftrightarrow
  \abs{f}\in L_1(\Omega,\Sigma,\mu)$, in diesem Fall gilt
  \begin{align*}
  \abs{\int_A f\dmu} \le \int_A \abs{f}\dmu.
  \end{align*}
\item\label{prop:3.30:2} Sei $f\in L_1(\Omega,\Sigma,\mu)$ und $\lambda\in\R$,
dann ist $\lambda f\in L_1(\Omega,\Sigma,\mu)$ und 
\begin{align*}
\int_A \lambda f \dmu = \lambda\int_A f\dmu.
\end{align*}
\item\label{prop:3.30:3} Sei $f\in L_1(\Omega,\Sigma,\mu)$ und $\abs{g}\le f$,
dann ist $g\in L_1(\Omega,\Sigma,\mu)$ und
\begin{align*}
  \abs{\int_A g\dmu}\le \int_A f\dmu.
\end{align*}
\item\label{prop:3.30:4} Sei $\mu(A)<\infty$ und $f\big|_A$ beschränkt, dann ist 
\begin{align*}
\chi_A\cdot f\in
L_1(\Omega,\Sigma,\mu).
\end{align*}
\item\label{prop:3.30:5} Seien $f,g\in L_1(\Omega,\Sigma,\mu)$ und $f\le g$,
dann ist
\begin{align*}
\int_A f\dmu \le \int_A g\dmu.
\end{align*}
\item\label{prop:3.30:6} Sei $f\in L_1(\Omega,\Sigma,\mu)$, dann ist
\begin{align*}
\int_A f\dmu = \int_\Omega
\chi_A f \dmu.
\end{align*}
\item\label{prop:3.30:7} Sei $\mu(A) = 0$, dann ist
\begin{align*}
 \int_A f\dmu = 0.      
\end{align*}
\item\label{prop:3.30:8} Sei $A_1\subseteq A_2$ und $\mu(A_2\setminus A_1) =
0$, dann ist
\begin{align*}
\int_{A_1} f\dmu = \int_{A_2} f\dmu.
\end{align*}
\item\label{prop:3.30:9} Sei $f\in L_1(\Omega,\Sigma,\mu)$,
dann ist
\begin{align*}
\mu\left(\setdef{\omega\in\Omega}{f(\omega) = \pm \infty} \right) = 0.
\end{align*}
\item\label{prop:3.30:10} Sei $f\in L_1(\Omega,\Sigma,\mu)$ und positiv,
$A_1\subseteq A_2$, dann ist
\begin{align*}
\int_{A_1} f\dmu \le \int_{A_2} f\dmu.\fishhere
\end{align*}
\end{enumerate}
\end{prop}
\begin{proof}
``\ref{prop:3.30:1}'': Ist $f$ numerisch, dann ist $f = f_+ -f_-$ und $\abs{f} =
f_+ + f_-$, die Äquivalenz folgt aus der Definition. Für das Integral gilt
\begin{align*}
\int_A f\dmu &= \int_A f_+ \dmu - \underbrace{\int_a f_- \dmu}_{\ge 0}
\le \int_A f_+\dmu + \int_A f_- \dmu  \\ &= \sup\setdef{\int_A s\dmu}{0\le s\le
f_+} + \sup\setdef{\int_A t \dmu}{0\le t\le f_-} \\ &= \sup\setdef{\int_A
\sigma}{0\le\sigma\le f_++f_-} = \int_A (f_+ + f_-)\dmu.
\end{align*}
``\ref{prop:3.30:2},\ref{prop:3.30:3}'': Übung.

``\ref{prop:3.30:4}'': $\abs{f(x)} \le C$ für $x\in A$, also sind auch
$\abs{f_+},\abs{f_-}\le C$. Sei $s$ einfach mit $0\le s\le \chi_A f_+$, dann ist
\begin{align*}
\int_A s\dmu \le C\mu(A) \Rightarrow \int_\Omega (\chi_A f)_+ \dmu \le C\mu(A).
\end{align*}

``\ref{prop:3.30:5}'': Aus $f\le g$ folgt,
\begin{align*}
&f_+\le g_+ \Rightarrow \int_A f_+ \dmu \le \int_A g_+ \dmu,\\
&g_-\le f_- \Rightarrow \int_A g_- \dmu \le \int_A f_-\dmu.
\end{align*}

``\ref{prop:3.30:6},\ref{prop:3.30:7}'': Klar, vergleiche auf
\ref{prop:3.30:4}.

``\ref{prop:3.30:8}'': Für jede einfache Funktion auf $A_2$ gilt Gleichheit.

``\ref{prop:3.30:9}'': Falls
$\mu\left(\setdef{\omega\in\Omega}{f(\omega)=\infty}\right)>0$, gilt für
\begin{align*}
s_n(x) = \begin{cases}
0, & \Omega\setminus A,\\
n, & A=\setdef{\omega}{f(\omega)=\infty},
\end{cases}
\end{align*}
und damit $0\le s_n\le f$, also auch
\begin{align*}
&\int_\Omega s_n\dmu = n\mu(A) \to \infty,\\
\Rightarrow\;&\int_\Omega f_+\dmu = \infty.
\end{align*}

``\ref{prop:3.30:10}'': $f\chi_{A_1}, f\chi_{A_2}\in L_1$ und $f\chi_{A_1}\le
f\chi_{A_2}$.
\end{proof}

\subsection{Konvergenzsätze und mehr}

\begin{prop}
\label{prop:3.31}
Seien $s_n,\sigma: \Omega\to\R$ einfache Funktionen mit $0\le s_1\le s_2\le
\ldots$ und $0\le \sigma\le \lim_{n\to\infty} s_n$. Dann gilt
\begin{align*}
\int_\Omega \sigma\dmu \le \lim\limits_{n\to \infty} \int_\Omega
s_n\dmu.\fishhere
\end{align*}
\end{prop}
\begin{proof}
$\sigma$ ist einfach, also Linearkombination charaktersitischer Funktionen,
\begin{align*}
\sigma = \sum\limits_{j=1}^N \alpha_j \chi_{A_j},\quad \forall A_j \exists
\alpha_j \in\R : \sigma^{-1}(\alpha_j) = A_j,
\end{align*}
also sind die $A_j$ messbar und $A_i\cap A_k =\varnothing$ für $i\neq k$.

Sei nun $\beta>1$ und $B_n = \setdef{\omega\in\Omega}{\sigma(\omega) \le \beta
s_n(\omega)}$ eine Folge von Mengen, dann existiert für jedes $\omega\in\Omega$
ein $n\in\N$, so dass $\omega\in B_n$ und daher gilt
\begin{itemize}
  \item $B_k \subseteq B_l$ für $l\ge k$,
  \item $\bigcup_{n\in\N} B_n = \Omega$, $\Rightarrow A_j = A_j\cap\Omega =
  \bigcup_{n\in\N} (A_j\cap B_n)$
  \item $\sigma \chi_{B_n} \le \beta s_n$,
\end{itemize}
Also gilt
\begin{align*}
\int_\Omega \sigma \dmu &= \sum\limits_{j=1}^n \alpha_j \mu(A_j) 
= \sum\limits_{j=1}^n \alpha_j \lim\limits_{m\to\infty} \mu(A_j\cap B_m)
= \lim\limits_{m\to\infty} \sum\limits_{j=1}^n \alpha_j \mu(A_j\cap B_m)
\\ &= \lim\limits_{m\to\infty} \int_\Omega \sigma \chi_{B_m}\dmu
\le  \lim\limits_{m\to\infty} \int_\Omega \beta s_m \dmu
= \beta\lim\limits_{m\to\infty} \int_\Omega s_m \dmu. 
\end{align*}
Da für $\beta >1$ beliebig gilt $\int_\Omega \sigma\dmu \le \beta
\lim\limits_{m\to\infty} \int_\Omega s_m \dmu$, folgt die Behauptung.\qedhere 
\end{proof}

\begin{prop}[Satz von der monotonen Konvergenz (B. Levi 1906)]
\label{prop:3.32}
Seien $f_n: \Omega\to\RA$ messbare numerische Funktionen mit $0\le f_1\le f_2\le
\ldots$, dann gilt
\begin{align*}
\lim\limits_{n\to\infty} \int_\Omega f_n \dmu = \int_\Omega
\lim\limits_{n\to\infty} f_n \dmu.\fishhere
\end{align*}
\end{prop}
\begin{proof}
\begin{enumerate}[label=\arabic{*}.)]
  \item $f(x) = \lim\limits_{n\to\infty} f_n(x)$ ist wohldefiniert und nach
  \ref{prop:3.24} messbar.
  \item $f_n\le f \Rightarrow \int_\Omega f_n \dmu\le \int_\Omega f\dmu$ und
  daher gilt insbesondere,
  \begin{align*}
  \lim\limits_{n\to\infty}  \int_\Omega f_n \dmu\le
  \int_\Omega f\dmu.
  \end{align*}
  \item Sei $s$ einfach mit $0\le s\le f$, $\beta > 1$ und $B_n =
  \setdef{\omega\in\Omega}{s(\omega) \le \beta\ f_n(\omega)}$, dann gilt
  \begin{itemize}
    \item $B_1\subseteq B_2\subseteq \ldots$,
    \item $s\ \chi_{B_n} \le \beta\ f_n$ und $s\ \chi_{B_n}$ einfach,
    \item $0\le s\ \chi_{B_1} \le s\ \chi_{B_2} \le \ldots$.
  \end{itemize}
  Offensichtlich ist $s = \lim\limits_{n\to\infty} s\chi_{B_n}$, also gilt
\begin{align*}
\int_\Omega s\dmu \le \lim\limits_{n\to\infty} \int_\Omega s\chi_{B_n}\dmu
\le \beta \lim\limits_{n\to\infty} \int_\Omega f_n\dmu.
\end{align*}
und da $\beta > 1$ beliebig war, folgt
\begin{align*}
\int_\Omega s\dmu \le \lim\limits_{n\to\infty} \int_\Omega f_n\dmu,
\end{align*}
insbesondere gilt dann
\begin{align*}
\int_\Omega f\dmu = \sup \setdef{\int_\Omega s\dmu}{0\le s\le f, s \text{
einfach}} \le \lim\limits_{n\to\infty} \int_\Omega f_n\dmu.\qedhere
\end{align*}
\end{enumerate}
\end{proof}

\begin{prop}[Additivität]
\label{prop:3.33}
Seien $f,g\in L_1(\Omega,\Sigma,\mu)$, dann ist $f+g\in L_1(\Omega,\Sigma,\mu)$
und es gilt für $A\in\Sigma$,
\begin{align*}
\int_A (f+g) \dmu = \int_A f\dmu + \int_A g\dmu.\fishhere 
\end{align*}
\end{prop}
\begin{proof} 
\begin{enumerate}[label=\arabic{*}.)]
  \item Seien $f,g\ge 0$, dann ist $f+g$ nach \ref{prop:3.24} messbar.
  
  Es existieren also Folgen $(s_n), (t_n)$ einfacher Funktionen mit $0\le s_1\le
  s_2\le \ldots$ und $0\le t_1\le t_2\le \ldots$, sodass
  \begin{align*}
  f(x) = \lim\limits_{n\to\infty} s_n(x),\quad   g(x) = \lim\limits_{n\to\infty} t_n(x),
  \end{align*}
Sei $\sigma_n = s_n+t_n$, dann ist $\sigma_n$ einfach und monoton steigend und
es gilt $(f+g)(x) = \lim_{n\to\infty} \sigma_n(x)$.

Es folgt daher mit \ref{prop:3.32}, dass
\begin{align*}
\int_\Omega (f+g) \dmu &= \lim\limits_{n\to\infty}\int_\Omega \sigma_n \dmu
= \lim\limits_{n\to\infty}\int_\Omega s_n \dmu +
\lim\limits_{n\to\infty}\int_\Omega t_n \dmu
\\ &= \int_\Omega f \dmu + \int_\Omega g\dmu.
\end{align*}
Sei nun $A\in\Sigma$, dann ist
\begin{align*}
\int_A (f+g)\dmu &= \int_\Omega (f+g)\ \chi_A \dmu = \int_\Omega f\ \chi_A\dmu +
\int_\Omega g\ \chi_A \dmu \\ &= \int_A f \dmu + \int_A g\dmu.
\end{align*}
\item Seien nun $f,g\in L_1(\Omega,\Sigma,\mu)$, und
\begin{align*}
A_1 &:= \setdef{\omega\in\Omega}{f(\omega)\ge 0 \text{ und } g(\omega) < 0
\text{ und } (f+g)(\omega) \ge 0}\\
&= f^{-1}([0,\infty]) \cap g^{-1}([-\infty,0])\cap (f+g)^{-1}([0,\infty]),
\end{align*}
dann ist
\begin{align*}
&\int_{A_1} f\dmu = \int_{A_1} \underbrace{f+g}_{\ge 0} + \underbrace{-g}_{\ge
0} \dmu,\\
\Rightarrow\; & \int_{A_1} (f+g)\dmu = \int_{A_1} f\dmu + \int_{A_1} g\dmu.
\end{align*}
Entsprechend für
\begin{align*}
A_2 := \setdef{\omega\in\Omega}{f(\omega)\ge 0 \text{ und } g(\omega) < 0
\text{ und } (f+g)(\omega) < 0},
\end{align*}
usw.\qedhere
\end{enumerate}
\end{proof}

\begin{cor}
\label{prop:3.34}
Seien $f_n: (\Omega,\Sigma)\to \RA$ messbar und $f_n\ge 0$, dann gilt
\begin{align*}
\int_\Omega \left( \sum\limits_{n=1}^\infty f_n  \right) \dmu
= \sum\limits_{n=1}^\infty \int_\Omega f_n \dmu.\fishhere
\end{align*}
\end{cor}
\begin{proof}
Zum Beweis betrachten wir die Reihe als Grenzwert ihrer Partialsummen
\begin{align*}
\int_\Omega \left( \sum\limits_{n=1}^\infty f_n  \right) \dmu
&= \int_\Omega \left( \lim\limits_{N\to\infty} \sum\limits_{n=1}^N f_n 
\right) \dmu
\overset{\ref{prop:3.32}}{=}
\lim\limits_{N\to\infty} \int_\Omega \left(  \sum\limits_{n=1}^N f_n 
\right) \dmu
\\ &\overset{\ref{prop:3.33}}{=} \lim\limits_{N\to\infty} \sum\limits_{n=1}^N
\int_\Omega f_n \dmu
= \sum\limits_{n=1}^\infty \int_\Omega f_n\dmu.\qedhere
\end{align*}
\end{proof}

\begin{bsp}
\label{bsp:3.35}
Sei $\Omega = \N,\;\Sigma = \PP(\N)$ und $\mu(A) = \card A$, dann gilt:
\begin{enumerate}[label=\arabic{*}.)]
  \item Jede Funktion $f: \N\to[0,\infty]$ ist messbar und es gilt
  \begin{align*}
  \int_\N f \dmu = \sum\limits_{n=1}^\infty f(n),
  \end{align*}
denn betrachten wir $f_n:= f\cdot\chi_{\{n\}}$, dann ist $f_n \ge 0$ und $f =
\sum\limits_{n=1}^\infty f_n$ und wir können \ref{prop:3.34} anwenden.
\item
Sei $f_n: \N\to[0,\infty]$ und $a_{nk} = f_n(k)$, dann gilt
\begin{align*}
\sum\limits_{n=1}^\infty \sum\limits_{k=1}^\infty a_{nk} =
\sum\limits_{k=1}^\infty \sum\limits_{n=1}^\infty a_{nk},  
\end{align*}
denn $\sum\limits_{k=1}^\infty a_{nk} = \int_\N f_n \dmu$ und wir können erneut
\ref{prop:3.34} anwenden.\bsphere
\end{enumerate}
\end{bsp}

\begin{lem}[Lemma von Fatou (1906)]
\label{prop:3.36}
Sei $f_n:(\Omega,\Sigma)\to\RA$ messbar, $f_n\ge 0$, dann gilt
\begin{align*}
\int_\Omega \left(\liminf\limits_{n\to\infty} f_n \right) \dmu
\le \liminf\limits_{n\to\infty} \int_\Omega f_n \dmu.
\end{align*}
Ist darüber hinaus $f_n\in L_1(\Omega,\Sigma,\mu)$ und gibt es ein $M\in\R$,
sodass $\int_\Omega f_n \dmu \le M$ für alle $n\in\N$, dann ist
$\liminf\limits_{n\to\infty} f_n\in L_1(\Omega,\Sigma,\mu)$ und es gilt
\begin{align*}
\int_\Omega \left(\liminf\limits_{n\to\infty} f_n \right) \dmu \le M.\fishhere
\end{align*}
\end{lem}
\begin{proof}
Sei $f := \liminf\limits_{n\to\infty} f_n$, dann ist $f$ positiv und nach
\ref{prop:3.24} messbar, ebenso ist $g_n := \inf\setdef{f_j}{j\ge n}$ messbar.
Nun gilt
\begin{itemize}
  \item $g_n\le f_j$ für $j\ge n$,
  \item $0\le g_1\le g_2 \le \ldots$,
  \item $g_n\to f$ punktweise,
\end{itemize}
also ist auch
\begin{align*}
\int_\Omega g_n\dmu \le \inf\setdef{\int_\Omega f_j \dmu}{j\ge n}.
\end{align*}
Aus dem Satz über monotone Konvergenz folgt somit die Behauptung,
\begin{align*}
\int_\Omega f\dmu &= \int_\Omega \left(\lim\limits_{n\to\infty} g_n \right)
\overset{\ref{prop:3.32}}{=} \lim\limits_{n\to\infty}  \int_\Omega g_n\dmu
\le \lim\limits_{n\to\infty} \inf \setdef{\int_\Omega f_j \dmu}{j\ge n}
\\ &= \liminf\limits_{n\to\infty} \int_\Omega f_n\dmu.\qedhere
\end{align*}
\end{proof}

\begin{prop}[Satz von der Majorisierten Konvergenz (Lebesgue 1910)]
\label{prop:3.37}
Seien $f_n,f: (\Omega,\Sigma)\to\RA$ und $f_n\to f$. Existiert ein $g\in
L_1(\Omega,\Sigma,\mu)$ mit $\abs{f_n}\le g$ für $n\in\N$, dann sind $f_n,f\in
L_1(\Omega,\Sigma,\mu)$ und es gilt
\begin{align*}
\int_\Omega \abs{f_n-f} \dmu \to 0,\quad
\int_\Omega f_n\dmu \to \int_\Omega f\dmu.\fishhere  
\end{align*}
\end{prop}
\begin{proof}
Aus \ref{prop:3.30} folgt, dass $f_n\in L_1(\Omega,\Sigma,\mu)$. Nun ist
$\abs{f} = \lim\limits_{n\to\infty} \abs{f_n} \le g$, also ist auch $f\in
L_1(\Omega,\Sigma,\mu)$.

Wegen $\abs{f_n-f} \le \abs{f_n}+\abs{f} \le 2g$ gilt
\begin{align*}
g_n := 2g - \abs{f_n-f} \ge 0,\quad
 g_n\to 2g,
\end{align*}
und damit erhalten wir, 
\begin{align*}
\int_\Omega 2 g \dmu &= \int_\Omega
\left(\lim\limits_{n\to\infty} g_n\right) \dmu
\le \liminf\limits_{n\to\infty} \int_\Omega g_n\dmu \\ &= \int_\Omega 2g\dmu +
\liminf\limits_{n\to\infty} \int_\Omega -\abs{f_n-f}\dmu
\\ &= \int_\Omega 2g \dmu - \underbrace{\limsup\limits_{n\to\infty} \int_\Omega
\abs{f_n-f}\dmu}_{\ge 0},
\end{align*}
und daher ist $\limsup\limits_{n\to\infty} \int_\Omega \abs{f_n-f} \dmu = 0$.
Nun sind die $f_n$ nichtnegativ, also existiert auch der Grenzwert und stimmt
mit dem limes superior überein, es gilt also,
\begin{align*}
\lim\limits_{n\to\infty} \int_\Omega \abs{f_n-f} = 0.
\end{align*}
Mithilfe der Dreiecksungleichung für Integrale folgt daher die übrige
Behauptung,
\begin{align*}
\abs{\int_\Omega f_n\dmu - \int_\Omega f \dmu} \le \int_\Omega \abs{f_n-f}\dmu
\to 0.\qedhere
\end{align*}
\end{proof}

\begin{prop}[Tschebyscheff-Ungleichung]
\label{prop:3.38}
Sei $f\in L_1(\Omega,\Sigma,\mu)$ und $f\ge 0$, dann gilt für
jedes $C\in(0,\infty]$,
\begin{align*}
\mu\left(\setdef{\omega\in\Omega}{f(\omega)\ge C}\right)
\le \frac{1}{C} \int_\Omega f\dmu.\fishhere
\end{align*}
\end{prop}
\begin{proof}
$\Omega' = \setdef{\omega\in\Omega}{f(\omega)\ge C} =
f^{-1}([C,\infty])\in\Sigma$
\begin{align*}
\int_\Omega f\dmu = \int_{\Omega'} f\dmu + \int_{\Omega\setminus\Omega'} f\dmu
\ge \int_{\Omega'} f\dmu  \ge \int_{\Omega'} C \dmu  = C\mu(\Omega').\qedhere   
\end{align*}
\end{proof}

\begin{cor}
\label{prop:3.39}
\begin{enumerate}[label=\arabic{*}.)]
  \item Ist $\int_\Omega \abs{f} \dmu = 0$, dann folgt $f=0$ fast überall.
  \item Ist $f\ge 0$ und $\int_\Omega f \dmu < \infty$, dann ist $f(\omega)
  <\infty$ fast überall auf $\Omega$.
\end{enumerate}
\end{cor}
\begin{proof}
\begin{enumerate}[label=\arabic{*}.)]
  \item Sei $A_n = \abs{f}^{-1}([\frac{1}{n},\infty])$, dann ist $A_n\in\Sigma$
  und da $A_n\subseteq A_{n+1}$, gilt
\begin{align*}
&\mu(A_n)\le n \underbrace{\int_\Omega \abs{f}\dmu}_{=0},\\
&\mu(\setdef{\omega\in\Omega}{f(\omega)\neq0}) = 
\mu\left(\bigcup_{n=1}^\infty A_n \right)
= \lim\limits_{n\to\infty} \mu(A_n) = 0.
\end{align*}
  \item Folgt direkt aus \ref{prop:3.38}.\qedhere
\end{enumerate}
\end{proof}

\begin{cor}
\label{prop:3.40}
Seien $f,g:(\Omega,\Sigma)\to\RA$ messbar.
\begin{enumerate}[label=(\roman{*})]
\item\label{prop:3.40:1}
Gilt $f=g$ fast überall, dann ist $\int_A f\dmu = \int_A g\dmu$ für alle
$A\in\Sigma$.
\item\label{prop:3.40:2}
Sind $f,g$ integrierbar und $\int_A f\dmu = \int_A g\dmu$ für alle
$A\in\Sigma$, dann ist $f=g$ fast überall.\fishhere 
\end{enumerate}
\end{cor}
\begin{proof}
``\ref{prop:3.40:1}'': trivial.

``\ref{prop:3.40:2}'': Sei ohne Einschränkung $g = 0$, dann ist für alle
$A\in\Sigma$, $\int_A f\dmu = 0$.

Wähle $A=f^{-1}([0,\infty])\in\Sigma$, dann ist
\begin{align*}
\int_\Omega f_+ \dmu = \int_A f \dmu = 0,
\end{align*}
und aus \ref{prop:3.39} folgt, $f_+ = 0$ fast überall. Analog verfährt man mit
$f_-$ und da $f = f_+-f_- = 0$ fast überall, ist auch $f=g$ fast
überall.\qedhere
\end{proof}

\begin{bem}
\label{bem:3.41}
Für $f,g:(\Omega,\Sigma)\to\RA$ messbar ist
\begin{align*}
f\sim g \Leftrightarrow f = g \text{ fast überall},
\end{align*}
eine Äquivalenzrelation.

Zu jedem $f\in L_1(\Omega,\Sigma,\mu)$ existiert ein
$\tilde{f}\in L_1(\Omega,\Sigma,\mu)$ so, dass $\tilde{f}\sim f$ und
$\tilde{f}(x) <\infty, \forall x\in\Omega$. Insbesondere gilt für jedes $A\in\Sigma$,
\begin{align*}
\int_A f\dmu = \int_A \tilde{f}\dmu.
\end{align*}
denn nach \ref{prop:3.39} gilt $\mu(f^{-1}(\{-\infty,\infty\})) = 0$.\maphere
 %TODO: was soll das hier :)
\end{bem}

\begin{defn}
\label{defn:3.42}
Sei $f:(\Omega,\Sigma)\to\RA$ messbar, dann heißt die Funktion
\emph{$\mu$-messbar}, falls $f$ messbar bezüglich
$(\Omega,\Sigma^*,\mu^*)$, d.h. der Vervollständigung von
$(\Omega,\Sigma,\mu)$, ist.\fishhere
\end{defn}

\begin{bem}
\label{bem:3.43}
\begin{enumerate}[label=\arabic{*}.)]
  \item Sei $f\in L_1(\Omega,\Sigma^*,\mu^*)$ und $f=g$ fast überall, dann ist
  auch $g\in L_1(\Omega,\Sigma^*,\mu^*)$.
  \item Alle Konvergenzsätze gelten auch für $\mu$-messbare Funktionen, wobei
  die Voraussetzungen nur fast überall gelten müssen.\maphere
\end{enumerate}
\end{bem}

\begin{bspn}
Seien $f,f_n$ messbar, $f=\lim\limits_{n\to\infty} f_n$ fast überall,
$\abs{f_n}\le g$ fast überall und $\int_\Omega g \dmu < \infty$, dann gilt
\begin{align*}
\lim\limits_{n\to\infty} \int_\Omega \abs{f_n-f}\dmu = 0.\bsphere
\end{align*}
\end{bspn}

\subsection{Riemann- und Lebesgueintegral}
\begin{prop}
\label{prop:3.44}
Seien $a,b\in\R$, $f:[a,b]\to\R$ beschränkt, dann sind äquivalent
\begin{enumerate}[label=(\roman{*})]
  \item $f$ ist Riemann-integrierbar,
  \item Das Lebesgue-Maß $\lambda^{(1)}$ der Unstetigkeitsstellen von $f$ ist 0.
\end{enumerate}
Sind beide Voraussetzungen erfüllt, dann gilt
\begin{align*}
\int_{[a,b]} f\dlamb{(1)} =\int_a^b f(x)\dx.\fishhere
\end{align*}
\end{prop}
\begin{proof}
Ohne Einschränkung sind $a=0,\;b=1$. Wir sagen $f$ ist genau dann
Riemann-integrierbar, wenn Ober- und Untersumme gegen denselben Wert
konvergieren.

\begin{enumerate}[label=\arabic{*}.)]
  \item Für das Riemann-Integral benötigen wir eine Unterteilung des 
Intervalls $[0,1]$, für $n\in\N$ seien also
\begin{align*}
&Y_1 = \left[0,\frac{1}{2^n}\right], Y_j =
\left(\frac{j-1}{2^n},\frac{j}{2^n}\right],
\\ &m_j = \inf \setdef{f(x)}{x\in
Y_j},\\ &M_j = \sup \setdef{f(x)}{x\in Y_j},
\end{align*}
für $j=2,3,\ldots,2^n$. Nun können wir $f$ durch folgende Funktionenfolgen von
unten und von oben annähern,
\begin{align*}
g_n(x) := \begin{cases}m_j, & \text{für }x\in Y_j,\\0, &
  x\in\R\setminus[0,1],\end{cases}\quad G_n(x) := \begin{cases}M_j, & \text{für }x\in Y_j,\\0, &
  x\in\R\setminus[0,1],\end{cases}
\end{align*}
wobei $g_n$ und $G_m$ per definitionem einfach sind. Aufgrund der Konstruktion
gilt $g_n \le g_{n+1} \le G_{n+1} \le G_n$.

``Unter''- und ``Obersumme'' können wir also wie folgt definieren,
\begin{align*}
&U_n = \int_{[0,1]} g_n \dlamb{(1)},\\
&O_n = \int_{[0,1]} G_n \dlamb{(1)}.
\end{align*}

Seien $g = \lim\limits_{n\to\infty} g_n,\; G = \lim\limits_{n\to\infty} G_n$,
beide Grenzwerte existieren, da $g_n$ und $G_n$ monoton sind. $g$ und $G$ sind
als Grenzwerte einfacher Funktionen messbar und beschränkt, da $f$ beschränkt
ist. Nun ist
\begin{align*}
\inf_{[0,1]} f \le g_n(x) \le g(x) \le G(x) \le G_n(x) \le \sup_{[0,1]} f.
\end{align*}
Nach \ref{prop:3.30} sind
$g\cdot \chi_{[0,1]},\; G\cdot \chi_{[0,1]} \in
L_1(\RA,\BB(\RA)^*,\lambda^{(1)}). $
\item Sei $D$ die Menge der Unstetigkeitsstellen von $f$, dann ist
\begin{align*}
D \subseteq\underbrace{\setdef{\frac{j}{2^n}}{n\in\N\text{ und
}j=0,\ldots,2^n}}_{\text{abzählbar, also Nullmenge}}\cup\setdef{x}{g(x)\neq
G(x)}
\end{align*}
\item ``$\Rightarrow$'': Sei $f$ Riemann-integrierbar und $I$ der Wert des
Integrals, dann ist $\lim\limits_{n\to\infty} U_n = \lim\limits_{n\to\infty} O_n
= I$,

Majorisierte Konvergenz: $\abs{g_n},\abs{G_n} \le \max\left\{ \abs{\inf f},
\abs{\sup f} \right\}$
\begin{align*}
&I = \int_{[0,1]} g\dlambda^{(1)} = \int_{[0,1]} G\dlambda^{(1)}\\
\Rightarrow\;& \int_{[0,1]} (G-g)\dlambda^{(1)} = 0,
\end{align*}
also ist $G=g$ fast überall und daher ist $\lambda^{(1)}(D) = 0$. Da $g\le
f\le G$, ist $g=f=G$ fast überall und es gilt
\begin{align*}
\int_{[0,1]} f\dlamb{(1)} = \int_{[0,1]} g\dlamb{(1)} = I = \int_0^1 f(x)\dx. 
\end{align*}
\item ``$\Leftarrow$'': Sei $\lambda^{(1)}(D) = 0$, dann ist $g=G$ fast überall
und damit auch $g=f=G$ fast überall. Damit gilt
\begin{align*}
\lim\limits_{n\to\infty} U_n = \int_{[0,1]} g\dlamb{(1)} = \int_{[0,1]}
G\dlamb{(1)} = \lim\limits_{n\to\infty} O_n,
\end{align*}
und $f$ ist Riemann-integrierbar\footnote{Dieser Beweis ist eigentlich
unvollständig, da man zeigen müsste, dass jede Riemannsumme konvergiert, man
kann den allgemeinen Fall aber aus dem hier konstruierten Spezialfall
herleiten.}.
\end{enumerate}
\end{proof}

\begin{prop}
\label{prop:3.45}
Sei $I=(a,b)$ und $-\infty\le a<b\le\infty$, $f: I\to\R$ sei
Riemann-integrierbar über jedem $[c,d]\subseteq I$, dann sind äquivalent
\begin{enumerate}[label=(\roman{*})]
  \item\label{prop:3.45:1} $f$ ist Lebesgue-integrierbar über $I$,
  \item\label{prop:3.45:2} $\abs{f}$ ist uneigentlich Riemann-integrierbar über
  $I$, d.h.
  \begin{align*}
  \lim_{d\uparrow b} \lim_{c\downarrow a} \int_c^d \abs{f(x)}\dx \text{
  existiert}.
  \end{align*}
\end{enumerate}
Falls beide Aussagen gelten, ist $\int_a^b f(x)\dx = \int_I f\dmu$.\fishhere 
\end{prop}
\begin{proof}
Seien $(a_n),(b_n)$ Folgen mit $a<a_n<b_n<b$ und $b_n\uparrow b$,
$a_n\downarrow a$ monoton.
\begin{enumerate}[label=\arabic{*}.)]
  \item\label{proof:3.45:1}  Aus \ref{prop:3.44}
  folgt, dass $\abs{f}\chi_{[a_n,b_n]}$ Lebesgue-integrierbar ist, also ist
  auch $\abs{f} = \lim\limits_{n\to\infty} \abs{f}\chi_{[a_n,b_n]}$ Lebesgue-messbar.
  
  Außerdem ist $\abs{f}\chi_{[a_n,b_n]}\le \abs{f}\chi_{[a_{n+1},b_{n+1}]}$,
  also gilt
\begin{align*}
\lim\limits_{n\to\infty} \int_{a_n}^{b_n} \abs{f(x)}\dx
= \lim\limits_{n\to\infty} \int_\R \abs{f}\chi_{[a_n,b_n]}\dmu
\overset{\ref{prop:3.33}}{=} \int_\R \abs{f}\dmu.  
\end{align*}
\item ``\ref{prop:3.45:2}$\Rightarrow$\ref{prop:3.45:1}'':
Aus \ref{prop:3.45:2} folgt, $\lim\limits_{n\to\infty} \int_{a_n}^{b_n}
\abs{f(x)}\dx$ existiert und $=\int_\R \abs{f}\dmu<\infty$, also ist $f$
Lebesgue-integrierbar.
\item ``\ref{prop:3.45:1}$\Rightarrow$\ref{prop:3.45:2}'': $\int_\R
\abs{f}\dmu<\infty$, also existiert $\lim\limits_{n\to\infty} \int_{a_n}^{b_n}
\abs{f(x)}\dx$ und daraus folgt \ref{prop:3.45:1}. 
\item Die selben Überlegungen wie in \ref{proof:3.45:1} für $f$ statt $\abs{f}$
und majorisierter statt monotoner Konvergenz mit
\begin{align*}
\abs{f \chi_{[a_n,b_n]}} \le \abs{f},
\end{align*}
im Fall dass \ref{prop:3.45:1} und \ref{prop:3.45:2} gelten, liefert
\begin{align*}
\int_a^b f(x)\dx =\int_I f\dmu.\qedhere 
\end{align*}
\end{enumerate}
\end{proof}

\begin{bsp}
\label{bsp:3.46}
$f(x) = \dfrac{\sin x}{x},\; I = (0,\infty)$. Bei $x=0$ ist $f$ stetig
ergänzbar durch $f(0) = 1$
\begin{align*}
\int_1^d \frac{\sin x}{x} \dx = -\underbrace{\frac{1}{x}\cos x\big|_1^d}_{\to
\cos 1} - \int_1^d \underbrace{\frac{\cos x}{x^2}}_{\abs{\cdot}\le
\frac{1}{x^2}}\dx < \infty,
\end{align*}
also ist $f$ über $I$ Riemann-integrierbar, aber
\begin{align*}
\int_\pi^{(n+1)\pi} \abs{\frac{\sin x}{x}} \dx
&= \sum\limits_{j=1}^n \int_{j\pi}^{(j+1)\pi} \abs{\frac{\sin x}{x}} \dx
\\ &\ge \sum\limits_{j=1}^n \frac{1}{(j+1)\pi} \int_{j\pi}^{(j+1)\pi} \abs{\sin
x}\dx \\ &= \left(\sum\limits_{j=1}^n \frac{1}{(j+1)\pi} \right)\int_0^\pi
\abs{\sin x}\dx \to \infty \text{ für } N\to\infty,
\end{align*}
also ist $\abs{\dfrac{\sin x}{x}}$ nicht uneigentlich Riemann-integrierbar über
$I$ also auch nicht Lebesgue-integrierbar.

Allgemeiner: $f(x) = \dfrac{\sin x}{x^\alpha}$.
\begin{align*}
\begin{cases}
\alpha\le 0, & f \text{ ist weder Riemann- noch Lebesgue-integrierbar},\\
0<\alpha \le 1, & f \text{ ist Riemann- aber nicht Lebesgue-integrierbar},\\
1<\alpha<2, & f \text{ ist sowohl Riemann- als auch Lebesgue-integrierbar},\\
\alpha \ge 2, & f \text{ ist weder Riemann- noch Lebesgue-integrierbar}.\bsphere 
\end{cases}
\end{align*}
\end{bsp}

\subsection{Produktmaße}

\begin{prop}[Definition/Satz]
\label{prop:3.47}
Seien $(\Omega,\Sigma)$ und $(\Omega',\Sigma')$ Maßräume, dann ist
\begin{align*}
h = \setdef{A\times B}{A\in\Sigma, B\in\Sigma'},
\end{align*}
ein Halbring und $\Sigma\otimes\Sigma':= \sigma(h)$ eine $\sigma$-Algebra,
die \emph{Produkt $\sigma$-Algebra}.

Für $M\in\Sigma\otimes\Sigma'$ und $a\in\Omega,b\in\Omega'$ seien die
Schnitte
\begin{align*}
&M_a = \setdef{y\in\Omega'}{(a,y)\in M}\subseteq\Omega',\\
&M^b = \setdef{x\in\Omega}{(x,b)\in M}\subseteq\Omega,
\end{align*}
dann gilt $M_a\in\Sigma'$ und $M^b\in\Sigma$.\fishhere
\end{prop}
\begin{proof}
\begin{enumerate}[label=(\roman{*})]
  \item\label{proof:3.47:1} $\MM:= \setdef{M\subseteq
  \Omega\times\Omega'}{M_a\in\Sigma', M^b\in\Sigma, a\in\Omega,b\in\Omega'}$ ist $\sigma$-Algebra.
  \item\label{proof:3.47:2} $h\subseteq \MM$: $A\times B\in h\Rightarrow
  (A\times B)_a =
  \begin{cases}
  B, &a\in A,\\
  \varnothing, & a\notin A,
  \end{cases}$
\end{enumerate}
\label{proof:3.47:1} und \label{proof:3.47:1} $\Rightarrow$
$\Sigma\otimes\Sigma' \subseteq \MM$.\qedhere
\end{proof}

\begin{corn}
Sei $f: \Omega\times\Omega'\to\RA$ messbar, dann sind auch die Schnitte
$f(x,\cdot) : \Omega'\to\RA$ und $f(\cdot,y): \Omega\to\RA$ für $x\in\Omega$,
$y\in\Omega'$ messbar.\fishhere
\end{corn}
\begin{proof}
Sei $C\subseteq\RA$ messbar, dann ist
\begin{align*}
(f(x,\cdot))^{-1}(C)
= \setdef{y\in\Omega'}{f(x,y)\in C} = f^{-1}(C)_x.\qedhere
\end{align*}
\end{proof}

\begin{prop}[Definition/Satz]
\label{prop:3.48}
Seien $(\Omega,\Sigma,\mu)$, $(\Omega',\Sigma',\mu')$ $\sigma$-endliche
Maßräume und $h$ wie in \ref{prop:3.47}. Dann ist $\rho(A\times B) :=
\mu(A)\mu'(B)$ mit $0\cdot\infty = 0$ ein $\sigma$-endliches Prämaß auf
$(\Omega\times\Omega',h)$, das \emph{Prä-Produktmaß}.\fishhere
\end{prop}
\begin{proof}
\begin{enumerate}[label=(\roman{*})]
  \item\label{proof:3.48:1} $\rho(\varnothing) = 0$,
  \item\label{proof:3.48:2} $\rho(A\times B) = \mu(A)\mu'(B) \ge 0$,
  \item\label{proof:3.48:3} $\sigma$-Additivität: $\rho(\dot{\bigcup}_{i=1}^n
  A_i\times B_i) = \sum\limits_{i=1}^n \rho(A_i\times B_i)$
  
  Sei $A\times B = \dot{\bigcup}_{n\in\N} A_n\times B_n$, dann gilt
  insbesondere für $k\neq n$, dass
\begin{align*}
(A_n\times B_n)\cap (A_k\times B_k) = \varnothing \Rightarrow (A_n\cap A_k =
\varnothing)\lor (B_n\cap B_k = \varnothing).
\end{align*}
\begin{align*}
\rho(A\times B) &= \mu(A)\mu'(B) = \int_\Omega \mu'(B)\chi_A \dmu
= \int_\Omega \mu'((A\times B)_x)\dmu(x)
\\ &= \int_\Omega \mu'\left(\dot{\bigcup}_{n\in\N}(A_n\times
B_n)_x\right)\dmu(x) \\ &= \int_\Omega \sum\limits_{n\in\N} \mu'((A_n\times B_n)_x)\dmu(x)
= \int_\Omega \sum\limits_{n\in\N} \underbrace{\mu'(B_n)\chi_{A_n}}_{\ge
0}\dmu
\\ &= \sum\limits_{n\in\N} \int_\Omega \mu'(B_n)\chi_{A_n} \dmu
= \sum\limits_{n\in\N} \mu(A_n)\mu'(B_n).
\end{align*}
\ref{proof:3.48:1}-\ref{proof:3.48:3} $\Rightarrow$ $\rho$ ist Prämaß.
\item $\mu,\mu'$ sind $\sigma$-endlich, es gilt daher
\begin{align*}
&\Omega = \bigcup_{m\in\N} A_m,\;\mu(A_m)<\infty,\\
&\Omega' = \bigcup_{n\in\N} B_n,\;\mu'(B_n)<\infty.
\end{align*}
Nun ist $\Omega\times\Omega' = \bigcup_{m,n\in\N} A_m\times B_n$ und
\begin{align*}
\rho(A_m\times B_n) = \mu(A_m)\mu'(B_n) < \infty,
\end{align*}
also ist $\rho$ ebenfalls $\sigma$-endlich.\qedhere
\end{enumerate}
\end{proof}

\begin{prop}[Definition/Satz]
\label{prop:3.49}
Seien $(\Omega_j,\Sigma_j,\mu_j)$, $\sigma$-endliche Maßräume für
$1\le j\le n$, dann existiert genau ein Maß auf
$\bigotimes_{j=1}^n \Sigma_j$, das \emph{Produktmaß} $\rho$, mit
$\rho(X_{j=1}^n A_j) = \prod_{j=1}^n \mu_j(A_j)$ für
$A_j\in\Sigma_j$.\fishhere 
\end{prop}
\begin{proof}
Der Beweis für zwei Maßräume wird analog zum Forsetzungssatz \ref{prop:3.9}
geführt, danach Induktion.\qedhere 
\end{proof}

\begin{prop}[Spezialfall]
\label{prop:3.50}
Seien $\Omega_j = \R$, $\Sigma_j = \BB(\R)$ die Borel $\sigma$-Algebra und
$\mu_j$ das Borel-Maß mit $\mu((a,b]) = \abs{b-a}$, dann ist $\mu^{(n)}:=
\bigotimes_{j=1}^n \mu_j$ das Lebesgue-Borel-Maß auf $\R^n$. Die
Vervollständigung $\lambda^{(n)}$ von $\mu^{(n)}$ ist das Lebesgue Maß auf
$\R^n$.

Man kann zeigen, dass $\bigotimes_{j=1}^n \BB(\R) =
\sigma\left(\left\{X_{j=1}^n (a_j,b_j] \right\} \right)$.\fishhere
\end{prop}

\begin{prop}
\label{prop:3.51}
Seien $(\Omega,\Sigma,\mu)$, $(\Omega',\Sigma',\mu')$ $\sigma$-endlich, dann
gilt
\begin{enumerate}[label=(\roman{*})]
  \item für $M\in\Sigma\otimes\Sigma'$ sind folgende Abbildungen,
\begin{align*}
&\ph_M : \Omega\to[0,\infty],\; x\mapsto \mu'(M_x),\\
&\ph_M' : \Omega'\to[0,\infty],\; y\mapsto \mu(M^y),
\end{align*}
messbar.
\item
Für die Abbildungen,
\begin{align*}
&\rho(M) := \int_\Omega \ph_M \dmu = \int_\Omega \mu'(M_x) \dmu(x),\\
&\rho'(M) := \int_{\Omega'} \ph_M' \dmu' = \int_{\Omega'} \mu(M^y) \dmu'(y),
\end{align*}
gilt $\rho = \rho' = \mu\otimes\mu'$.\fishhere
\end{enumerate}
\end{prop}
\begin{proof}
\begin{enumerate}[label=(\roman{*})]
  \item Setze $\MM = \setdef{M\in\Sigma\otimes\Sigma'}{\ph_M \text{ ist
  messbar}}\subseteq\Sigma\otimes\Sigma'$.

  
Sei $A\times B\in\Sigma\times\Sigma'$. Dann ist
\begin{align*}
\ph_{A\times B} : \Omega\to[0,1],\; x\mapsto \mu'(B)\chi_A(x),
\end{align*}
messbar nach \ref{prop:3.18} und daher ist $\Sigma\times\Sigma\subseteq\MM$.

Zeige: $\MM$ ist $\sigma$-Algebra, dann ist $\Sigma\otimes\Sigma' = \MM$.

\begin{enumerate}[label=\arabic{*}.)]
  \item $\varnothing\in \MM$, da $\ph_{\varnothing}$ trivialerweise messbar ist.
  \item $M\in \MM \Rightarrow M^c\in\MM$, denn $\Omega'$ ist $\sigma$-endlich,
  es existiert also eine Folge $\Omega_n'$ mit,
\begin{align*}
\Omega' = \bigcup_{n=1}^\infty \Omega_n',\quad \Omega_1' \subseteq
\Omega_2'\subseteq \ldots,\quad \mu'(\Omega_n')<\infty.
\end{align*}
\begin{align*}
\mu'(M_x^c) &:= \lim\limits_{n\to\infty} \mu'\left(\Omega_n'\cap M_x^c\right)
= \lim\limits_{n\to\infty} \mu'\left(\Omega_n' \setminus (M_x\cap
\Omega_n')\right)
\\ &= \lim\limits_{n\to\infty} \mu'\left(\Omega_n'\right) - \mu'(M_x\cap
\Omega_n') \\ &= \lim\limits_{n\to\infty} \mu'\left(\Omega_n'\right) -
\mu'((M\cap(\Omega\times\Omega_n'))_x),
\end{align*}
und alle Mengen sind messbar, also folgt mit \ref{prop:3.24}, dass $M_x^c$
messbar ist.
\item Sei $M=\dot{\bigcup}_{n\in\N} M_n,\; M_n\in\MM$. Es genügt hier disjunkte
Vereinigungen zu betrachten, also
\begin{align*}
\mu'(M_x) = \mu'\left(\dot{\bigcup}_{n\in\N} (M_n)_x\right)
= \sum\limits_{n\in\N} \mu'((M_n)_x),
\end{align*}
und $\mu'((M_n)_x)$ ist messbar.
\end{enumerate}
\item Zeige $\rho$ ist Maß auf $\Sigma\otimes\Sigma'$, $\rho(A\times B) =
\mu(A)\mu'(B)$, also ist $\rho$ Fortsetzung desselben Prämaßes wie
$\mu\otimes\mu'$ und da die Fortsetzung eindeutig ist, folgt $\rho =
\mu\otimes\mu'$.\qedhere
\end{enumerate}
\end{proof}

\begin{prop}[Satz von Fubini I]
\label{prop:3.52}
Seien $(\Omega,\Sigma,\mu)$, $(\Omega',\Sigma',\mu')$ $\sigma$-endlich,
$f:\Omega\times\Omega' \to [0,\infty]$ messbar, dann sind die Abbildungen
\begin{align*}
&f_1: \Omega\to[0,\infty],\; x\mapsto \int_{\Omega'} f(x,\cdot)\dmu',\\
&f_2: \Omega'\to[0,\infty],\; y\mapsto \int_\Omega f(\cdot,y)\dmu,
\end{align*}
messbar und es gilt,
\begin{align*}
\int_{\Omega\times\Omega'} f\dmumu &= \int_\Omega\left(
\int_{\Omega'} f(x,y) \dmu'(y)
\right) \dmu(x) \\ &= \int_{\Omega'} \left(\int_{\Omega} f(x,y) \dmu(x)
\right)\dmu'(y).\fishhere
\end{align*}
\end{prop}
\begin{proof}
Sei $M\in\Sigma\otimes\Sigma'$. Nach \ref{prop:3.51} ist die Abbildung,
\begin{align*}
(\chi_{M})_2: y\mapsto \mu(M^y) =
\int_\Omega \chi_{M^y}(x)\dmu(x) = \int_\Omega \chi_M(x,y)\dmu(x)
\end{align*}
messbar und es gilt,
\begin{align*}
\int_{\Omega\times\Omega'} \chi_M \dmumu
&= (\mu\otimes\mu')(M) \overset{\ref{prop:3.51}}{=} \rho(M)
= \int_{\Omega'} \mu(M^y) \dmu'(y)
\\ &= \int_{\Omega'} \left(
\int_\Omega \chi_M \dmu(x)
\right) \dmu'(y).
\end{align*}
Aufgrund der Linearität des Integrals gilt für jede positive einfache
Funktion $s$,
\begin{align*}
\int_{\Omega\times\Omega'} s(x,y)\dmumu
= \int_{\Omega'}\left(\int_\Omega s(x,y)\dmu(x) \right)\dmu'(y), 
\end{align*}
insbesondere ist $y\mapsto \int_\Omega s(x,y)\dmu(x)$ messbar.

\ref{prop:3.37} besagt, dass eine Folge $(s_n)$ positiver einfacher
Funktionen existiert mit $s_n\uparrow f$. Mit Hilfe des Satzes von der
monotonen Konvergenz, erhalten wir somit die Behauptung,
\begin{align*}
&\int_{\Omega\times\Omega'} f\dmumu = \lim\limits_{n\to\infty}
\int_{\Omega\times\Omega'} s_n \dmumu
\\ &= \lim\limits_{n\to\infty} \int_{\Omega'} \left(
\underbrace{\int_\Omega s_n(x,y) \dmu(x)}_{:=(s_n)_2 \text{ messbar,
monotonw.}} \right) \dmu'(y) \\ &= \int_{\Omega'} \left(
\underbrace{\int_\Omega f(x,y) \dmu(x)}_{:=f_2} \right) \dmu'(y),
\end{align*}
und $f_2$ ist Grenzwert von $(s_n)_2$ also messbar.\qedhere
\end{proof}

\begin{prop}[Satz von Fubini II]
\label{prop:3.53}
Seien $(\Omega,\Sigma,\mu)$, $(\Omega',\Sigma',\mu')$ $\sigma$-endlich und
$f:\Omega\times\Omega'\to\RA$ messbar, so gilt,
\begin{align*}
\int_{\Omega\times\Omega'} \abs{f}\dmumu &= 
\int_\Omega\left(\int_{\Omega'} \abs{f(x,y)} \dmu'(y) \right) \dmu(x)\tag{*}\\
&= \int_{\Omega'}\left(\int_{\Omega} \abs{f(x,y)} \dmu(x) \right) \dmu'(y).
\end{align*}
Ist eines der Integrale endlich, dann ist $f\in
L_1(\Omega\times\Omega',\Sigma\otimes\Sigma',\mu\otimes\mu')$ und es gilt,
\begin{align*}
&f(\cdot,y)\in L_1(\Omega,\Sigma,\mu),\quad \text{ für fast alle }
y\in\Omega',\tag{**}\\
&f(x,\cdot)\in L_1(\Omega',\Sigma',\mu'),\quad \text{ für fast alle }
x\in\Omega,
\end{align*}
sowie
\begin{align*}
\int_{\Omega\times\Omega'} f\dmumu &= \int_\Omega\left(
\int_{\Omega'} f(x,y) \dmu'(y) \right) \dmu(x)
\\ &= \int_{\Omega'} \left( \int_\Omega f(x,y) \dmu(x) 
\right) \dmu'(y).\fishhere
\end{align*}
\end{prop}
\begin{bemn}
Die iterierten Integrale sind eventuell nicht definiert, da z.B. $f_1(x) =
\int_{\Omega'} f(x,y)\dmu'(y)$ nur für fast alle $x\in\Omega$ definiert ist.

Zu Abhilfe sei $A:=\setdef{x\in\Omega}{\int_{\Omega'} \abs{f(x,y)} \dmu'(y) =
\infty}$, dann ist $A$ messbar mit $\mu(A) = 0$ (siehe Beweis). Nun kann man
$f_1$ auf $\Omega$ messbar fortsetzten zu $\tilde{f}_1$ durch,
\begin{align*}
\tilde{f}_1(x) = \begin{cases}
f_1(x), & x\in A^c,\\
0, & x\in A,
\end{cases}
\end{align*}
damit kann das Integral definiert werden durch
\begin{align*}
\int_\Omega f_1\dmu := \int_\Omega \tilde{f}_1\dmu.\maphere
\end{align*}
\end{bemn}
\begin{proof}
\begin{enumerate}[label=\arabic{*}.)]
  \item $f$ ist messbar, also ist auch $\abs{f}$ messbar und mit 
  Fubini I (\ref{prop:3.52}) folgt 
  die Gleichheit in (*). Außerdem sind $f_+$ und $f_-$ messbar und damit auch
\begin{align*}
f_{1,+}(\cdot,y), f_{2,+}(x,\cdot), f_{1,-}(\dot,y), f_{2,-}(x,\cdot). 
\end{align*}
%. Also ist auch $f_1(y):=f(\cdot,y) = f_+(\cdot,y) -
%f_-(\cdot,y)$ definiert und messbar, analog $f_2(x):=f(x,\cdot)$.
\item Sei nun $\int_{\Omega\times\Omega'} \abs{f}\dmumu<\infty$, dann ist
$f\in L_1(\Omega\times\Omega',\Sigma\otimes\Sigma',\mu\otimes\mu')$ und
\begin{align*}
g_1(x) := \int_{\Omega'} \abs{f(x,\cdot)} \dmu',
\end{align*}
ist ebenfalls $L_1$. Mit \ref{prop:3.30} folgt, dass
\begin{align*}
\mu\left(\setdef{x\in\Omega}{\abs{g_1(x)} = \infty}\right) = 0.
\end{align*}
Also sind $f(x,\cdot)$ und $f(y,\cdot)$ $\mufu$ integrierbar (**).

Der Rest folgt durch Anwendung von \ref{prop:3.52} auf $f_+$ und $f_-$.\qedhere 
\end{enumerate}
\end{proof}

\begin{prop}[Definition/Satz]
\label{prop:3.54}
Eine Menge $K$ heißt \emph{$x$-projezierbar}, falls es Funktionen $o(x)$ und
$u(x)$ gibt, sodass der Einschluss der Graphen gerade $K$ ist.

\begin{pspicture}(-1,-1)(4,3)
 %\psgrid

 \psaxes[labels=none,ticks=none]{->}%
 (0,0)(-0.5,-0.5)(3.5,2.5)%
 [\color{gdarkgray}$x$,-90]%
 [\color{gdarkgray}$y$,0]
 
 \psbezier[linecolor=yellow]%
	(0.5,0.5)(1,2.2)%
	(2.5,2.2)(3,0.5)

 \psbezier[linecolor=darkblue]%
	(0.5,0.5)(1.5,1.5)%
	(2.5,-0.8)(3,0.5)
	
 \rput(2.1,1){\color{gdarkgray}$K$}
 
 \rput[lb](2.6,1.6){\color{gdarkgray}$o(x)$}
 \rput[lt](1,0.5){\color{gdarkgray}$u(x)$}
 \rput(0.5,0){\color{gdarkgray}$[$}
 \rput(3,0){\color{gdarkgray}$]$}
 \rput[t](2.5,-0.1){\color{gdarkgray}$M$}
	
\end{pspicture}

Sei $(\Omega,\Sigma,\mu)$ $\sigma$-endlich, $\Omega'=\R$, $\Sigma'=\BB(\R)$,
$\mu'$ das Lebesgue-Borel-Maß auf $\R$ und $M\in\Sigma$.

Sind $u,o: M\to\R$ messbar, $u\le o$ auf $M$, so gilt,
\begin{align*}
K=\setdef{(x,y)\in\Omega\times\Omega'}{x\in M\land u(x)\le y\le o(x)}
\in\Sigma\otimes\Sigma',
\end{align*}
und für jede Abbildung $f:K\to\R$ ist $f(x,\cdot):[u(x),o(x)]\to\R$ messbar für
festes $x\in M$. Ist weiter
\begin{align*}
\int_M \left(\int\limits_{[u(x),o(x)]} \abs{f(x,y)} \dmu'(y) \right)\dmu(x)
<\infty,
\end{align*}
so gilt $f\in L_1(K,\Sigma\otimes\Sigma\cap K, \mu\otimes\mu'\big|_K)$ und
\begin{align*}
\int_K f\dmumu = \int_M \left( \int\limits_{[u(x),o(x)]} f(x,y) \dmu'(y)
 \right)\dmu(x).\fishhere
\end{align*}
\end{prop}

\begin{proof}
\begin{enumerate}
  \item Sei $\R=\dot{\bigcup}_{n\in\N} \left[a_n^{(k)},a_n^{(k)}+\frac{1}{k}
  \right)$, dann gilt $\sigma^{-1}\left(\left[a_n^{(k)},a_n^{(k)}+\frac{1}{k}
  \right) \right)\in\Sigma$ und
\begin{align*}
&\bigcap_{k=1}^\infty \bigcup_{n=1}^\infty
\underbrace{\sigma^{-1}\left(\left[a_n^{(k)},a_n^{(k)}+\frac{1}{k} \right)
\right)\times (-\infty,a_n^{(k)}+\frac{1}{k})}_{\in\Sigma\otimes\Sigma'} \\ &=
\setdef{(x,t)}{x\in M, t\le o(x)}
\end{align*}
\begin{pspicture}(-1,-1)(4,3)
 %\psgrid

 \psline[linestyle=none,fillstyle=solid,fillcolor=glightgray]%
 (0.6,-0.5)(0.6,1.45)(0.8,1.45)(0.8,-0.5)(0.6,-0.5)

 \psline[linestyle=none,fillstyle=solid,fillcolor=glightgray]%
 (1.9,-0.5)(1.9,1.45)(3,1.45)(3,-0.5)(1.9,-0.5)

 \psline[linecolor=gdarkgray]%
 (0,1.2)(3.5,1.2)

 \psline[linecolor=gdarkgray]%
 (0,1.45)(3.5,1.45)
 
 \psaxes[labels=none,ticks=none]{->}%
 (0,0)(-0.5,-0.5)(3.5,2.5)%
 [\color{gdarkgray}$x$,-90]%
 [\color{gdarkgray}$y$,0]
 
 \psbezier[linecolor=darkblue]%
	(0.5,1)(1.5,2.8)%
	(2.5,0.2)(3,1.5)

 \rput[r](-0.12,1.15){\color{gdarkgray}$a_n^{(k)}$}
 %\rput[rb](-0.1,1.45){\color{gdarkgray}$a_n^{(k)}$$\frac{1}{n}$}
 
 \rput(0.6,0){\color{gdarkgray}$[$}
 \rput(0.8,0){\color{gdarkgray}$)$}
 \rput(1.9,0){\color{gdarkgray}$($}
 \rput(3,0){\color{gdarkgray}$)$}
 
 \rput{90}(0,1.2){\color{gdarkgray}$[$}
 \rput{90}(0,1.45){\color{gdarkgray}$)$}
\end{pspicture}
\item Rest folgt aus Fubinit mit z.B.,
\begin{align*}
\int_K \abs{f}\dmumu &= \int_{\Omega\times\Omega'}\abs{f}\chi_K \dmumu
\\ &= \int_\R \left(\int_\R
\abs{f(x,t)}\underbrace{\chi_K(x,t)}_{=\chi_M(x)\chi_{[u(x),o(x)]}(t)}\dmu'(t)
\right)\dmu(x)\\ &
= \int_M \left(\int\limits_{[u(x),o(x)]} \abs{f(x,t)}\dmu'(t)
\right)\dmu(x).\qedhere
\end{align*}
\end{enumerate}
\end{proof}

\begin{bsp}
\label{bsp:3.55}
Kegel, $K:= \setdef{(x,y,z)\in\R^3}{0\le z\le 3\text{ und } x^2+y^2 \le z^2}$.

Sei $\mu$ das Lebesgue Maß im $\R^3$, wir betrachten $K$ als,
\begin{align*}
K = \setdef{(x,y,z)}{x^2+y^2\le 9\land \sqrt{x^2+y^2}\le z\le 3},
\end{align*}
dann ist
\begin{align*}
\int_K z\dmu^{(3)} = \int\limits_{x^2+y^2\le 9}
\left(\int\limits_{\sqrt{x^2+y^2}}^3 z\dmu^{(1)}(z) \right)\dmu^{(2)}(x,y),  
\end{align*}
da für stetige Funktionen auf kompakten Intervallen Lebesgue- und
Riemann-Integral übereinstimmen, ist
\begin{align*}
\ldots &= \frac{1}{2} \int\limits_{x^2+y^2\le 9} 9 - x^2-y^2 \dmu^{(2)}(x,y)
\\ &\overset{\ref{prop:3.54}}{=} \frac{1}{2} \int\limits_{-3}^3 \left( 
\int\limits_{-\sqrt{9-x^2}}^{\sqrt{9-x^2}} 9 - x^2-y^2 \dmu^{(1)}(y)
\right) \dmu^{(1)}(x)
\\ &= \frac{1}{2} \int\limits_{-3}^3
\left[(9-x^2)y-\frac{1}{3}y^3\right]_{-\sqrt{9-x^2}}^{\sqrt{9-x^2}}
= \frac{1}{2}\frac{4}{3} \int\limits_{-3}^3 (9-x^2)^{3/2} 
\dmu^{(1)}(x)
\end{align*}
Substitution: $x=3\sin \ph,\; \dx = 3\cos \ph\dph$,
\begin{align*}
&= \frac{2}{3} \int\limits_{-\pi/2}^{\pi/2} (9\cos^2 \ph)^{3/2} 3\cos\ph\dph
= \frac{2}{3} 3^4 \underbrace{\int\limits_{-\pi/2}^{\pi/2} \cos^4\ph
\dph}_{=\frac{3}{8}\pi}\\ &= \frac{81}{4}\pi <\infty.
\end{align*}
Das letzte Integral existiert also und ist endlich, wenn wir die Gleichung nun
``zurückgehen'' sehen wir, dass alle Integrale existieren und endlich sind.

Das berechnete Integral beschreibt im Übrigen das Trägheitsmoment des Kreisels
bei Rotation um die $z$-Achse.\bsphere
\end{bsp}

\begin{defn}
\label{defn:3.56}
Seien $U,V\subseteq\R$. Eine Abbildung $\phi: U\to V$ heißt
\emph{$C^k$-Diffeomorphimus} ($k\ge 1$), falls $\phi$ bijektiv und
$\phi,\phi^{-1}$ beide $C^k$ sind.\fishhere
\end{defn}

\begin{bem}
\label{bem:3.57}
Der Satz über Umkehrabbildungen besagt, dass für eine Abbildung $\phi\in
C^1(U\to V), x_0\in U$ mit $\det \D\phi(x_0) \neq 0$, eine Umgebung
$U(x_0)\subseteq U$ existiert, so dass
\begin{enumerate}[label=(\roman{*})]
  \item $\phi\big|_{U(x_0)}$ injektiv,
  \item $\phi(U(x_0))$ offen,
  \item $\phi^{-1}\in C^1(\phi(U(x_0))\to U(x_0))$,
  \item $\D\phi^{-1}(\phi(x_0)) =
(\D\phi(x_0))^{-1}$.\maphere
\end{enumerate}
\end{bem}

\begin{prop}
\label{prop:3.58}
Seien $U,V\subseteq\R^n$ offen, $\phi\in C^1(U\to V)$ bijektiv, dann sind
äquivalent,
\begin{enumerate}[label=(\roman{*})]
  \item $\phi$ ist $C^1$ Diffeomorphismus,
  \item $\det \D\phi\neq 0$ auf $U$.\fishhere
\end{enumerate}
\end{prop}

\begin{prop}[Transformationssatz]
\label{prop:3.59}
Seien $U,V\subseteq\R^n$ offen, $\phi: U\to V$ ein $C^1$-Diffeomorphismus,
$A\subseteq U$ messbar, dann gilt
\begin{align*}
\int_{\phi(A)} f\dmu^{(n)} = \int_A (f\circ \phi) \abs{\det D\phi} \dmu^{(n)},
\end{align*}
falls $f: \phi(A)\to[0,\infty]$ messbar oder $f\in L_1(\phi(A),\Sigma\cap
\phi(A),\mu\big|_{\phi(A)})$.\fishhere
\end{prop}

\begin{bsp}
\label{bsp:3.60}
Sei $U=(0,\infty)\times(0,2\pi), V=\R^2\setminus\setdef{(x,0)}{x\ge0}$.
\begin{align*}
&\phi:
\begin{pmatrix}
r\\\ph
\end{pmatrix}
\mapsto
\begin{pmatrix}
r\cos\ph\\
r\sin\ph
\end{pmatrix}\\
&\phi^{-1}:
\begin{pmatrix}
x\\y
\end{pmatrix}
\mapsto
\begin{pmatrix}
\sqrt{x^2+y^2}\\
\begin{cases}
\arccos\left(\frac{x}{\sqrt{x^2+y^2}} \right),& y\ge0,\\
-\arccos\left(\frac{x}{\sqrt{x^2+y^2}} \right)+2\pi,& y<0
\end{cases}
\end{pmatrix} 
\end{align*}
\begin{pspicture}(-6,-3)(6,3)
 %\psgrid

 \psline[linestyle=none,fillstyle=solid,fillcolor=glightgray]%
 (-3,0)(-3,1)(-1.1,1)(-1.1,0)(-3,0)

 \psline[linestyle=none,fillstyle=solid,fillcolor=glightgray]%
 (0.9,-1.9)(0.9,1.9)(4.9,1.9)(4.9,-1.9)(0.9,-1.9)
	
	
 \psaxes[labels=none,ticks=none]{->}%
 (-3,0)(-5,-2)(-1,2)%
 [\color{gdarkgray}$r$,-90]%
 [,-90]
 %[\color{gdarkgray}$\ph$,0]

 \psaxes[labels=none,ticks=none]{->}%
 (3,0)(1,-2)(5,2)%
 [\color{gdarkgray}$x$,-90]%
 [,-90]
 %[\color{gdarkgray}$y$,0]

 \psline[linestyle=dotted,linecolor=gdarkgray]%
 (-1,0)(-3,0)(-3,1)(-1,1)
	 
 \psbezier[linecolor=darkblue,arrows=->]%
	(-0.5,1.5)(-0.2,1.8)%
	(0.2,2)(0.5,1.5)

 \rput(-1.4,0.6){\color{gdarkgray}$U$}	
 \rput(4,1){\color{gdarkgray}$V$}
 
 \rput(-3.2,2.2){\color{gdarkgray}$\ph$}
 \rput(2.8,2.2){\color{gdarkgray}$y$}
 \rput(0,2){\color{gdarkgray}$\phi$}
\end{pspicture}
Die Funktionaldeterminante von $\phi$ ist,
\begin{align*}
\det \D\phi(r,\ph) = \begin{vmatrix}
\cos \ph & -r\sin\ph\\
\sin\ph & r\cos\ph
\end{vmatrix}
= r(\cos^2\ph + \sin^2\ph) = r >0,
\end{align*}
da $r\in(0,\infty)$. Damit ist $\phi$ regulär in $U$, also
$C^1$-Diffeomorphismus und es gilt,
\begin{align*}
\int_V f\dmu = \int_U f\circ\phi \abs{\det\D\phi}\dmu
= \int_U f\circ\phi\ r\dmu,
\end{align*}
falls $f$ geeignet. $V$ entspricht dem $\R^2$ ohne die positive reelle
Halbachse $\R_+$, das Integral soll jedoch über den ganzen $\R^2$ berechnet
werden. Jedoch ist $\R_+$ als Teilmenge des $\R^2$ eine Nullmenge, denn
\begin{align*}
&\mu^{(2)}\left(\setdef{(x,0)}{x\ge 0}\right) =
\mu^{(2)}\left([0,\infty)\times\{0\}\right) = \mu^{(1)}([0,\infty))
\mu^{(1)}(\{0\}) = 0,\\
&\mu^{(2)}\left(\setdef{(x,2\pi)}{x\ge 0}\right) = 0.
\end{align*}
Daher gilt für das Integral,
\begin{align*}
\int_{\R^2} f\dmu &= \int\limits_{[0,\infty)\times[0,2\pi]} f\circ\phi
\dmu(r,\ph) \\ &\overset{\text{Fubini *}}{=}
\int\limits_{r=0}^\infty\int\limits_{\ph=0}^{2\pi} f(r\cos\ph,r\sin\ph) r\dph
\dr.
\end{align*}
* ist anwendbar, falls $f$ messbar und positiv oder $f\in L_1$.

Wir wollen nun damit erneut das Trägheitsmoment des Kegels aus Beispiel
\ref{bsp:3.55} berechnen,
\begin{align*}
\int_K z\dmu^{(3)} &= \frac{1}{2} \int_{x^2+y^2\le 9} 9-(x^2+y^2)\dmu^{(3)}(x,y)\\
& = \frac{1}{2}\int\limits_{r=0}^3 \left(
\int\limits_{\ph=0}^{2\pi} (9-r^2) r\dph \right) \dr  
= \pi \left[\frac{9}{2}r^2 - \frac{1}{3}r^3\right]_{0}^3 \\
&= \pi \left(\frac{3^4}{2} - \frac{3^4}{4} \right)
=\frac{3^4}{4}\pi.
\end{align*}
Wir sehen also, dass die Koordinatentransformation oft schneller und eleganter
ans Ziel führt.
\end{bsp}

\begin{bem}
\label{bem:3.61}
\begin{propn}[Satz von Sard]
Sei $\phi\in C^1(U\to\R^n)$, $K:=\setdef{x\in U}{\D\phi(x)=0}$. Dann ist
$\mu(\phi(K)) = 0$.\fishhere
\end{propn}
Dadurch können wir den Transformationssatz verallgemeinern, es muss also nur
vorausgesetzt werden, dass $\phi\in C^1(U\to V)$ und
$\phi\big|_{U\setminus K}$ injektiv.\maphere
\end{bem}

\subsection{$\LL^p$-Räume}
\begin{bem}[Vorbemerkung.]
\label{bem:3.62}
Ist $f:(\Omega,\Sigma,\mu)\to\C$, so definieren wir,
\begin{align*}
\int_\Omega f\dmu = \int_\Omega \Re f\dmu + i\int_\Omega \Im f\dmu.
\end{align*}
Sätze, die die Anordnung von $\R$ nicht benötigen, also insbesondere die Sätze
von Lebesgue und Fubini, gelten auch für diese Funktionen.

Wir schreiben nun $\KH$, wenn sowohl $\RA$ also auch $\C$ möglich ist.\maphere 
\end{bem}

\begin{defn}
\label{defn:3.63}
Sei $f:(\Omega,\Sigma,\mu)\to\KH$ messbar.
\begin{enumerate}[label=\arabic{*}.)]
  \item Für $1\le p<\infty$, sei
\begin{align*}
N_p(f) := \left(\int_\Omega \abs{f}^p \dmu\right)^{1/p},
\end{align*}
denn $\abs{f}^p = (\cdot)^p\circ\abs{f}$ ist messbar\footnote{Wir setzen
$\infty^p=\infty$}.
\item Für $p=\infty$, heißt
\begin{align*}
N_\infty(f) := \inf\setdef{c\in[0,\infty]}{\abs{f}\le c \mufu}
:= \esssup_\Omega(f),
\end{align*}
\emph{wesentliches Supremum von $f$}.\fishhere
\end{enumerate}
\end{defn}

\begin{cor}
\label{prop:3.64}
\begin{enumerate}[label=\arabic{*}.)]
  \item Für $1\le p\le\infty$ gilt,
  \begin{enumerate}[label=(\roman{*})]
    \item $0\le N_p(f) \le \infty$,
    \item $N_p(\alpha f) = \abs{\alpha}N_p(f)$ für $\alpha\in\C$.
  \end{enumerate}
\item $\abs{f}\le N_\infty(f) \mufu$.
\item $N_\infty(f+g) \le N_\infty(f)+N_\infty(g)$.\fishhere
  \end{enumerate}
\end{cor}

\begin{proof}
\begin{enumerate}[label=\arabic{*}.)]
  \item Offensichtlich.
  \item per definitionem gilt $\abs{f}\le N_\infty+\ep \mufu, \forall\ep > 0$,
  also ist
\begin{align*}
&\mu\left(\setdef{\omega\in\Omega}{\abs{f(\omega)}> N_\infty(f)}\right) \\
&= \mu\left(\bigcup_{n\in\N}
\underbrace{\setdef{\omega\in\Omega}%
{\abs{f(\omega)}> N_\infty(f)+\frac{1}{n}}}_{\mu(\cdot) = 0, \text{ nach
Definition von } N_\infty}\right) \\ &= \lim\limits_{n\to\infty} \mu\left(\setdef{\omega\in\Omega}{\abs{f(\omega)}> N_\infty(f)+\frac{1}{n}}\right)
= 0.
\end{align*}
\item $\abs{f+g}(\omega) \le\abs{f}(\omega) + \abs{g}(\omega) \le N_\infty(f) +
N_\infty(g) \mufu$.\qedhere
\end{enumerate}
\end{proof}

\begin{defn}
\label{defn:3.65}
$p,q\in[1,\infty]$ heißen \emph{konjugiert}, falls sie eine dieser
Bedingungen erfüllen,
\begin{enumerate}[label=(\roman{*})]
  \item $p,q\neq \infty$ und $\frac{1}{p}+\frac{1}{q} = 1$,
  \item $p=1, q=\infty$ oder $p=\infty, q= 1$.\fishhere
\end{enumerate}
\end{defn}
Wir werden sehen, dass $p=q=2$ ein wichtiger Spezialfall ist.
\begin{prop}
\label{prop:3.66}
Seien $f,g:(\Omega,\Sigma,\mu)\to\KH$ messbar.
\begin{enumerate}[label=\arabic{*}.)]
  \item Für $1<p, q<\infty$ und $p,q$ konjugiert, gilt die \emph{Höldersche
  Ungleichung},
\begin{align*}
\underbrace{\int_\Omega \abs{fg}\dmu}_{=N_1(fg)} \le
\underbrace{\left(\int_\Omega
\abs{f}^p \right)^{1/p} \left(\int_\Omega \abs{g}^q
\right)^{1/q}}_{=N_p(f)N_q(g)}.
\end{align*}
\item Für $1\le p<\infty$ gilt die \emph{Minkowski Ungleichung},
\begin{align*}
\underbrace{\left(\int_\Omega \abs{f+g}^p\right)^{1/p}}_{=N_p(f+g)}
\le \underbrace{\left(\int_\Omega \abs{f}^p\right)^{1/p} + \left(\int_\Omega
\abs{g}^p\right)^{1/p}}_{N_p(f)+N_p(g)}.\fishhere
\end{align*}
\end{enumerate}
\end{prop}

\begin{proof}
\begin{enumerate}[label=\arabic{*}.)]
\item\begin{enumerate}[label=(\alph{*})]
\item Wir betrachten zunächst die Sonderfälle.\\
Ist $N_p(f) = 0$, dann ist
$f=0 \mufu$ also auch $fg=0 \mufu$ und damit folgt $N_1(fg) = 0$.

Ist $N_p(f) > 0$ und $N_q(g) = \infty$, dann folgt die Behauptung
trivialerweise.
\item Sei nun $0<N_p(f),N_q(g)<\infty$.

Wir wollen uns auf den Spezialfall,
\begin{align*}
N_p(f) = 1\text{ und }N_q(g)=1 \Rightarrow
\int_\Omega \abs{fg}\dmu \le 1,\tag{*}
\end{align*}
zurückziehen, denn für beliebige $f,g$ ist,
\begin{align*}
N_p\left(\frac{f}{N_p(f)} \right) = 1,\; N_q\left(\frac{g}{N_q(g)} \right) =1,
\end{align*}
und damit folgt,
\begin{align*}
1 \ge \int_\Omega \abs{\frac{f}{N_p(f)}\frac{g}{N_q(g)}} \dmu =
\frac{1}{N_p(f)}\frac{1}{N_q(g)} \int_\Omega \abs{fg}\dmu.
\end{align*}
\item Hilfsungleichung: 

Seien $0<x,y<\infty$, dann existieren $u,v\in\R$, so dass
\begin{align*}
x=e^{\frac{1}{p} u},\;y=e^{\frac{1}{q} v}.
\end{align*}
Die Abbildung $t\mapsto e^t$ ist konvex, setze nun 
$\lambda=\frac{1}{p}$, dann ist $\frac{1}{q}=1-\lambda$ und daher gilt,
\begin{align*}
\underbrace{e^{\frac{1}{p}u+\frac{1}{q}v}}_{=x\cdot y}
= e^{\lambda u + (1-\lambda) v}
 \le \lambda e^u + (1-\lambda) e^v = 
\underbrace{\frac{1}{p}e^{u}+\frac{1}{q}e^{v}}_{=\frac{1}{p}x^p +
\frac{1}{q}y^q}.
\end{align*}
Somit ist $x\cdot y \le \frac{1}{p}x^p +
\frac{1}{q}y^q$ für $x,y\in(0,\infty)$. Offensichtlich gilt die Gleichung 
sogar für $x,y\in[0,\infty]$.
\item Wir zeigen nun den Spezialfall, indem wir über die Hilfsungleichung
integrieren, sei also $N_p(f)=N_q(g)=1$,
\begin{align*}
\int_\Omega \abs{f\cdot g} \le \frac{1}{p}\int_\Omega \abs{f}^p\dmu +
\frac{1}{q}\int_\Omega \abs{g}^q \dmu = \frac{1}{p} + \frac{1}{q} = 1.
\end{align*}
\end{enumerate}
\item Für $p=1$ ist dies gerade die Dreiecksungleichung für das Integral. Sei
nun $1<p<\infty$, dann gilt ohne Einschränkung  $N_p(f+g)>0$ und
$N_p(f),\;N_p(g)<\infty$. Es gilt also
\begin{align*}
\abs{f+g}^p & \le \left(2\max\left\{ \abs{f},\abs{g}\right\}\right)^p
= 2^p \max\left\{ \abs{f}^p,\abs{g}^p\right\}
\\ &\le 2^p\left(\abs{f}^p + \abs{g}^p\right),
\end{align*}
also ist $N_p(f+g)<\infty$.
Es folgt daher,
\begin{align*}
N_p(f+g)^p &= \int_\Omega \abs{f+g}^p \dmu
= \int_\Omega \abs{f+g}\abs{f+g}^{p-1}\dmu
\\ &\le \int_\Omega \abs{f}\abs{f+g}^{p-1}\dmu + \int_\Omega
\abs{g}\abs{f+g}^{p-1}\dmu
 \\ &\le N_p(f)N_q\left(\abs{f+g}^{p-1}\right) +
 N_p(g)N_q\left(\abs{f+g}^{p-1}\right)
 \\ &= \left(N_p(f)+N_p(g) \right)\left(\int_\Omega
 \abs{f+g}^{q(p-1)}\right)^{1/q}.
\end{align*}
$p$ und $q$ sind assoziert, es gilt daher $\frac{p}{q}=p-1$,
\begin{align*}
&= \left(N_p(f)+N_p(g) \right)\left(\left(\int_\Omega
 \abs{f+g}^{p}\right)^{\frac{1}{p}}\right)^{p/q}
 \\ &= \left(N_p(f)+N_p(g) \right) N_p(f+g)^{\frac{p}{q}}.
\end{align*}
Nun ist $N_p(f+g)< \infty$ also gilt auch
\begin{align*}
N_p(f+g)^{p-\frac{p}{q}} \le N_p(f)+N_p(g),
\end{align*}
und $p-\frac{p}{q}=1$.\qedhere
\end{enumerate}
\end{proof}


\begin{cor}
\label{prop:3.67}
\begin{enumerate}[label=\arabic{*}.)]
  \item Für $1\le p,q\le \infty$ konjugiert, gilt
\begin{align*}
N_1(f\cdot g) \le N_p(f)N_q(g).
\end{align*}
  \item Für $1\le p\le\infty$ gilt, 
\begin{align*}
N_p(f+g) \le N_p(f) + N_p(g).\fishhere
\end{align*}
\end{enumerate}
\end{cor}

\begin{proof}
\begin{enumerate}[label=\arabic{*}.)]
  \item Wir müssen noch den Fall $p=1, q=\infty$ betrachten, doch hier ist
\begin{align*}
N_1(f\cdot g) = \int_\Omega \underbrace{\abs{f\cdot g}}_{\le\norm{f}_\infty
\abs{g}\mufu}\dmu \le \norm{f}_\infty \int_\Omega \abs{g}\dmu.
\end{align*}
\item Für $p=\infty$ siehe \ref{prop:3.64}.\qedhere
\end{enumerate}
\end{proof}

\begin{cor}
\label{prop:3.68}
Sei $1\le p\le \infty$, dann ist
\begin{align*}
L^p(\Omega,\Sigma,\mu) := \setdef{f:\Omega\to\KH}{f\text{ ist messbar und
}N_p(f)<\infty},
\end{align*}
ein Vektorraum über $\R$ oder $\C$. $N_p$ ist eine Seminorm auf
$L^p(\Omega,\Sigma,\mu)$.\fishhere
\end{cor}

Streng genommen ist $L^p$ mit der gewöhnlichen Addition kein Vektorraum, denn
es sind auch auch Funktionen zugelassen, so dass auf einer Nullmenge gilt
$f(\omega) = \pm \infty$ und $g(\omega)=\mp\infty$ und dann ist $f+g$ nur
$\mufu$ definiert.

Darüber hinaus ist $L^p$ aber auch kein normierter Vektorraum, denn $N_p(f)
= 0$ impliziert lediglich, dass $f=0\mufu$ und nicht, dass $f$ die Nullfunktion ist. Wir können
dieses Problem aber mithilfe des Faktorraums lösen.

\begin{defn}
\label{defn:3.69}
Sei $1\le p\le\infty$, dann ist
\begin{align*}
N := \setdef{f: \Omega\to\KH}{N_p(f) =0} = \setdef{f:\Omega\to\KH}{f=0\mufu},
\end{align*}
ein Unterraum von $L^p$.
\begin{align*}
f\sim g \Leftrightarrow f-g\in N,
\end{align*}
definiert eine Äquivalenzrelation auf $L^p$. Seien $\nrm{f}$ die
Äquivalenzklassen mit Vertreter $f$, dann ist
\begin{align*}
\LL^p(\Omega,\Sigma,\mu) = \setdef{[f]}{f\in L^p} = L^p/N,
\end{align*}
ein Vektorraum und $\norm{[f]}_p := N_p(f)$ eine Norm für
$[f]\in\LL^p$.\fishhere
\end{defn}

Bei Elementen in $\LL^p$ handelt es sich also um Vertreter einer
Äquivalenzklasse. Aussagen die für die Äquivalenzklasse $[f]$ gelten, gelten
daher für einen Vertreter $f$ stets nur $\mufu$. Außerdem muss man
beispielsweise bei Widerspruchsargumenten darauf achten, den Widerspruch für die ganze
Äquivalenzklasse herbeizuführen und nicht nur für einen einzigen Vertreter.

Wir wollen nun die Äquivalenzklassen untersuchen. Dazu schreiben wir im Folgenden $\LL^p(\Omega)$ bzw. $\LL^p$ für
$\LL^p(\Omega,\Sigma,\mu)$ und meinen mit $f$ stets die Äquivalenzklasse $[f]$.

\begin{bemn}
Jede Klasse $[f]$ enthält ein Element $\tilde{f}$ mit $\abs{\tilde{f}} <\infty$
auf ganz $\Omega$.\maphere
\end{bemn}
Besonders sind wir daran interessiert, ob jede Äquivalenzklasse ein
stetiges Element enthält und wenn ja, ob dieses eindeutig ist. Die
Eindeutigkeit ist wie des Öfteren eine leichte Angelegenheit.

\begin{prop}
\label{prop:3.70}
Sei $\Omega\subseteq\R^n$ offen und $f,g\in C(\Omega)\cap\LL^p(\Omega)$ und
$\norm{f-g}_p=0$, dann ist $f=g$ auf ganz $\Omega$. Insbesondere enthält jede
Äquivalenzklasse höchstens ein stetiges Element.\fishhere
\end{prop}

\begin{proof}
Seien $f,g$ wie vorausgesetzt und $\norm{f-g}_p=0$, dann ist $f=g\mufu$.
Angenommen es gibt ein $\omega\in\Omega$ mit $f(\omega) \neq g(\omega)$, dann
gibt es aufgrund der Stetigkeit von $f$ und $g$ auch eine Umgebung
$U\subseteq\R^n$ von $\omega$ auf der $f$ und $g$ verschieden sind. Aber $U$
ist offen in $\R^n$ und nichtleer und daher gilt $\mu(U)\neq 0$, ein Widerspruch.\qedhere
\end{proof}
Für die Existenz müssen wir jedoch bestimmte Voraussetzung an die
Äquivalenzklasse treffen. Dies überschreitet jedoch den Rahmen der Vorlesung,
weshalb wir den folgenden Satz nicht beweisen werden.

\begin{prop}[Sobolevsche Einbettung]
\label{prop:3.71}
Sei $f: \R^n\to\R$ mindestens $\frac{n}{2}$-mal ``schwach differenzierbar'' und
alle  Ableitungen in $\LL^p(\R^n)$, dann ist $f$ stetig, d.h. $[f]$ enthält ein
stetiges Element.\fishhere
\end{prop}

\begin{prop}
\label{prop:3.72}
\begin{enumerate}[label=\arabic{*}.)]
  \item Seien $\mu(\Omega)<\infty$, $1\le p\le p'\le\infty$ und $f\in
  \LL^{p'}$, dann ist $f\in\LL^p(\Omega)$ und es gilt $\norm{f}_p\le
  \mu(\Omega)^{\frac{1}{p}-\frac{1}{p'}}\norm{f}_{p'}$.
  \item Seien $1\le p< p'\le \infty$, dann ist
\begin{align*}
&\LL^p(\Omega)\setminus \LL^{p'}(\Omega) \neq \varnothing,\\
&\LL^{p'}(\Omega)\setminus \LL^p(\Omega)\neq\varnothing.
\end{align*}
\item Seien $\Omega = \N, \Sigma=\PP(\N), \mu(M) = \card M$ und $1\le p\le p'\le
\infty$, dann folgt
\begin{align*}
l^p = \LL^p(\N)\subseteq l^{p'}.\fishhere
\end{align*}
\end{enumerate}
\end{prop}

\begin{proof}
Der Beweis sei als Übungsaufgabe überlassen.\qedhere
\end{proof}

\begin{prop}[Satz von Fischer-Riesz]
\label{prop:3.73}
Für $1\le p\le \infty$ ist $\LL^p(\Omega)$ ein Banachraum.\fishhere
\end{prop}
\begin{proof}
\begin{enumerate}[label=\arabic{*}.)]
  \item Sei $p=\infty$ und $(f_n)$ Cauchyfolge in $\LL^p(\Omega)$. Setze
\begin{align*}
&A_n :=\setdef{\omega\in\Omega}{\abs{f_n(\omega)} > \norm{f_n}_\infty},\\
&B_{k,l} :=
\setdef{\omega\in\Omega}{\abs{f_k(\omega)-f_l(\omega)}>\norm{f_k-f_l}_\infty},
\end{align*}
dann ist klar, dass $\mu(A_n) = \mu(B_{k,l}) = 0$. Ebenso ist,
\begin{align*}
E := \bigcup_{n\in\N} A_n \cup \bigcup_{k,l\in\N} B_{k,l},
\end{align*}
als Vereinigung abzählbar vieler Nullmengen eine Nullmenge.

Für $\omega\in\Omega\setminus E$ gilt $\abs{f_k(\omega)-f_l(\omega)} \le
\norm{f_k-f_l}_\infty\to 0$ für $k,l\to \infty$, d.h. $f_k(\omega)\to
f(\omega)$ gleichmäßig auf $\Omega\setminus E$. Setze
\begin{align*}
f(\omega) := \begin{cases}
\lim\limits_{n\to\infty} f_n(\omega),& \omega\in\Omega\setminus E,\\
0,& \omega\in E,
\end{cases}
\end{align*}
dann ist $f = \lim\limits_{n\to\infty} f_n(\omega)\mufu$ also messbar und
es gilt,
\begin{align*}
\abs{f_n(\omega)-f(\omega)} = \lim\limits_{m\to\infty} \abs{f_n(\omega) -
f_m(\omega)} \le \lim\limits_{m\to\infty} \norm{f_n - f_m} < \ep,
\end{align*}
für $n\ge N_\ep$ und $\omega\in\Omega\setminus E$. Damit ist 
$\norm{f_n-f}<\ep$ und $f-f_n \in \LL^\infty$.

Insbesondere ist $f = \underbrace{f-f_n}_{\in\LL^\infty} + f_n \in \LL^\infty$.
\item Sei $1\le p < \infty$ und $(f_n)$ Cauchyfolge in $\LL^p$.

Wähle $n_1$ mit $\norm{f_{n_1}-f_m}_p < \frac{1}{2}$ für $m>n_1$, $n_2> n_1$
mit $\norm{f_{n_2}-f_m}_p < \frac{1}{4}$ für $m>n_2$ usw., dann ist
\begin{align*}
\norm{f_{n_k}-f_{n_{k+1}}} \le \frac{1}{2^{k}}.
\end{align*}
\begin{enumerate}[label=(\alph{*})]
  \item Grenzfunktion $f$ konstruieren:
\begin{align*}
&g_n(\omega) := \sum\limits_{k=1}^n f_{n_k}(\omega) -
f_{n_{k+1}}(\omega),\\
&g(\omega) := \sum\limits_{k=1}^\infty
f_{n_k}(\omega)-f_{n_{k+1}}(\omega).
\end{align*}
Dann ist $\norm{g_n}_p \le \sum\limits_{k=1}^n \norm{f_{n_k} -
f_{n_{k+1}}}_p \le \sum\limits_{k=1}^n \frac{1}{2^n}< 1$ und es gilt,
\begin{align*}
0\le g_n \uparrow g \Rightarrow 0\le g_n^p \uparrow g^p,
\end{align*}
und $g^p$ ist messbar. Wir können also den Satz der monotonen Konvergenz
anwenden und erhalten,
\begin{align*}
\norm{g}_p^p = \int_\Omega \abs{g}^p \dmu =  \lim\limits_{n\to\infty}
\int_\Omega \abs{g_n}^p \dmu =  \lim\limits_{n\to\infty} \norm{g_n}_p^p \le 1.
\end{align*}
$g_n$ konvergiert daher $\mufu$ absolut und daher konvergiert auch
$(f_{n_1}-f_{n_k})_k$ gegen eine messbare Funktion und es gilt,
\begin{align*}
f_{n_{l+1}} = f_{n_1} + \sum\limits_{k=1}^l (f_{n_{k+1}}-f_{n_k}),
\end{align*}
$\mufu$. Sei $f = \lim\limits_{l\to\infty} f_{n_l}$. 
\item Zeige $f\in\LL^p$ und $\norm{f_n-f}_p \to 0$.

Sei $\ep > 0,\;N\in\N$ mit $\norm{f_{n} - f_{m}}_p < \ep$ für $n,m >N$,
dann besagt das Lemma von Fatou,
\begin{align*}
\norm{f_n-f}_p^p &= \int_\Omega \abs{f_n -f}^p \dmu
 = \int_\Omega \lim\limits_{l\to\infty} \abs{f_n-f_{n_l}}^p\dmu
\\ &\le \liminf_{l\to\infty} \int_\Omega \abs{f_n-f_{n_l}}^p\dmu
=  \liminf_{l\to\infty} \norm{f_n-f_{n_l}}_p^p  \\
 &\le \ep^p.
\end{align*}
Also ist $f-f_n\in\LL^p$ und damit folgt $f = \underbrace{f-f_n}_{\in\LL^p} +
\underbrace{f_n}_{\in\LL^p}\in\LL^p$.\qedhere
\end{enumerate}
\end{enumerate}
\end{proof}

\begin{cor}[Satz von Weyl]
\label{prop:3.74}
Sei $1\le p\le\infty$ und $(f_n)$ Cauchyfolge in $\LL^p$. Dann existiert ein
$f\in\LL^p$ und eine Teilfolge $(f_{n_k})$ so, dass
\begin{align*}
\norm{f_n-f}\to 0\text{ und } f_{n_k} \unito f\mufu.\fishhere
\end{align*}
\end{cor}
\begin{proof}
Ergibt sich aus der Konstruktion im Satz von Fischer-Riesz.\qedhere
\end{proof}

\begin{prop}
\label{prop:3.75}
Sei $1\le p<\infty$.
\begin{enumerate}[label=\arabic{*}.)]
  \item Für eine einfache Funktion $s$ gilt,
\begin{align*}
s\in \LL^p(\Omega) \Leftrightarrow
\mu\left(\setdef{\omega\in\Omega}{s(\omega)\neq 0}\right) <\infty.
\end{align*}
\item $\setdef{s\in\LL^p(\Omega)}{s\text{ ist einfach}}$ liegt dicht in
$\LL^p$.\fishhere
\end{enumerate}
\end{prop}
\begin{proof}
\begin{enumerate}[label=\arabic{*}.]
  \item Übung.
  \item Sei zunächst $f\in\LL^p(\Omega)$ und $f\ge 0$. Dann existiert eine
  Folge $(s_n)$ einfacher Funktionen mit $0\le s_1\le s_2\le \ldots$ und $s_n\to
  f$. $f$ majorisiert $s_n$ und $f-s_n$, es gilt
  daher $s_n\in \LL^p$ sowie,
\begin{align*}
\lim\limits_{n\to\infty} \norm{f-s_n}_p^p = \lim\limits_{n\to\infty} \int
= \int_\Omega \lim\limits_{n\to\infty} \abs{f-s_n}\dmu = 0.
\abs{f-s_n}^p\dmu
\end{align*}
Für allgemeines $f$ betrachte $f_+$ und $f_-$.
\end{enumerate}
\end{proof}

\begin{prop}[Ausblick]
\label{prop:3.76}
Sei $1\le p<\infty$. Dann liegt
\begin{align*}
C_c^\infty(\R^n) := \setdef{f\in C^\infty(\R^n\to\C)}{\supp f\text{ ist
kompakt}},
\end{align*}
dicht in $\LL^p(\R^n)$.\fishhere
\end{prop}

\begin{bem}
\label{bem:3.77}
$\LL^2(\Omega)$ ist Hilbertraum mit dem Skalarprodukt,
\begin{align*}
\lin{f,g} = \int_\Omega f\overline{g}\dmu,
\end{align*}
d.h. vollständig bezüglich der durch das Skalarprodukt induzierten Norm,
\begin{align*}
\norm{f} := \sqrt{\lin{f,f}} = \norm{f}_2.
\end{align*}
Dabei erfüllt die Höldersche Ungleichung,
\begin{align*}
\abs{\lin{f,g}} \le \int_\Omega \abs{f\overline{g}}\dmu \le
\norm{f}_2\norm{g}_2,
\end{align*}
die Cauchy-Schwartzsche-Ungleichung.\maphere
\end{bem}