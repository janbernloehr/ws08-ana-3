\section{Grenzwerte}

\subsection{Grundlagen}

\begin{defn}
\label{def:1.1}
Eine \emph{Folge} in einer Menge $M$ ist eine Abbildung der Form
\begin{align*}
a : \N \to M,
\end{align*}
die man durch Aufzählen ihrer Funktionswerte als $(a_n)_{n\in\N}$
angibt.\fishhere
\end{defn}
\begin{bemn}[Bemerkung zur Notation]
Wir wollen eine Folge als Ganzes mit $(a_n)_{n\in\N}$ bezeichnen, um
sie von dem \emph{n-ten Folgenglied} $a_n$ zu unterscheiden.\maphere
\end{bemn}

An einer Folge interessiert uns vor allem ihr asymptotisches Verhalten. Um
dieses beschreiben zu können, benötigen wir einen Abstandsbegriff.

\begin{defn}
\label{def:1.2}
Sei $M$ eine Menge, dann heißt eine Abbildung $d : M\times M\to\R$
\emph{Metrik}, falls sie den folgenden Eigenschaften genügt:
\begin{enumerate}
  \item $\forall x,y\in M: d(x,y)\ge 0$ und $d(x,y) = 0 \Leftrightarrow x =y$.
  \item $\forall x,y\in M: d(x,y) = d(y,x)$.
  \item $\forall x,y,z\in M: d(x,z) \le d(x,y) + d(y,z)$.\fishhere
\end{enumerate}
\end{defn}

Metriken sind ein allgemeineres Konzept als Normen, es gilt folgender
Zusammenhang.

\begin{propn}
Sei $(V,\norm{\cdot})$ ein normierter Vektorraum, dann wird durch
\begin{align*}
d(x,y) = \norm{x-y},\quad x,y\in V,
\end{align*}
eine Metrik induziert.\fishhere
\end{propn}
\begin{proof}
Die positive Definitheit sowie die Symmetrie sind klar. Um die
Dreiecksungleichung zu zeigen bedienen wir uns eines beliebten Tricks,
\begin{align*}
d(x,y) &= \norm{x-y} = \norm{x-z+z-y} \le \norm{x-z}+\norm{z-y} \\ 
&= d(x,z) + d(z,y).
\end{align*}
Womit alle Metrikeigenschaften nachgewiesen sind.\qedhere
\end{proof}

\begin{bsp}
\label{bsp:1.3}
\begin{enumerate}
  \item $M=\R,\; d(x,y) = \abs{x-y}$.\\
  Wie durch jede Norm, wird auch durch die Betragsnorm eine Metrik induziert.  
  \item $M=\C^n,\; d(x,y) := \norm{x-y}_p := \left(\sum_{j=1}^n \abs{x_j-y_j}^p
  \right)^{1/p}$.\\
  Selbiges gilt für die $p$-Norm.
  \item Für eine Menge $M$ ist die diskrete Metrik definiert als,
  \begin{align*}
 d(x,y)= \begin{cases}0, & x=y,\\1, & x\neq y.\bsphere\end{cases}
  \end{align*}
\end{enumerate}
\end{bsp}

\begin{defn}
\label{def:1.4}
Sei $(M,d)$ ein metrischer Raum, $(a_n)$ eine Folge in $M$.
\begin{enumerate}
  \item $(a_n)$ heißt \emph{konvergent gegen $a\in M$} bezüglich $d$, falls gilt
\begin{align*}
\forall\ep>0\exists N \in\N\forall n\ge N : d(a_n,a) < \ep.
\end{align*}
Schreibe $a_n\to a$ für $n\to\infty$ oder kurz $\lim\limits_{n\to\infty} a_n =
a$. 
  \item $(a_n)$ heißt \emph{konvergent}, falls ein solches $a$ existiert.
  \item $(a_n)$ heißt \emph{Cauchyfolge}, falls
\begin{align*}
\forall \ep > 0 \exists N\in\N \forall n,m\ge N: d(a_n,a_m) < \ep.
\end{align*}
  \item  $(M,d)$ heißt \emph{vollständig}, falls jede Cauchyfolge
konvergent ist.\fishhere
\end{enumerate}
\end{defn}

Erst der folgende Satz erlaubt es uns überhaupt von einem Grenzwert zu sprechen.

\begin{propn}
Existiert der Grenzwert einer Folge, so ist er eindeutig bestimmt.\fishhere
\end{propn}
\begin{proof}
Beweis siehe Skript Prof. Pöschel Seite 61f, man muss lediglich die Norm als
Metrik interpretieren.\qedhere
\end{proof}

Wir wissen bereits, dass nicht jeder metrische Raum volltändig ist und nur in
vollständigen Räumen folgt aus der Cauchyeigenschaft die Konvergenz. Die
Umkehrung gilt aber immer.

\begin{propn}
Jede konvergente Folge ist eine Cauchyfolge.\fishhere
\end{propn}
\begin{proof}
Beweis siehe Skript Prof. Pöschel Seite 72.\qedhere
\end{proof}

\begin{bsp}
\label{bsp:1.5}
\begin{enumerate}
\item Der $\R^n$ ist mit jeder durch eine Norm induzierten Metrik
vollständig.
\item Der Raum $C([a,b],\R)$ mit der Metrik
\begin{align*}
d(f,g) = \norm{f-g}_2 := \left( \int\limits_a^b \abs{f(x)-g(x)}^2 \dx
\right)^{1/2},
\end{align*}
ist nicht vollständig.
\begin{proof}
Um dies zu zeigen, genügt es eine Cauchyfolge in $C([a,b],\R)$ zu finden, die
nicht in $C([a,b],\R)$ konvergiert. Dazu fixieren wir $a=0,\;b=2$ und betrachten
die Funktionenfolge
\begin{align*}
f_n(x) =\begin{cases}x^n, & 0\le x\le 1,\\ 1, & 1<x\le 2.\end{cases}
\end{align*}
$f_n$ ist Cauchy, denn
\begin{align*}
\norm{f_n-f_m}_2^2 &= \int\limits_0^1 \abs{x^n-x^m}^2\dx = \int\limits_0^1
\abs{x^{2n}-2x^{n+m}+x^{2m}}\dx
\\ &= \frac{1}{2n+1} - \frac{2}{n+m+1} + \frac{1}{2m+1} \le \frac{1}{2n+1}+
\frac{1}{2m+1} \\& \le \frac{1}{n}+\frac{1}{m} < \ep \text{ für } n,m >
\frac{2}{\ep}.
\end{align*}
Die Grenzfunktion dieser Funktionenfolge ist nicht stetig und damit nicht in
$C([a,b],\R)$, denn angenommen $\exists f\in C([0,2],\R) : d(f,f_n) \to 0$,
dann gilt
\begin{align*}
\Rightarrow \begin{cases} \int\limits_0^1 \abs{f_n(x)-f(x)}^2 \dx \to 0
\Rightarrow f(x) = 0 & \text{für } 0\le x\le 1,\\
\int\limits_1^2 \abs{f_n(x)-f(x)}^2 \dx \to 0
\Rightarrow f(x) = 1 & \text{für } 1< x\le 2.\\
\end{cases}
\end{align*}
Dies impliziert
\begin{align*}
\int\limits_0^1 \abs{f_n(x)-f(x)}^2 \dx = \int\limits_0^1 x^{2n}
\dx = \frac{1}{2n+1} \to 0,
\end{align*}
also kann $f$ nicht stetig sein.\qedhere\bsphere
\end{proof}
\end{enumerate}
\end{bsp}

\begin{defn}[Definition/Satz]\label{def:1.6} Sei $(a_n)$ eine Folge in $(M,d)$.
\begin{enumerate}
  \item Ist $(n_k)$ eine streng monoton wachsende Folge in $\N$, so heißt $(a_{n_k})$
\emph{Teilfolge} von $(a_n)$.
  \item Sei $a\in M$ ein \emph{Häufungspunkt} von $(a_n)$, dann sind folgende Aussagen
äquivalent
\begin{enumerate}
  \item Es existiert eine Teilfolge $(a_{n_k})\to a$.
  \item $\forall\ep>0\forall N\in\N\exists m\ge N: d(a_m,a) < \ep$.\fishhere
\end{enumerate}
\end{enumerate}
\end{defn}
\begin{proof}
Der Beweis ist eine leichte Übung.\qedhere
\end{proof}

\begin{defnn}
Die \emph{Menge der Häufungspunkte} einer Folge $(a_n)$ bezeichnen wir mit
\begin{align*}
HP\left((a_n)\right) = \setdef{a\in M}{a \text{ ist Häufungspunkt von }
(a_n)}.\fishhere
\end{align*}
\end{defnn}

\begin{bsp}
\label{bsp:1.7}
\begin{enumerate}
\item $a_n = (-1)^n \dfrac{n^2-n}{2n^2+1} \Rightarrow HP\left((a_n)\right) =
\{1/2,-1/2\}$.

\item $a_n = 2e^{in \pi/6}$.
\begin{center}
\psset{unit=0.5cm}
\begin{pspicture}(-4.5,-4.5)(4.5,4.5)
% \psline[linecolor=framecolor](-1,-1)(-1,6.5)(6.5,6.5)(6.5,-1)(-1,-1)
 
 \psaxes[labels=none,ticks=none]{->}%
 (0,0)(-4,-4)(4,4)[\color{gdarkgray}$\Re$,-90][\color{gdarkgray}$\Im$,0]
 
\pscircle[linestyle=dotted](0,0){3}
 
 \psxTick(3){\color{gdarkgray}$1$}
 \psyTick(3){\color{gdarkgray}$1$}
 
 \psdots[dotstyle=o,fillstyle=solid,%
		   fillcolor=darkblue,%
	       linecolor=darkblue](3,0)(2.6,1.5)(1.5,2.6)(0,3)(-1.5,2.6)(-2.6,1.5)
	       (-3,0)(-2.6,-1.5)(-1.5,-2.6)(0,-3)(1.5,-2.6)(2.6,-1.5)
 
\end{pspicture}
\end{center}
Die 12 Häufungspunkte liegen auf einem Kreis mit
Radius 2 um den Ursprung und haben den Winkelabstand $\pi/6$.
\item Für eine Abzählung $(a_n)$ von $\Q$ in $\R$ gilt $HP\left((a_n) \right) =
\R$.\bsphere
\end{enumerate}
\end{bsp}

\begin{defn}
\label{defn:1.8}
Sei $(a_n)$ eine Folge in $(\R,\abs{\cdot})$, dann sind \emph{limes
superior} und \emph{inferior} wie folgt definiert.
\begin{enumerate}
  \item Ist $(a_n)$ nach oben beschränkt und $\HP{(a_n)}\neq\varnothing$, so
  ist
  \begin{align*}
  \limsup a_n = \sup \HP{(a_n)}.
  \end{align*}
  \item Ist $(a_n)$ nach oben beschränkt und $\HP{(a_n)}=\varnothing$, so
  ist $\limsup a_n = -\infty$.
  \item Ist $(a_n)$ nach oben unbeschränkt, so ist $\limsup a_n =
  \infty$.
\end{enumerate}
Analoges gilt für den $\liminf a_n$. \fishhere 
\end{defn}

\begin{bsp}
\label{bsp:1.9}
\begin{enumerate}
\item $a_n = (-1)^n \dfrac{n^2-n}{2n^2+1} \Rightarrow \limsup a_n = 1/2,\;
\liminf a_n = -1/2$.
\item $a_n = n \Rightarrow \limsup a_n = \liminf a_n = \infty$.
\item $a_n = \begin{cases}(1/2)^n & \text{für $n$ ungerade},\\ 2^n & \text{für
$n$ gerade}.\end{cases}$

$\liminf a_n = 0, \limsup a_n = \infty$.\bsphere
\end{enumerate}
\end{bsp}

\begin{prop}
\label{prop:1.10}
\label{limsupsatz}
Sei $(a_n)$ reell, nach oben beschränkt und
$\HP{(a_n)}\neq\varnothing$, dann gilt
\begin{align*}
\forall \ep>0 \exists N_\ep \in\N \forall n\ge N_\ep : a_n \le \limsup a_n +
\ep.\fishhere
\end{align*}
\end{prop}
\begin{proof}
Angenommen, die Aussage gilt nicht, also
\begin{align*}
\exists \ep > 0 \forall N \in\N\exists n\ge N : a_n > \limsup a_n + \ep,
\end{align*}
dann existiert eine Teilfolge $(a_{n_k})$ mit $a_{n_k} > \limsup a_n + \ep$.
Da $(a_n)$ beschränkt ist, existiert eine konvergente Teilfolge $(a_{n_k}')$ von
$(a_{n_k})$ mit Grenzwert
\begin{align*}
a:=\lim_{k\to\infty} a_{n_k}' \ge \limsup a_n + \ep.
\end{align*}
Dann ist aber $a\in\HP{(a_n)}$ und $a\ge \sup\HP{(a_n)} + \ep$, ein
Widerspruch.\qedhere
\end{proof}

\begin{prop}
\label{prop:1.11}
Sei $(a_n)$ Folge in $(\R,\abs{\cdot})$. Dann ist $(a_n)$ konvergent genau
dann, wenn $(a_n)$ beschränkt ist und gilt $\limsup a_n = \liminf a_n$.\fishhere
\end{prop}
\begin{proof}
``$\Rightarrow$'': $(a_n)$ ist konvergent, also beschränkt und es
gilt $\HP{(a_n)} = \{\lim a_n\}$. Daher ist $\limsup a_n = \liminf a_n = \lim
a_n$.

``$\Leftarrow$'': Sei $a:= \limsup\limits_{n\to\infty} a_n =
\liminf\limits_{n\to\infty} a_n$. Da $(a_n)$ beschränkt ist, ist
$\HP{(a_n)}\neq\varnothing$. Sei nun $\ep > 0 $ beliebig aber fest, dann
existieren nach \ref{limsupsatz} $N,N'\in\N$ so, dass gilt:
\begin{align*}
&\forall n\ge N: a_n \le \limsup a_n + \ep,\\
&\forall n\ge N': a_n \ge \liminf a_n - \ep.
\end{align*}
Für alle $n \ge N_0 = \max\{N,N'\}$ gilt daher $a - \ep \le a_n \le a + \ep$,
also $a_n\to a$.\qedhere
\end{proof}

Funktionenfolgen spielen in vielen Gebieten der Analysis eine Rolle. Hier
können ganz unterschiedliche Arten von Konvergenz auftreten. Die
Konvergenz einer Funktionenfolge kann davon abhängen, welcher Punkt aus dem
Definitionsbereich gerade betrachtet wird, weshalb man beispielsweise in
Funktionenräumen ein allgemeineres Konzept als Metriken benötigt, um diese
Konvergenz zu erfassen. Mithilfe der Konvergenz lässt sich auch oft
feststellen, wie sich Eigenschaften wie Stetigkeit oder Differenzierbarkeit
aller Folgenglieder auf die Grenzfunktion vererben.

\begin{defn}
\label{defn:1.12}
Sei $D$ Menge, $(M,d)$
metrischer Raum, $(f_n)$ eine Folge von Abbildungen $f_n: D\to M$ und $f: D\to M$.
\begin{enumerate}
  \item $(f_n)$ heißt \emph{punktweise konvergent} auf $D$ gegen $f$, falls
  \begin{align*}
  \forall x\in D \forall \ep > 0 \exists N\in \N \forall n\ge N :
  d(f_n(x), f(x)) < \ep.
  \end{align*}
  \item $(f_n)$ heißt \emph{gleichmäßig konvergent} auf $D$ gegen $f$, falls
  \begin{align*}
  \forall \ep > 0 \exists N\in\N \forall n\ge N, x\in D  : d(f_n(x), f(x)) <
  \ep.\fishhere
  \end{align*}
\end{enumerate}
\end{defn}

Nicht überraschend ist folgender Zusammenhang.

\begin{prop}
\label{prop:1.13}
Konvergiert eine Funktionenfolge $(f_n)$ gleichmäßig gegen eine Grenzfunktion
$f$, so konvergiert sie auch punktweise.\fishhere
\end{prop}
\begin{proof}
Der Beweis sei als Übung überlassen.\qedhere
\end{proof}

Im Folgenden wollen wir durch $f_n\unito f$ ausdrücken, dass die Folge $(f_n)$
gleichmäßig gegen $f$ konvergiert.

\begin{bsp}
\label{bsp:1.13}
\begin{enumerate}
\item $f_n(x) = x + \frac{1}{n}\sin x$ konvergiert gleichmäßig gegen
$f(x) = x$, denn
\begin{align*}
\abs{f_n(x)-f(x)} = \frac{1}{n}\abs{\sin x} \le \frac{1}{n} < \ep \text{ für }
n > \frac{1}{\ep}.
\end{align*}
\item Auf $D=[0,2]$ konvergiert die Funktionenfolge
\begin{align*}
f_n(x) = \begin{cases} x^n
& 0\le x\le 1,\\ 1 & 1< x \le 2,\end{cases}
\end{align*}
lediglich punktweise gegen die Grenzfunktion
\begin{align*}
f(x) = \begin{cases} 0 & 0\le x \le 1 \\ 1 & 1 < x \le 2\end{cases}.
\end{align*}
Wie wir noch sehen werden, kann die Konvergenz
nicht gleichmäßig gewesen sein, da $f$ nicht stetig ist.\bsphere
\end{enumerate}
\end{bsp}

\subsection{Vertauschung von Grenzwerten}

\begin{bsp}
\label{bsp:1.15}
\begin{enumerate}
\item Für die Doppelfolge $(a_{np})_{n,p\in \N}$ mit $a_{np} = \dfrac{n}{n+p+1}
+ \dfrac{1}{n} + \dfrac{2}{p}$ gilt
\begin{align*}
&\lim\limits_{n\to\infty}\lim\limits_{p\to\infty} a_{np} =
\lim\limits_{n\to\infty} \frac{1}{n} = 0,\\
&\lim\limits_{p\to\infty}\lim\limits_{n\to\infty} a_{np} =
\lim\limits_{p\to\infty} \left(1+\frac{2}{p}\right) = 1.
\end{align*}
Offensichtlich hängt der Grenzwert von der Reihenfolge ab, in der die Indizes
nach Unendlich laufen.\bsphere
\end{enumerate}
\end{bsp}

\begin{prop}
\label{prop:1.16}
Sei $(M,d)$ ein vollständiger metrischer Raum, $a : \N\times\N \to M$ so, dass
gilt
\begin{align*}
&\lim\limits_{n\to\infty} a_{np} = u(p) \quad \text{ für } p\in\N,\\
&\lim\limits_{p\to\infty} a_{np} = v(n) \quad \text{ für } n\in\N,
\end{align*}
Gilt $a_{np}\twoheadrightarrow u(p)$ bzw. $a_{np}\twoheadrightarrow v(n)$, dann
existieren die Grenzwerte $\lim\limits_{p\to\infty} u(p)$ und
$\lim\limits_{n\to\infty} v(n)$ und stimmen überein.\fishhere
\end{prop}

\begin{proof}
ObdA können wir annehmen $a_{np}\twoheadrightarrow u(p)$, dann gilt
\begin{align*}
d(u(p),u(q)) \le d(u(p),a_{np}) + d(a_{np},a_{nq}) + d(a_{nq},u(q)).
\end{align*}
Für $n_0$ hinreichend groß folgt aus der gleichmäßigen Konvergenz von $a_{np}$
\begin{align*}
d(u(p),u(q)) < \frac{2\ep}{3} + d(a_{n_0p},a_{n_0q}) < \ep,
\end{align*}
für $p,q > P_\ep$. Somit ist $u(p)$ Cauchyfolge, also konvergent mit Grenzwert
$u$.

Es bleibt noch zu zeigen, dass $v(n)$ ebenfalls den Grenzwert $u$ hat.
\begin{align*}
d(v(n),u) &\le d(v(n),a_{np}) + d(a_{np},u(p)) + d(u(p),u)\\
&< d(v(n),a_{np}) + \frac{\ep}{3} + d(u(p),u),
\end{align*}
für $n> N_\ep$. Wählt man nun $n_0 > N_\ep$ beliebig aber fest, folgt daher
\begin{align*}
d(v(n_0),u) < \frac{\ep}{3} + \underbrace{d(v(n_0),a_{n_0p})}_{< \frac{\ep}{3}
\text{ für } p > P_{\ep,n_0}} + \underbrace{d(u(p),u)}_{< \frac{\ep}{3} \text{ für } 
p > P_\ep'} < \ep,
\end{align*}
für $p > \max\{P_{\ep,n_0}, P_\ep'\}$.

Da $n_0>N_\ep$ beliebig war, gilt
für alle $n>N_\ep$ und $p > \max\{P_{\ep,n_0}, P_\ep'\}$, dass $d(v(n),u) <
\ep$.\qedhere
\end{proof}

\begin{bem}[Festlegung]
\label{bem:1.17}
Im Folgenden seien $(X,d_X)$ und $(Y,d_Y)$ metrische Räume.\maphere
\end{bem}

\begin{defn}
\label{defn:1.18}
Sei $D\subseteq X$, dann heißt $x_0\in X$ \emph{Häufungspunkt} von $D$, falls
\begin{align*}
\forall \ep > 0 \exists y\in D: (y\neq x_0) \land (d(y,x_0)< \ep),
\end{align*}
oder äquivalent
\begin{align*}
\exists (x_n) \text{ in } D : (x_n \to x_0) \land (\forall n\in \N : x_n \neq
x_0).
\end{align*}
Die Menge aller Häufungspunkte von $D$ bezeichnen wir mit $D'$, den Abschluss
von $D$ mit $\overline{D} = D\cup D'$.\fishhere
\end{defn}

\begin{bsp}
\label{bsp:1.19}
\begin{enumerate}
\item Sei $D = \N$, dann ist $D' = \varnothing$ und  $\overline{D} = \N$.
\item Sei $D = \setdef{\frac{1}{n}}{n\in\N}$, dann ist $D' = \{0\}$ und 
$\overline{D} = D\cup\{0\}$.
\item Sei $D = \Q$, dann ist $D' =\R$ und $\overline{D} = \R$.
\end{enumerate}
Die Menge der Häufungspunkte kann also leer, ``größer'' oder ``kleiner'' als die Menge
selbst sein.\bsphere
\end{bsp}

\begin{defn}
\label{defn:1.20}
Sei $D\subseteq X, f: D\to Y$
\begin{enumerate}
  \item $f$ heißt \emph{stetig} in $x_0\in D$, falls
  \begin{align*}
  \forall \ep > 0 \exists \delta > 0 \forall x\in D : d_X(x,x_0) < \delta 
  \Rightarrow d_Y(f(x_0), f(x)) < \ep.
  \end{align*}
  Äquivalent lässt sich Stetigkeit mit einem Folgenkriterium definieren. $f$
  ist stetig in $x_0$, falls für alle Folgen $(x_n)$ in $D$ mit $x_n \to x_0$
  und $x_n\neq x_0$ gilt $f(x_n)\to f(x_0)$.
  \item $f$ heißt \emph{stetig auf $D$}, wenn $f$ in jedem Punkt aus $D$ stetig
  ist.
  \item Der Raum der stetigen Funktionen $f: D\to Y$ wird mit \emph{$C(D\to Y)$}
  bezeichnet.
  \item Sei $\xi\in D'$, dann schreiben wir $\lim\limits_{x\to\xi} f(x) = \eta$
  bzw. $f(x)\to \eta$ für $x\to \xi$, falls
  \begin{align*}
  \forall \ep > 0 \exists \delta > 0 \forall x\in D : d_X(x,\xi) < \delta
  \Rightarrow d_Y(f(x),\eta) < \ep,
  \end{align*}
  oder äquivalent dazu, falls für jede Folge $(x_n)$ in $D$ gilt
  \begin{align*}
  x_n\to\xi \Rightarrow f(x_n)\to\eta.\fishhere
  \end{align*}
\end{enumerate}
\end{defn}

\begin{prop}
\label{prop:1.21}
Sei $Y$ vollständig, $D\subseteq X$, $f_n,f: D\to Y$. Gilt $f_n\unito f$ auf
$D$ und sind alle $f_n$ stetig auf $D$, so ist auch $f$ stetig auf $D$.\fishhere
\end{prop}
\begin{proof}
Sei $x_0\in D$, $(x_p)$ Folge in $D$ mit $x_p\to x_0$. Setze $a_{np} :=
f_n(x_p)$. Dann gilt $a_{np} \unito f(x_p)$ bezüglich $p$ und
$\lim\limits_{p\to\infty} a_{np} = f_n(x_0)$. Nun folgt mit \ref{prop:1.16}
\begin{align*}
\lim\limits_{n\to\infty} \underbrace{\lim\limits_{p\to\infty}
f_n(x_p)}_{f_n(x_0)} =
\lim\limits_{p\to\infty}\underbrace{\lim\limits_{n\to\infty} f_n(x_p)}_{f(x_p)},
\end{align*}
also gilt $f(x_0) = \lim\limits_{p\to\infty} f(x_p)$ und $f$ ist stetig.\qedhere
\end{proof}
Ein analoges Ergebnis erhalten wir für Häufungspunkte:
\begin{prop}
\label{prop:1.22}
Sei $Y$ vollständig, $D\subseteq X$, $\xi\in D'$ und $f_n,f: D\to Y$. Gilt
$f_n\unito f$ auf $D$ und gilt für $n\in\N : f_n(x) \to a(n)$ für $x\to\xi$,
dann existiert $\lim\limits_{x\to\xi} f(x)$ und es gilt
\begin{align*}
\lim\limits_{x\to\xi} f(x) = \lim\limits_{n\to\infty} a(n).\fishhere
\end{align*}
\end{prop}
\begin{proof}
Sei $(x_p)$ Folge in $D$, $x_p\to\xi$. Setze $a_{np}:=f_n(x_p)$, dann folgt
\begin{align*}
&a_{np}\unito f(x_p),\quad n\to \infty,\\
&a_{np} \to a(n),\quad p\to\infty.
\end{align*}
Mit \ref{prop:1.16} erhalten wir,
\begin{align*}
&\lim\limits_{n,p\to\infty} a_{np} = \lim\limits_{p,n\to \infty} a_{np},\\
&\lim\limits_{p\to\infty} f(x_p) = \lim\limits_{n\to\infty} a(n).
\end{align*}
Da $(x_p)$ beliebig war, folgt $\lim\limits_{x\to\xi} f(x) =
\lim\limits_{n\to\infty} a(n)$.\qedhere
\end{proof}

\begin{bem}
\label{bem:1.23}
Die selben Sätze gelten für Reihen, man muss sie nur als Folge von
Partialsummen lesen.\\
Sei z.B. $f_n(x)\in C(D\to Y)$, $Y$ Bannachraum und $\sum\limits_{n=1}^\infty
f_n(x)$\footnote{Achtung: Mit $\sum\limits_{n=1}^\infty a_n$ kann sowohl
Reihe selbst, als auch ihr Grenzwert gemeint sein.} gleichmäßig konvergent auf $D$, dann folgt auch
$\sum\limits_{n=1}^\infty f_n(x) \in C(D\to Y)$.\maphere
\end{bem}

\begin{bsp}
\label{bsp:1.24}
Sei $(a_n)$ Folge in $\C$, $\sum\limits_{n=1}^\infty a_n$ absolut konvergent,
dann ist
\begin{align*}
f(x) := \sum\limits_{n=1}^\infty a_n\sin(nx),
\end{align*}
gleichmäßig konvergent auf $\R$, also $f\in C(\R\to \C)$.
\begin{proof}
\begin{align*}
\abs{f(x) -  \sum\limits_{n=1}^N a_n\sin(nx)} = \abs{\sum\limits_{n=N+1}^\infty
a_n\sin(nx)} \le \sum\limits_{n=N+1}^\infty \abs{a_n} < \ep,
\end{align*}
für $N\ge N_\ep$, also konvergiert die Reihe gleichmäßig.\qedhere
\end{proof}
Die wenigsten Funktionenfolgen sind auf ganz $\R$ gleichmäßig konvergent. Hier
ist die Beschränktheit des Sinus ausschlaggebend.\bsphere
\end{bsp}

\begin{prop}
\label{prop:1.25}
Sei $(Y,\norm{\cdot})$ Banachraum, $f_n\in C([a,b]\to Y)$, $f_n\unito f$ auf
$[a,b]$. Dann gilt
\begin{align*}
\lim\limits_{n\to\infty} \int\limits_a^b f_n(x) \dx = \int\limits_a^b
\lim\limits_{n\to\infty} f_n(x) \dx.\fishhere
\end{align*}
\end{prop}
\begin{proof}
Mit \ref{prop:1.21} folgt, dass $f\in C([a,b],Y)$, also gilt
\begin{align*}
&\norm{\int\limits_a^b f(x)\dx - \int\limits_a^b f_n(x)\dx} =
\norm{\int\limits_a^b f(x)-f_n(x) \dx} \\ & \le \int\limits_a^b
\norm{f(x)-f_n(x)} \dx \le \ep \int\limits_a^b \dx = \ep(b-a), 
\end{align*}
für $n>N_\ep$.\qedhere
\end{proof}

\begin{prop}
\label{prop:1.26}
Sei $f_n\in C^1([a,b]\to Y)$, $(Y,\norm{\cdot})$ Bannachraum. Existieren
$x_0\in[a,b]$ und $\ph\in C([a,b]\to Y)$ so, dass $(f_n(x_0))$ konvergent ist
und gilt $f_n'\unito \ph$ auf $[a,b]$, dann folgt:
\begin{enumerate}
  \item $f_n$ konvergiert gleichmäßig auf $[a,b]$.
  \item Sei $f(x) := \lim\limits_{n\to \infty} f_n(x)$. Dann ist $f\in
  C^1([a,b]\to Y)$ und $f' = \ph$ also $\left(\lim\limits_{n\to \infty}
  f_n(x)\right)' = \lim\limits_{n\to \infty} f_n'(x)$.\fishhere
\end{enumerate}
\end{prop}
\begin{proof}
$f_n'$ ist stetig, also gilt nach dem Hauptsatz $f_n(x) = f_n(x_0) +
\int\limits_{x_0}^x f_n'(t)\dt$.
\begin{align*}
&\norm{f_n(x)-f_m(x)} = \norm{\int\limits_{x_0}^x f_n'(t) - f_m'(t) \dt + 
f_n(x_0) - f_m(x_0)} \\
&\le \abs{\int\limits_{x_0}^x \norm{f_n'(t) - f_m'(t)}\dt} + \norm{f_n(x_0) -
f_m(x_0)} \\
&\le \abs{\int\limits_{x_0}^x \norm{f_n'(t) - \ph(t)} + \norm{\ph(t)-
f_m'(t)}\dt} + \ep \\
&\le 2\ep\abs{x-x_0} + \ep \le 2\ep\abs{b-a} + \ep,
\end{align*}
für $n\ge N_\ep$ und unabhängig von $x$. Somit ist $(f_n)$ Cauchyfolge also
konvergent. Da die obige Abschätzung auch für $m\to\infty$ gilt, ist $(f_n)$
sogar gleichmäßig konvergent, es gilt also $\norm{f_n(x) - f(x)} < \ep$
unabhängig von $x$.

Es ist $f_n(x) = f_n(x_0) + \int\limits_{x_0}^x f_n'(t)\dt$ und alle Summanden
konvergieren. Durch Grenzübergang erhalten wir nach \ref{prop:1.25}
\begin{align*}
f(x) = f(x_0) + \int\limits_{x_0}^x \ph(t)\dt,
\end{align*}
und damit folgt $f' = \ph\in C([a,b]\to Y)$.\qedhere
\end{proof}

\begin{prop}[Definition/Satz]
\label{prop:1.27}
Sei $(a_n)$ eine Folge in $\C$ und
\begin{align*}
R := \begin{cases}
     \infty, & \text{ falls } \limsup a_n = 0,\\
     0, & \text{ falls } \limsup a_n = \infty,\\
     \frac{1}{\limsup\limits_{n\to\infty}
\sqrt[n]{\norm{a_n}}}, & \text{ sonst}.
     \end{cases}
\end{align*}
Dann gilt für die \emph{Potenzreihe}
\begin{align*}
f(z) = \sum\limits_{n=1}^\infty a_n (z-z_0)^n,
\end{align*}
\begin{enumerate}[label=(\roman{*})]
  \item\label{prop:1.16:1} $f(z)$ konvergiert absolut, falls $\abs{z-z_0} < R$,
  \item\label{prop:1.16:2} $f(z)$ divergiert, falls $\abs{z-z_0} > R$,
  \item\label{prop:1.16:3} $f(z)$ konvergiert gleichmäßig auf der Menge
  $\setdef{z\in\C}{\abs{z-z_0} \le R'}$ für jedes $R'\in (0,R)$,
  \item\label{prop:1.16:4} $f(z)$ ist stetig auf $\setdef{z}{\abs{z-z_0} < R}$.
\end{enumerate}
$R$ nennt man den \emph{Konvergenzradius} der Potenzreihe.\fishhere
\end{prop}
\begin{proof}
Sei $\limsup\limits_{n\to\infty}
\sqrt[n]{\norm{a_n}} \in (0,\infty)$
\begin{enumerate}[label=(\roman{*})]
  \item Mit Satz \ref{prop:1.10} folgt für $z\in\C$ mit $\abs{z-z_0}<R$,
  \begin{align*}
  \sqrt[n]{\abs{a_n(z-z_0)^n}} &= \sqrt[n]{\abs{a_n}}\abs{z-z_0} \le \left(
  \underbrace{\limsup \sqrt[n]{\abs{a_n}}}_{=1/R} + \ep
  \right)\underbrace{\abs{z-z_0}}_{< R} \\ &=
  \underbrace{\frac{\abs{z-z_0}}{R}}_{<1} + \ep\abs{z-z_0} < 1,
  \end{align*}
für $n > N_\ep$, wenn $\ep$ hinreichend klein. Also konvergiert
$\sum\limits_{n=1}^\infty a_n(z-z_0)^n$ absolut.
  \item Sei $z\in\C$ mit $\abs{z-z_0}>R$ und $(a_{n_k})$ Teilfolge mit
  $\sqrt[n_k]{\abs{a_{n_k}}} \to \limsup \sqrt[n]{\abs{a_n}}$, dann folgt
  \begin{align*}
  \abs{a_{n_k}(z-z_0)^{n_k}} &=
  \left(\sqrt[n_k]{\abs{a_{n_k}}}\abs{z-z_0}\right)^{n_k} \\
  &>   \abs{\left(
  \limsup \sqrt[n]{\abs{a_n}} - \ep\right)\abs{z-z_0}}^{n_k} \\ &= \abs{\frac{\abs{z-z_0}}{R} - \ep\abs{z-z_0}}^{n_k} > 1,
  \end{align*}
für $\ep$ hinreichend klein. Damit divergiert die Potenzreihe.
  \item Aus \ref{prop:1.16:1} folgt $\sum\abs{a_n} {R'}^n$ ist konvergent. 
  Für $z\in\C$ mit $\abs{z-z_0} \le R'$ folgt damit für $N>N_\ep$,
  \begin{align*}
  \sum\limits_{n=N}^\infty \abs{a_n(z-z_0)^n} \le \sum\limits_{n=N}^\infty
  \abs{a_n}{R'}^n < \ep.
  \end{align*}
  \item 
  Sei $z\in\C$ mit $\abs{z-z_0}<R$, und $R'\in(\abs{z-z_0},R)$ beliebig aber
  fest, dann folgt aus \ref{prop:1.21} die Stetigkeit von $f$ auf
  \begin{align*}
  \setdef{\zeta\in\C}{\abs{\zeta-z_0} \le R'},
  \end{align*}
  also insbesondere in $z$.\qedhere
\end{enumerate}
\end{proof}

\begin{prop}
\label{prop:1.28}
Sei $(a_n)$ reelle Folge, $R$ wie in \ref{prop:1.27}. Dann ist die Potenzreihe
\begin{align*}
f(x) = \sum\limits_{n=1}^\infty a_n(x-x_0)^n,
\end{align*}
für $x\in(x_0-R,x_0+R)$ beliebig oft differenzierbar und es gilt,
\begin{align*}
f^{(k)}(x) = \sum\limits_{n=k}^\infty n(n-1)\ldots(n-k+1)
a_n(x-x_0)^{n-k}.\fishhere
\end{align*}
\end{prop}
\begin{proof}
Da $f$ für $\abs{x} < R$ gleichmäßig konvergiert, kann man Summation und
Differentiation vertauschen. Der Konvergenzradius der Ableitung stimmt mit dem
von $f(x)$ überein, denn
\begin{align*}
\limsup \sqrt[n]{n(n-1)\ldots(n-k+1) a_n} = \limsup
\sqrt[n]{a_n},
\end{align*}
da $\sqrt[n]{n}\to 1, \sqrt[n]{n-1} \to 1,$ usw.

Daraus folgt für $0<R'<R$ gleichmäßige Konvergenz für jede Ableitung im
Intervall $(x_0-R,x_0+R)$.\qedhere
\end{proof}